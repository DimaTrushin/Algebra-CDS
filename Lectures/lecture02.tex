\ProvidesFile{lecture02.tex}[Lecture 2]


\begin{claim}
\label{claim::CyclicClass}
Пусть $G$ -- некоторая группа и $g\in G$ ее элемент.
Тогда есть два возможных случая
\begin{enumerate}
\item Все элементы $g^n$ и $g^m$ различны при различных $n,m\in \mathbb Z$.

\item Существует положительное целое $n$ такое, что степени $1, g, g^2, \ldots, g^{n-1}$ различны.
Более того, степени повторяются по циклу, а именно в ряду
\[
\underbrace{\ldots, g^{-2},g^{-1}},\underbrace{1, g, g^2, \ldots, g^{n-1}}, \underbrace{g^n, g^{n+1},\ldots,g^{2n - 1}}, \underbrace{g^{2n},\ldots}\\
\]
элементы $g^{kn}, g^{1 + kn}, \ldots, g^{n-1 + nk}$ совпадают с элементами $1, g, \ldots, g^{n-1}$ для любого $k\in \mathbb Z$.
В частности, в этом случае
\[
\langle g\rangle =\{1,g,\ldots g^{n-1}\}
\]
При этом $n = \ord g$.
\end{enumerate}
\end{claim}
\begin{proof}
Если $g^n \neq g^m$ для всех различных $m, n\in \mathbb Z$, то доказывать нечего, у нас первый случай.

Давайте предположим, что для каких-то $m\neq n\in \mathbb Z$ у нас выполнено равенство $g^n = g^m$.
Можно считать, что $n > m$.
Тогда умножим обе части равенства на $g^{-m}$ и по правилам перемножения степеней получим $g^{n-m} = 1$.
Значит для некоторого $n > 0$ имеем $g^n = 1$.

Рассмотрим минимальное положительное $n$ такое, что $g^n = 1$.
Я утверждаю, что все степени $1, g, \ldots, g^{n-1}$ различны.
Действительно, если $g^k = g^s$ для некоторых $k,s \in [0, n-1]$ и $k > s$, тогда $g^{k-s} = 1$.
А это значит, что $k - s$ не ноль и строго меньше, чем $n$.
Последнее противоречит выбору $n$.

Теперь проверим, что любая степень $g^N$ совпадает со степенью из списка  $1, g, \ldots, g^{n-1}$.
Для этого поделим $N$ с остатком на $n$, получим $N = qn + r$, где $0 \leqslant r < n$.
Тогда
\[
g^N = g^{qn + r} = (g^n)^q g^r = g^r
\]
Осталось лишь заметить, что выбранное нами $n$ по определению является $\ord g$.
\end{proof}

\paragraph{Замечания}

\begin{itemize}
\item Отметим, что $n$ может быть равным $1$ в случае, когда $g$ совпадает с нейтральным элементом.

\item Из предыдущего утверждения следует, что $\ord g$ совпадает с количеством элементом в подгруппе $\langle g \rangle$.
\end{itemize}

Теперь я хочу описать все подгруппы в группе целых чисел по сложению.

\begin{claim}
\label{claim::Zsubgroups}
Всякая подгруппа $H$ группы $\mathbb Z$, точнее $(\mathbb Z, +)$, имеет вид $k\mathbb Z$ для некоторого неотрицательного целого $k$.
\end{claim}
\begin{proof}
В начале давайте проверим, что $k \mathbb Z$ действительно является подгруппой для любого $k$.
Мы должно проверить три свойства подгруппы.
Во-первых, $k\mathbb Z$ должно быть замкнуто по сложению
Но это ясно из определения.
Во-вторых, нейтральный элемент, то есть ноль, должен быть в $k\mathbb Z$.
Это так же ясно, так как $0 = k \cdot 0$.
В-третьих, для любого $m = kh \in k\mathbb Z$, его обратный $-m= k(-h)$ так же в $\mathbb Z$, и мы проверили все три свойства.

Теперь покажем, что всякая подгруппа $H$ имеет вид $k \mathbb Z$ с неотрицательным $k$.
Если $H$ содержит только нейтральный элемент $0$,  то $H = 0\mathbb Z$ и все доказано.
Предположим $H$ содержит ненулевой элемент.
Возьмем произвольное ненулевое $n \in H$.
Если $n < 0$, то $-n$ должно быть в $H$ по определению подгруппы.
А значит, мы можем считать, что $H$ содержит некоторое положительное число.
Пусть $k$ -- наименьшее положительное число в $H$.
Давайте покажем, что $H = k \mathbb Z$.

В начале покажем, $H \supseteq k \mathbb Z$.
Действительно, если $k\in H$, то по определению и вся подгруппа степеней $k$ лежит в $H$.
В аддитивной записи это значит, что
\[
mk  = \underbrace{k + \ldots + k}_m \in H\;\text{ и }\;(-n)k=\underbrace{(-k) + \ldots + (-k)}_n\in H\; \text{ для любых }\;m, n\in \mathbb N
\]
Значит, $k\mathbb Z\subseteq H$.

Теперь покажем, что $H\subseteq k\mathbb Z$.
Если $n\in H$ -- произвольный элемент, давайте разделим его на $k$ с остатком: $n = q k + r$, где $q\in \mathbb Z$ и $0 \leqslant r < k$.
Мы уже знаем, что $qk \in k\mathbb Z\subseteq H$.
Значит, $r = n - qk \in H$.
Но $r$ является неотрицательным целым числом из $H$ меньшим $k$.
Так как $k$ является минимальным положительным целым в $H$, то остается только случай $r = 0$.
А значит, $n = qk\in k\mathbb Z$ и все доказано.
\end{proof}

\begin{claim}
\label{claim::Znsubgroups}
Всякая подгруппа $H$ группы $\mathbb Z_n$, точнее $(\mathbb Z_n, +)$, имеет вид $k\mathbb Z_n =\{kh\in \mathbb Z_n \mid h\in \mathbb Z_n\}$ для некоторого положительного целого $k \mid n$.
\end{claim}
\begin{proof}
В начале проверим, что все числа кратные $k$ для $k \mid n$ образуют подгруппу в $\mathbb Z_n$.
Во-первых, покажем, что $k\mathbb Z_n$ замкнуто относительно сложения по модулю $n$.
Допустим $m_1 = k h_1$ и $m_2 = k h_2$ -- элементы $k \mathbb Z_n$.
Тогда их сумма по модулю $n$ -- это остаток $r$ такой, что $ m_1+ m_2 = r \pmod{n}$.
В этом случае
\[
r = m_1 + m_2 + q n = k h_1 + kh_2 + qn
\]
Так как $k$ делит $n$ все выражение целиком делится на $k$.
Значит и $r$ делится на $k$.
Последнее означает, что $k \mathbb Z_n$ замкнуто относительно сложения по модулю $n$.
Во-вторых,  надо проверить, что нейтральный элемент содержится в $k\mathbb Z_n$.
Это ясно из равенства $0 = k \cdot 0\in k\mathbb Z_n$.
В-третьих, если $m\in k\mathbb Z_n$ -- не нулевой элемент, то его обратный имеет вид $n - m$.
А так как $n$ делится на $k$, то и $n - m$ делится на $k$, а значит лежит в $k\mathbb Z_n$.
В случае $m = 0$ его обратный есть $0$, а он лежит в $k \mathbb Z_n$.
Потому для любого $k\mod n$, $k\mathbb Z_n$ является подгруппой в $\mathbb Z_n$.

Теперь давайте покажем, что любая подгруппа $H$ в $\mathbb Z_n$ совпадает с подгруппой вида $k\mathbb Z_n$ где $k\mid n$.
Подгруппа $H$ должна содержать нейтральный элемент  $0$.
Если больше нет других элементов в $H$, то $H = \{0\} = n \mathbb Z_n$ и все доказано.
Значит, мы можем предположить, что в $H$ есть ненулевые элементы.
Пусть $k$ -- наименьший положительный элемент в $H$.
По определению циклическая подгруппа $k\mathbb Z_n$ лежит в $H$.
Потому нам надо показать только обратное включение $H\subseteq k \mathbb Z_n$ и показать, что $k$ делит $n$.

В начале покажем, что $k$ делит $n$.
Давайте разделим $n$ с остатком на $k$, мы получим $n = qk + r$, где $0\leqslant r < k$.
Теперь, $r = n - q k$, а значит $r = -q k \pmod{n}$.
Так как $k\in H$, последнее означает, что $r$ тоже в $H$.
Но это противоречит выбору $k$, оно было наименьшим положительным целым в $H$.
Значит, $r$ должен быть нулем, а это и означает, что $k$ делит $n$.
Теперь, давайте покажем, что каждый элемент $H$ лежит в $k\mathbb Z_n$.
Возьмем произвольный элемент $h\in H$.
Разделим его на $k$ с остатком и получим $h = q k + r$.
Значит, $r = h - q k$.
Так как $h\in H$ и $k\in H$, все выражение $h - qk$ лежит в $H$, то есть $r\in H$.
Так как $k$ был наименьшим положительным целым в $H$, получается, что $r = 0$.
Последнее означает, что $h$ делится на $k$, то есть лежит в $k\mathbb Z_n$.
\end{proof}

\subsection{Смежные классы}

Алгебра тяготеет к тому, чтобы изучать группы с помощью подгрупп, а не только элементов.
Важным инструментом в таком подходе являются смежные классы.

\begin{definition}
Пусть $G$ -- некоторая группа, $H\subseteq G$ -- ее подгруппа и $g\in G$ -- произвольный элемент.
Тогда множество
\[
gH = \{gh\mid h\in H\}
\]
называется левым смежным классом элемента $g$ по подгруппе $H$.
Аналогично определяются правые смежные классы.
Множество
\[
Hg = \{hg\mid h\in H\}
\]
называется правым смежным классом элемента $g$ по подгруппе $H$.
\end{definition}

\paragraph{Замечания}

\begin{enumerate}
\item Стоит заметить, что если $G$ коммутативна, то нет разницы между левыми и правыми смежными классами для любой подгруппы $H\subseteq G$.

\item Сама подгруппа $H$ является левым и правым смежным классом.
Действительно, $H = 1 \cdot H = H \cdot 1$.

\item В произвольной группе, вообще говоря левый смежный класс $gH$ не обязан равняться правому смежному классу $Hg$ как показывает пример ниже.
\end{enumerate}


\begin{examples}
Некоторые примеры смежных классов.
\begin{enumerate}
\item Пусть $G = (\mathbb Z, +)$ и $H = 2\mathbb Z$  -- подгруппа четных целых чисел.
Тогда $2\mathbb Z$ и $1 + 2\mathbb Z$ -- все возможные смежные классы $H$.

\item Пусть $G = S_3$ -- группа перестановок на трех элементах и $H = \langle (1, 2)\rangle$ -- циклическая подгруппа порожденная элементом $(1,2)$.
Мы можем перечислить все элементы $G$ и $H$
\[
G = \{1, (1,2), (1, 3), (2, 3), (1, 2, 3), (3, 2, 1)\},\; H = \{1, (1, 2)\}
\]
Теперь мы видим, что есть три разных левых смежных класса $H$
\[
H = \{1, (1, 2)\}, \;(1,3)H = \{(1, 3), (1, 2, 3)\},\;(2, 3)H = \{(2, 3), (3,2,1)\}
\]
А так же, три разных правых смежных класса $H$
\[
H =  \{1, (1, 2)\}, \; H(1, 3) = \{(1, 3), (3, 2, 1)\},\; H (2, 3) = \{(2, 3), (1, 2, 3)\}
\]
Этот пример показывает, что $(1, 3) H \neq H (1, 3)$.
Так же этот пример показывает, что
\[
(1, 2)H = H,\; (1, 3)H = (1, 2, 3)H,\; (2, 3)H = (3, 2, 1)H
\]
То есть одинаковые смежные классы могут порождаться разными элементами.

\item Пусть $G = S_n$ -- группа перестановок на $n$ элементах и $H = A_n$ -- подгруппа четных перестановок.
Тогда для всякой четной перестановки$\sigma\in A_n$, множество $\sigma A_n$ состоит из всех четных перестановок.
Аналогично, для всякой нечетной перестановки $\sigma\in S_n\subseteq A_n$, множество $\sigma A_n$ состоит из всех нечетных перестановок.
Потому, есть всего два левых смежных класса по$A_n$, это
\[
A_n\;\text{и}\;(1, 2) A_n
\]
Аналогично мы можем заметить, что есть всего два правых смежных класса по $A_n$, это
\[
A_n\;\text{и}\; A_n(1, 2)
\]
Более того, мы видим, что $\sigma A_n = A_n \sigma$ для всех $\sigma \in S_n$.
\end{enumerate}
\end{examples}

\begin{definition}
Пусть $G$ группа и $H$ ее подгруппа.
Подгруппа $H$ называется нормальной если $gH = Hg$ для любого элемента $g\in G$.
\end{definition}

\begin{claim}
\label{claim::normal_crit}
Пусть $G$ -- некоторая группа и $H$ -- ее подгруппа.
Следующие условия эквивалентны:
\begin{enumerate}
\item $gH = Hg$ для любого $g\in G$.

\item $gHg^{-1} = H$ для любого $g\in G$.

\item $gHg^{-1}\subseteq H$ для любого $g\in G$.
\end{enumerate}
\end{claim}
\begin{proof}
(1)$\Leftrightarrow$(2).
Предположим $gH = Hg$.
Умножая это равенство справа на $g^{-1}$, мы получаем $gH g^{-1} = H$.
А если нам задано равенство $g H g^{-1} = H$, умножая его справа на $g$, мы получим $gH = Hg$.

(2)$\Leftrightarrow$(3).
Надо проверить, что если выполнено $gHg^{-1}\subseteq H$ для любого $g\in G$, то и $gHg^{-1} = H$ выполнено для любого $g\in G$.
Если  $gHg^{-1}\subseteq H$ для любого $g\in G$, то оно выполнено и для $g^{-1}$ вместо $g$.
Значит, $g^{-1}Hg \subseteq H$ для любого $g\in G$.
Умножим это равенство слева на $g$, получим $Hg \subseteq gH$.
Теперь умножим это равенство справа на $g^{-1}$ и получим $H \subseteq gH g^{-1}$.
А это завершает доказательство.
\end{proof}

\subsection{Теорема Лагранжа}

\paragraph{Свойства смежных классов}

Прежде всего я хочу доказать некоторые свойства смежных классов.
Так окажется, что левые смежные классы образуют разбиение группы $G$ на не пересекающиеся множества одного размера.
Аналогичное верно и для правых смежных классов.
Подобное утверждение позволяет применить к изучению группы комбинаторные соображения.

\begin{claim}
\label{claim::cosets_disj}
Пусть $G$ -- некоторая группа, $H\subseteq G$ -- ее подгруппа и $g_1, g_2\in G$ -- произвольные элементы.
Тогда возможны только два случая:
\begin{enumerate}
\item Смежные классы не пересекаются: $g_1 H \cap g_2 H = \varnothing$.

\item Смежные классы совпадают: $g_1 H = g_2 H$.
\end{enumerate}
Последнее означает, что каждый элемент группы $G$ лежит в единственном смежном классе.
\end{claim}
\begin{proof}
Если $g_1H$ не пересекает $g_2H$, то доказывать нечего.

Предположим, что пересечение смежных классов $g_1 H\cap g_2 H$ не пусто.
Мы должны доказать, что $g_1 H = g_2 H$.
Предположим, что $g\in g_1 H \cap g_2H$.
Тогда $g\in g_1H$, $g = g_1 h_1$ для некоторого $h_1\in H$.
Аналогично, $g\in g_2H$ влечет $g = g_2 h_2$ для некоторого $h_2\in H$.
Значит $g_1 h_1 = g_2 h_2$.
Разделив на $h_1$ справа, мы получим $g_1 = g_2 h_2 h_1^{-1}$.
Так как $H$ является подгруппой, то $h = h_2 h_1^{-1}\in H$.
То есть $g_1 = g_2 h$  для некоторого $h\in H$.

Давайте покажем, что $g_1 H \subseteq g_2 H$.
Предположим, что произвольный элемент $g\in g_1H$ имеет вид $g = g_1 h'$, где $h'\in H$.
Тогда $g =g_2 h h'\in g_2 H$ потому что $hh'\in H$.
Аналогично показывается обратное вложение $g_2 H \subseteq g_1 H$.
А именно, возьмем $g\in g_2H$ в виде $g = g_2 h'$ где $h'\in H$.
Значит, $g = g_1 h^{-1}h'\in g_1 H$ потому что $h^{-1}h'\in H$.
\end{proof}

\begin{remark}
Обратим внимание, что $g_1 H  = g_2H$ тогда и только тогда, когда $g_1 H \cap g_2 H \neq \varnothing$.
Более того, это происходит тогда и только тогда, когда найдется элемент $h\in H$ такой, что $g_1 = g_2 h$.
Последнее эквивалентно условию $g_2^{-1}g_1 \in H$.
Это дает нам удобный способ проверять являются ли смежные классы одинаковыми.
\end{remark}

\begin{claim}
\label{claim::cosets_size}
Пусть $G$ -- некоторая группа, $H\subseteq G$ -- конечная подгруппа и $g\in G$ -- некоторый элемент.
Тогда $|gH| = |H| = |Hg|$.
\end{claim}
\begin{proof}
Я докажу утверждение для левых смежных классов.
Для правых делается аналогично.
Рассмотрим отображение
\[
\phi \colon H \to g H\quad x \mapsto gx
\]
Оно переводит элементы $H$ в элементы $gH$.
С другой стороны, существует обратное отображение
\[
\psi \colon gH \to H\quad x \mapsto g^{-1}x
\]
Поэтому $\phi$ и $\psi$ являются взаимно обратными биекциями.
\end{proof}

\begin{claim}
\label{claim::cosets_l_r_same}
Пусть $G$ -- конечная группа и $H\subseteq G$ -- ее подгруппа.
Тогда
\begin{enumerate}
\item Количество левых смежных классов группы $H$ равно $|G|/|H|$.

\item Количество правых смежных классов группы $H$ равно $|G|/|H|$.
\end{enumerate}
В частности, количество левых и правых смежных классов одно и то же.
\end{claim}
\begin{proof}
Я докажу первое равенство для левых смежных классов.
Утверждение~\ref{claim::cosets_disj} показывает, что  $G$ является дизъюнктным объединением своих смежных классов, то есть $G = g_1 H \sqcup \ldots \sqcup g_k H$.
С другой стороны, утверждение~\ref{claim::cosets_size} говорит, что все смежные классы $g_1H,\ldots, g_kH$ имеют один и тот же размер $|H|$.
Значит
\[
|G| = |g_1H| + \ldots +|g_k H| = |H| + \ldots + |H| = k |H|
\]
Здесь $k$ -- это число различных левых смежных классов.
\end{proof}

\begin{definition}
Пусть $G$ -- конечная группа и $H\subseteq G$ -- некоторая ее подгруппа.
Тогда количество левых смежных классов $H$ называется индексом $H$ и обозначается $(G:H)$.
Это число так же совпадает с количеством правых смежных классов.
\end{definition}

Используя это определение, мы можем переписать утверждение~\ref{claim::cosets_l_r_same} следующим образом.

\begin{claim}
[Теорема Лагранжа]
Пусть $G$ -- конечная группа и $H\subseteq G$ -- некоторая ее подгруппа.
Тогда, $|G| = (G : H)|H|$
\end{claim}


\paragraph{Следствия теоремы Лагранжа}

\begin{enumerate}
\item Пусть $G$ -- конечная группа и $H\subseteq G$ -- ее некоторая подгруппа.
Тогда $|H|$ делит $|G|$.

\item Пусть $G$ -- конечная группа и $g\in G$ -- произвольный элемент.
Тогда $\ord(g)$ делит $|G|$.
Действительно, $\ord(g) = |\langle g \rangle|$ по утверждению~\ref{claim::CyclicClass}.
Но $|\langle g\rangle|$ делит $|G|$ по предыдущему пункту.

\item Пусть $G$ -- конечная группа и $g\in G$ -- некоторый элемент.
Тогда $g^{|G|} = 1$.
Действительно, мы уже знаем, что $|G| = \ord(g) k$.
Значит, 
\[
g^{|G|} = g^{\ord(g)k} = \left(g^{\ord(g)}\right)^{k} = 1^k = 1
\]

\item Пусть $G$ -- группа простого порядка.
Тогда $G$ циклическая.
Действительно, так как порядок $G$ прост, то он больше $1$.
Значит, существует элемент $g\in G$ такой, что $g\neq 1$.
Тогда подгруппа $\langle g\rangle$ имеет порядок больше $1$.
Но $|\langle g \rangle|$ делит $|G| = p$.
Так как $p$ простое, единственно возможный случай -- это $|\langle g \rangle| = p = |G|$.
Последнее означает, что $\langle g \rangle = G$ и все доказано.

\item Малая теорема Ферма.
Пусть $p\in \mathbb Z$ -- простое число и $a\in \mathbb Z$.
Если $p$ не делит $a$,  то $p$ делит $a^{p-1}-1$.
Действительно, давайте рассмотрим группу $(\mathbb Z_p^*, \cdot)$.
Для любого элемента $b\in \mathbb Z_p^*$, имеем $b^{|\mathbb Z_p^*|} = 1 \pmod p$ по пункту~(3).
Но $\mathbb Z_p^*$ состоит из $p-1$ элемента.
Теперь возьмем произвольное $a\in \mathbb Z$ взаимно простое с $p$.
Пусть $b$ -- остаток от деления $a$ на $p$.
Тогда $a^{p-1} = b^{p-1} = 1\pmod p$ и все доказано.
\end{enumerate}
