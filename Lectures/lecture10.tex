\ProvidesFile{lecture10.tex}[Lecture 10]


\subsection{Критерий Бухбергера}

Смотрите какая у нас есть  скрытая проблема.
А существуют ли базисы Грёбнера вообще?
Или по-другому, как проверить является ли множество $G$ базисом Грёбнера?
Подобные вопросы решаются с помощью критерия Бухбергера.
Я отложу доказательства критерия до следующей лекции, а сейчас хочу обсудить его формулировку и приложения.

\begin{definition}
Пусть $F$ -- поле, $f_1,f_2\in F[x_1,\ldots,x_n]$ -- некоторые ненулевые многочлен, и мы зафиксировали лексикографический порядок на мономах.
Предположим, что $f_1 = c_1 m_1 + f'_1$, где $c_1m_1$ -- старший член, и $f_2 = c_2 m_2 + f'_2$, где $c_2m_2$ -- старший член.
Пусть $m$ -- наименьшее общее кратное двух мономов  $m_1$ и $m_2$, тогда $m = m_1 t_1 = m_2 t_2$.
В этом случае многочлен
\[
S_{f_1\,f_2} = c_2t_1 f_1 - c_1 t_2 f_2 = c_2t_1 f'_1 - c_1 t_2 f'_2
\]
называется S-многочленом $f_1$ и $f_2$.
\end{definition}

Идея S-многочлена состоит в том, что мы хотим сократить старшие члены $f_1$ и $f_2$ наиболее экономным способом.

\begin{example}
\label{example::SPoly}
Рассмотрим $\mathbb Q[x, y, z]$, $f_1 = xy - 1$, $f_2 = yz - 1$, и $\operatorname{Lex}(x, y, z)$ как порядок.
Тогда старшие члены будут $m_1 = xy$ и $m_2 = yz$, и НОК старших членов будет $xyz = m_1 z = m_2 x$.
Тогда
\[
S_{f_1\,f_2} = z f_1 - x f_2 = z(xy - 1) - x(yz - 1) = x-z
\]
\end{example}


\begin{claim}
[Критерий Бухбергера]
Предположим $F$ -- некоторое поле, мы зафиксировали лексикографический порядок на мономах от $n$ переменных и $G\subseteq F[x_1,\ldots,x_n]\setminus\{0\}$ -- некоторое множество ненулевых многочленов.
Тогда следующие условия эквивалентны:
\begin{enumerate}
\item $G$ является базисом Грёбнера.

\item Для любых $g_1,g_2\in G$, $S_{g_1\,g_2}$ редуцируется к нулю с помощью любой редукции по множеству $G$.

\item Для любых $g_1,g_2\in G$, $S_{g_1\,g_2}$ редуцируется к нулю с помощью какой-то редукции по множеству $G$.
\end{enumerate}
\end{claim}

\begin{remark}
В примере~\ref{example::SPoly}, мы видим, что S-полином от $f_1$ и $f_2$ не редуцируется к нулю относительно множества $\{f_1, f_2\}$.
Значит это множество не является базисом Грёбнера.
\end{remark}

Существует специальный случай, когда мы можем гарантировать, что S-полином автоматически редуцируется к нулю.%
\footnote{На самом деле есть еще один хитрый трюк, но я его упоминать не буду.}

\begin{claim}
Пусть $F$ -- поле, $g_1,g_2\in F[x_1,\ldots,x_n]\setminus\{0\}$, и мы фиксировали какой-то лексикографический порядок.
Предположим, что старшие мономы $g_1$ и $g_2$ взаимно просты.
Тогда, $S_{g_1\,g_2}$ редуцируется к нулю с помощь множества $\{g_1,g_2\}$.
\end{claim}
\begin{proof}
Пусть $g_1 = c_1 m_1 + g_1'$ и $g_2 = c_2 m_2 + g_2'$, где $m_i$ -- старший моном в $g_i$, $c_i$ -- старший коэффициент в $g_i$ и $g_i'$ -- хвост $g_i$.
Так как $m_1$ и $m_2$ взаимно просты, то S-многочлен от $g_1$ и $g_2$ имеет вид
\[
S_{g_1\,g_2} = c_2m_2 g_1 - c_1 m_1 g_2 = c_2m_2 (c_1 m_1 + g_1') - c_1 m_1 (c_2 m_2 + g_2') = c_2m_2 g_1' - c_1 m_1 g_2'
\]
Теперь используем равенство $c_i m_i = g_i - g_i'$ и перепишем S-полином в следующем виде
\[
S_{g_1\,g_2} = c_2m_2 g_1' - c_1 m_1 g_2' = (g_2 - g_2')g_1' - (g_1 - g_1')g_2' = g_2' g_1 - g_1'g_2
\]
То есть мы выразили S-полином в виде $a g_1 + b g_2$, при этом все мономы в $a$ строго младше, чем $m_2$, а все мономы в $b$ строго младше, чем $m_1$.

Теперь наша цель показать, что многочлен представленный в виде $a g_1 + b g_2$ либо является нулевым, либо мы можем сделать хотя бы один шаг редукции с помощью множества $\{g_1, g_2\}$.
Чтобы доказать это, мы покажем, что старший моном $M(a g_1 + b g_2)$ либо совпадает с $M(a g_1)$ либо с $M(bg_2)$, то есть старший моном суммы будет старшим мономом одного из слагаемых и не сократится.
Если $M(a g_1)> M(bg_2)$, то $M(a g_1)$ будет старшим мономом $a g_1 + b g_2$.
Если $M(a g_1) < M(bg_2)$, то $M(bg_2)$ является старшим мономом $a g_1 + b g_2$.
Теперь нам осталось рассмотреть случай $M(a g_1) = M(bg_2)$.
Однако, в этом случае $m_1 = M(g_1)$ делит $M(a g_1)$ и $m_2 = M(g_2)$ делит $M(b g_2)$.
Так как $m_1$ и $m_2$ являются взаимно простыми, то  $m_1 m_2$ делит $M(a g_1) = M(a) M(g_1) = M(a) m_1$.
Значит $m_2$ делит $M(a)$.
Но это противоречит тому, что все мономы в $a$ строго меньше, чем $m_2$.
Значит такого не может быть, что старшие члены обоих слагаемых совпадут и все доказано.

Теперь предположим, что $S_{g_1, g_2} = a g_1 + b g_2$ не является нулем.
Тогда старший моном будет либо $M(a) m_1$ либо $M(b) m_2$.
В любом случае мы можем сделать редукцию относительно $g_1$ или $g_2$.
Без потери общности давайте считать, что старший моном будет $M(a) m_1$.
Давайте явно отредуцируем старший моном с помощью $g_1$.
\[
S_{g_1, g_2}\stackrel{g_1}{\longrightarrow} a g_1 + b g_2 - C(a) M(a) g_1 = (a - C(a) M(a)) g_1 + b g_2
\]
Как мы видим в результате редукции мы уничтожили старший член в $a$, а значит мы получили многочлен такого же вида $a g_1 + b g_2$ с условиями, что все мономы $a$ строго младше $m_2$ и все мономы $b$ строго младше $m_1$.
А значит мы можем продолжать этот процесс редукции.
Так как по утверждению~\ref{claim::ReductFin} редукция должна остановиться, то мы придем к нулевому многочлену (иначе можно было бы продолжить редукцию).
\end{proof}

\begin{example}
Пусть $\mathbb Q[x, y, z]$ с порядком $\operatorname{Lex}(x, y, z)$, $g_1 = x^2 y - z$, и $g_2 = z^2 + 1$.
Тогда старшие мономы $M(g_1) = x^2 y$ и $M(g_2) = z^2$ будут взаимно простыми.
Значит их S-полином редуцируется к нулю с помощью $\{g_1,g_2\}$.
Мы можем проверить этот факт руками, это не сложно.
Но куда проще воспользоваться утверждением и полностью избежать вычислением.
\end{example}


\subsection{Идеалы в кольце многочленов}

Так оказывается, что большинство проблем связанных с многочленами можно решить используя идеалы.
Потому очень важно уметь эффективно решать задачи связанные с идеалами.
Давайте немного обсудим как устроены идеалы в кольце многочленов от нескольких переменных.

\begin{definition}
Пусть $F$ -- поле и у нас задано конечное множество многочленов $g_1,\ldots,g_k\in F[x_1,\ldots,x_n]$.
тогда множество
\[
(g_1,\ldots,g_k) = \{g_1 h_1 + \ldots + g_k h_k\mid h_1,\ldots,h_k\in F[x_1,\ldots,x_n]\}
\]
является идеалом в $F[x_1,\ldots,x_n]$ и называется идеалом порожденным многочленами $g_1,\ldots,g_k$.
Если $G = \{g_1,\ldots,g_k\}$, то идеал $(g_1,\ldots,g_k)$ так же обозначается $(G)$ для краткости.
\end{definition}

Оказывается, что в кольце многочленов от нескольких переменных любой идеал в кольце $F[x_1,\ldots,x_n]$ порожден конечным множеством.
Этот результат был получен Гильбертом и известен как Теорема Гильберта о базисе.

\begin{claim}
[Теорема Гильберта о базисе]
Пусть $F$ -- поле и $I\subseteq F[x_1,\ldots,x_n]$ -- некоторый идеал.
Тогда существуют многочлены $g_1,\ldots,g_k\in F[x_1,\ldots,x_n]$ такие, что $I = (g_1,\ldots,g_k)$.
\end{claim}

Оказывается, что если мы хотим эффективно решать задачи связанные с идеалами, то мы хотим уметь порождать эти идеалы с помощью базиса Грёбнера.
Потому нам надо уметь решать следующую проблему:
Пусть идеал $I$ порожден многочленами $g_1,\ldots,g_k$ и нам задан лексикографический порядок.
Так может оказаться, что множество $\{g_1,\ldots,g_k\}$ вообще говоря не базис Грёбнера.
Потому мы хотим заменить множество порождающих на новое множество $G$ такое, что оно так же порождает идеал $I$, то есть $I = (G)$ и при этом $G$ является базисом Грёбнера.
В таком случае будем говорить, что $G$ -- это базис Грёбнера идеала $I$.
Кроме того, мы хотим чтобы множество $G$ было конечно для того, чтобы можно было эффективно что-то посчитать.
Последняя проблема решается алгоритмом Бухбергера.

\paragraph{Алгоритм Бухбергера}

Как обычно, нам дано поле $F$, кольцо многочленов $F[x_1,\ldots,x_n]$, и зафиксирован некоторый лексикографический порядок на мономах.

\paragraph{Вход}

Конечное множество многочленов $G = \{g_1,\ldots,g_k\}\subseteq F[x_1,\ldots,x_n]$.

\paragraph{Выход}

Конечное множество многочленов $G_0\subseteq F[x_1,\ldots,x_n]$ таких, что
\begin{enumerate}
\item $G_0$ является базисом Грёбнера.

\item $(G) = (G_0)$.
\end{enumerate}

\paragraph{Алгоритм}

\begin{enumerate}
\item В начале инициализируем $G_0 = G$.

\item Для каждой пары $g_i,g_j\in G_0$ мы считаем $S_{g_i\,g_j}$ и редуцируем его с помощью $G_0$ к остатку: $S_{g_i\,g_j}\stackrel{G_0}{\rightsquigarrow}r_{ij}$.

\item Если все $r_{ij}$ нули, то $G_0$ является базисом Грёбнера по критерию Бухбергера.
Если же есть ненулевые остатки, то мы обновляем $G_0$ следующим образом $G_0\cup \{r_{ij}\mid r_{ij} \neq 0\}$ и повторяем алгоритм с шага~(2).
\end{enumerate}

Если мы хотим показать, что алгоритм работает, мы должны показать несколько вещей: 1) алгоритм останавливается за конечное число шагов, 2) множество $G_0$ конечное, 3) полученное множество $G_0$ является базисом Грёбнера, 4) множества $G$ и $G_0$ порождают один и тот же идеал.
Самое сложное -- доказать остановку алгоритма.
Я отложу этот вопрос на будущее.
Давайте объясним остальные пункты.

2) Мы изначально инициализируем $G_0$ конечным множеством.
После этого на каждом шаге алгоритма мы добавляем лишь конечное число элементов к $G_0$.
Значит после конечного числа шагов $G_0$ будет конечным.

3) Заметим, что алгоритм останавливается тогда и только тогда, когда все S-полиномы для $G_0$ редуцируются к нулю.
Что по критерию Бухбергера означает, что $G_0$ является базисом Грёбнера.

4) Надо показать, что $(G) = (G_n)$.
Достаточно показать, что на каждом шаге $G_0$ и $G_0 \cup \{r_{ij}\}$ порождают один и тот же идеал.
Так как второе множество включает первое, то достаточно показать, что $(G_0 \cup \{r_{ij}\}) \subseteq (G_0)$.
А для этого достаточно показать, что каждый остаток $r_{ij}\in (G_0)$.
Действительно, для каждой пары $g_i,g_j\in G_0$ S-полином $S_{g_i\,g_j}$ имеет вид $c_2t_1 g_i - c_1 t_2 g_j\in (G_0)$ по определению.
Теперь рассмотрим процесс редукции, $S_{g_i\,g_j}\stackrel{G_0}{\rightsquigarrow}r_{ij}$.
Это значит, что существует цепочка элементарных редукций: $S_{g_i\,g_j}\stackrel{g_1'}{\longrightarrow}f_1\stackrel{g_2'}{\longrightarrow}f_2\stackrel{g_3'}{\longrightarrow}\ldots\stackrel{g_k'}{\longrightarrow}f_k = r_{ij}$.
Потому нам лишь надо показать, что если $f\in (G_0)$, $g\in G_0$, и $f\stackrel{g}{\longrightarrow}f'$, то $f'\in (G_0)$.
Но это ясно по определению, так как $f' = f - \lambda t g$ для некоторого $\lambda \in F$ и некоторого монома $t$.

\subsection{Задача принадлежности идеалу и исключение переменных}

Давайте обсудим парочку задач, которые можно решать с помощью базисов Грёбнера.

\paragraph{Принадлежность идеалу}

Пусть у нас дано поле $F$, кольцо многочленов $F[x_1,\ldots,x_n]$, идеал $I = (g_1,\ldots,g_k)$ в кольце многочленов и некоторый многочлен $f$.
Вопрос в том, как проверить принадлежит ли многочлен $f$ идеалу $I$.
Вот как можно решать эту задачу:
\begin{enumerate}
\item Зафиксируем какой-нибудь лексикографический порядок $\Lex$.

\item Найдем базис Грёбнера $G$ для идеала $I$.

\item Посчитаем остаток $f$ относительно $G$, то есть $f\stackrel{G}{\rightsquigarrow}r$.

\item Многочлен $f$ лежит в $I$ тогда и только тогда, когда $r = 0$.
\end{enumerate}

Надо отметить, что если $G$ не является базисом Грёбнера, то алгоритм выше <<работает только наполовину>>.
А именно, если мы получили, что остаток $r = 0$, то это гарантирует, что $f$ лежит в идеале $I$.
Однако, если $r \neq 0$ и $G$ -- не базис Грёбнера, то отсюда еще не следует, что $f$ не лежит в идеале.
Действительно, в примере~\ref{example::SPoly} S-полином $S_{f_1\,f_2}$ принадлежит идеалу $(f_1, f_2)$.
Однако, $S_{f_1\,f_2}$ вообще не редуцируется с помощью множества $\{f_1,f_2\}$.

\paragraph{Исключение переменных}

Пусть нам дано поле $F$, кольцо многочленов $F[x_1,\ldots,x_n, x_{k+1}, \ldots, x_n]$, и идеал $I = (g_1,\ldots,g_k)$ в кольце многочленов.
Теперь вопрос: как посчитать $I\cap F[x_1,\ldots,x_k]$.
Это значит, что мы хотим исключить все многочлены из $I$, которые зависят от переменных $x_{k+1},\ldots,x_n$.
Прежде чем решать задачу, отметим, что множество  $I\cap F[x_1,\ldots,x_k]$ является идеалом в кольце $F[x_1,\ldots,x_k]$.
Потому в ответе мы бы хотели задать этот идеал с помощью образующих.
Оказывается можно сразу найти базис Грёбнера этого идеала.
Вот как можно это сделать.
\begin{enumerate}
\item  Зафиксируем такой лексикографический порядок, чтобы $x_1,\ldots,x_k < x_{k+1},\ldots,x_n$.
порядок между $x_1,\ldots,x_k$ или между $x_{k+1},\ldots,x_n$ не имеет значения, главное, чтобы переменные, которые мы хотим оставить были младше тех, которые мы хотим исключить.

\item Вычислим базис Грёбнера $G_0$ для идеала $I$ относительно этого порядка.

\item Положим $G = \{g\in G_0\mid g \text{ не зависит от } x_{k+1},\ldots,x_n\}$.

\item Тогда $G$ является базисом Грёбнра для $I\cap F[x_1,\ldots,x_k]$ относительно ограничения выбранного лексикографического порядка на переменные $x_1,\ldots,x_k$.
\end{enumerate}

Теперь я хочу обсудить некоторую полезную вспомогательную лемму.

\begin{claim}
\label{claim::ReductionProps}
Пусть $F$ -- поле, рассмотрим кольцо многочленов  $F[x_1,\ldots,x_n]$ и фиксируем лексикографический порядок на мономах.
\begin{enumerate}
\item Если $f\stackrel{G}{\rightsquigarrow}f'$ и $m$ -- некоторый моном, то $mf \stackrel{G}{\rightsquigarrow}mf'$.

\item Если $f_1 - f_2 \stackrel{G}{\rightsquigarrow}0$, тогда $f_1$ и $f_2$ можно редуцировать к одному и тому же остатку с помощью $G$, то есть существует нередуцируемый с помощью $G$ многочлен $r$ такой, что $f_1\stackrel{G}{\rightsquigarrow}r$ и $f_2\stackrel{G}{\rightsquigarrow}r$.
\end{enumerate}
\end{claim}
\begin{proof}
1) Если $f\stackrel{G}{\rightsquigarrow}f'$, то это значит, что у нас есть цепочка элементарных редукций вида
\[
f\stackrel{g_1}{\longrightarrow}f_1 \stackrel{g_2}{\longrightarrow}\ldots  \stackrel{g_k}{\longrightarrow}f_k = f'
\]
В частности $f_{i+1} = f_i - c_i t_i g_{i+1}$.
Если умножить все эти равенства на $m$ мы получим цепочку редукций
\[
mf\stackrel{g_1}{\longrightarrow}mf_1 \stackrel{g_2}{\longrightarrow}\ldots  \stackrel{g_k}{\longrightarrow}mf_k =m f'
\]
что и требовалось.

2) Пусть $f_1 - f_2 \stackrel{g_1}{\longrightarrow}\varphi_1 \stackrel{g_2}{\longrightarrow}\ldots  \stackrel{g_k}{\longrightarrow}\varphi_k = 0$.
Мы покажем результат индукцией по $k$.

База индукции $k = 0$.
В этом случае $f_1 - f_2 = 0$, значит $f_1 = f_2$ и они точно редуцируются к одному и тому же остатку.
Пусть теперь $k > 0$.
Посмотрим на первый шаг редукции $f_1 - f_2 \stackrel{g}{\longrightarrow}\varphi_1$.
Пусть в этом шаге мы редуцировали моном $m$ из разности $f_1 - f_2$.
Этот моном взялся либо из $f_1$ либо из $f_2$ либо из обоих многочленов и не сократился.
В любом случае можно выделить этот моном в $f_1$ и в $f_2$: $f_1 = a m + h_1$ и $f_2 = bm + h_2$, где $a,b\in F$ моном $m$ не присутствует в $h_1$ и $h_2$  и при этом $a \neq b$.
Обратим внимание кто-то один из $a$ или $b$ может оказаться нулем.
Пусть $g = c_g m_g + g_0$, где $c_g$ -- старший коэффициент и $m_g$ -- старший моном и пусть $m = t m_g$.
Тогда редукция разности выполняется по следующей формуле
\[
f_1 - f_2 \stackrel{g}{\longrightarrow}(a-b)m + h_1 -h_2 - \frac{a-b}{c_g}t g = \varphi_1
\]
Давайте перепишем это следующим образом
\[
\varphi_1 = f_1 - \frac{a}{c_g}t g - \left(f_2 - \frac{b}{c_g}t g\right)
\]
Если $a$ не равно нулю, то есть если моном $m$ присутствует в $f_1$, тогда выражение $f_1 - \frac{a}{c_g}t g$ будет редукцией монома $m$ в $f_1$ с помощью $g$.
А если $a = 0$, то это выражение просто равно $f_1$.
Аналогично для $f_2$.
Теперь зададим
\[
f_1'=f_1 - \frac{a}{c_g}tg,\quad f_2'=f_2 - \frac{b}{c_g}tg
\]
Мы видим, что либо $f_i = f_i'$ либо $f_i$ редуцирутся элементарной редукцией к $f_i'$.
При этом по построению $\varphi_1 = f_1' - f_2'$.
А значит по индукции $f_1'$ и $f_2'$ можно отредуцировать к одному и тому же остатку.
Так как $f_i$ можно отредуцировать к $f_i'$, то результат доказан.
\end{proof}
