\ProvidesFile{lecture09.tex}[Lecture 9]


\section{Базисы Грёбнера}

Сейчас я хочу поговорить о многочленах от нескольких переменных.
Одна из причин почему мы так любим многочлены от одной переменной -- Евклидов алгоритм деления с остатком.
Оказывается, что многие задачи эффективно решаются с помощью деления с остатком.
Однако, когда мы переходим к многочленам от нескольких переменных у нас уже не существует Евклидова алгоритма деления (если дать строгие определения, то можно даже доказать, что в многочленах от нескольких переменных не может быть деления с остатком).
Что же делать в такой тяжелой и грустной ситуации?
Предлагается вместо честного деления с остатком ввести очень похожую процедуру называемую редукцией.
Более того с помощью такой процедуры можно будет редуцировать многочлен относительно целого семейства многочленов и получать в результате аналоги остатков.
Основная проблема будет заключаться в том, что если редуцировать по <<плохому>> множеству, то остатки будут получаться не единственными.
В результате среди всех семейств выделяют только <<хорошие>>, которые дают корректно определенные остатки.
Такие хорошие множества и называются базисами Грёбнера.

\subsection{Многочлены от нескольких переменных}

Пусть $F$ -- некоторое поле.
Многочлен $f$ от переменных  $x_1,\ldots,x_n$ с коэффициентами в $F$ -- это картинка следующего вида
\[
f = \sum_{k_1,\ldots,k_n\geqslant 0} a_{k_1\ldots k_n} x_1^{k_1} \ldots x_n^{k_n},\quad a_{k_1\ldots k_n} \in F
\]
здесь только конечное число коэффициентов отлично от нуля.
Сложение и умножение многочленов заданы обычными формулами.
Множество всех многочленов от переменных $x_1,\ldots,x_n$ с коэффициентами в поле $F$ будет обозначаться $F[x_1,\ldots,x_n]$.
Множество $F[x_1,\ldots,x_n]$ с операциями сложения и умножения является коммутативным кольцом.

Выражение $m = x_1^{k_1}\ldots x_n^{k_n}$ называеся мономом.
Степень монома -- $\deg m = k_1+ \ldots + k_n$.
Степень многочлена $f$ -- это максимум степеней всех мономов в него входящих с ненулевыми коэффициентами.
Степень нулевого многочлена считается равной $-\infty$ так же как и в случае одной переменной.
Прямое вычисление показывает, что если $f,g\in F[x_1,\ldots,x_n]$, то $\deg(fg) = \deg f + \deg g$.
Моном умноженный на коэффициент из поля $F$ называется термом.
Таким образом любой многочлен всегда является какой-то конечной суммой термов.

\subsection{Мономиальные порядки}

Прежде всего нам нужен способ упорядочить мономы.
В случае многочленов от одной переменной, мы могли эффективно сравнивать мономы по степени единственной переменной.
В случае нескольких переменных требуется какая-то новая процедура, позволяющая сказать, какой моном в многочлене старший, а какой младший.
Существует огромное количество <<хороших>> порядков на мономах.
Однако, чтобы упростить изложение, я собираюсь обсудить лишь самый полезный для наших целей -- лексикографический порядок.

\begin{definition}
Мы хотим определить лексикографический порядок на мономах.
\begin{enumerate}
\item Прежде всего надо зафиксировать порядок переменных $x_1,\ldots,x_n$.
Пусть для определенности порядок будет $x_1 > x_2 > \ldots > x_n$.
Тем не менее, мы можем взять любой порядок на переменных.

\item Теперь мы предполагаем порядок $x_1 > \ldots > x_n$ зафиксированным.
Определим лексикографический порядок $\Lex(x_1,\ldots,x_n)$ следующим образом.
\end{enumerate}
Пусть $m = x_1^{k_1}\ldots x_n^{k_n}$ и $m'= x_1^{k_1'}\ldots x_n^{k_n'}$ -- два монома.
В начале сравниваем $k_1$ и $k_1'$.
Если $k_1 > k_1$, то $m > m'$.
Если $k_1 < k_1'$, то $m < m'$.
Если $k_1 = k_1'$, то мы переходим к сравнению $k_2$ и $k_2'$ и повторяем алгоритм как и выше.
В частности, $m > m'$ тогда и только тогда, когда существует $1\leqslant j \leqslant n$ такой, что $k_1 = k_1',\ldots, k_{j-1} = k_{j-1}'$ и $k_j > k_j'$.
\end{definition}

\begin{examples}
Рассмотрим кольцо $F[x, y, z]$ и мономы $m_1 = x^2 y z^4$, $m_2 = x y^3 z$ и $m_3 = x y z^5$.
Всего существует $6$ способов упорядочить переменные $x, y, z$.
Для каждого упорядочивания будет свой лексикографический порядок.
Ниже написано как мономы будут сравниваться относительно всех шести порядков.
\begin{enumerate}
\item If $x > y > z$, then $m_1 > m_2 > m_3$.

\item If $x > z > y$, then $m_1 > m_3 > m_2$.

\item If $y > x > z$, then $m_2 > m_1 > m_3$.

\item If $y > z > x$, then $m_2 > m_3 > m_1$.

\item If $z > x > y$, then $m_3 > m_1 > m_2$.

\item If $z > y > x$, then $m_3 > m_1 > m_2$.
\end{enumerate}
\end{examples}


\begin{remarks}
Здесь я хочу перечислить некоторые свойства лексикографического порядка.
Все мономы ниже от $n$ переменных и мы используем какой-то лексикографический порядок.
\begin{enumerate}
\item Для любого монома $m\neq 1$, имеем $1 < m$.

\item Если $m$ и $m'$ -- мономы такие, что $m'$ делит $m$, то есть $m = m' t$ для некоторого монома $t$, то $m' \leqslant m$.
Если при этом $t\neq 1$, последнее неравенство является строгим.

\item Если $m, t, s$ -- некоторые мономы такие, что $m \leqslant t$, тогда $ms \leqslant ts$.
Если при этом $m < t$, то $ms < ts$.
\end{enumerate}
\end{remarks}



\begin{claim}
\label{claim::LexWellOrder}
Предположим у нас зафиксирован какой-то лексикографический порядок на мономах от $n$ переменных и $m_1 > m_2 > \ldots > m_k > \ldots$ -- строго убывающая цепочка мономов.
Тогда эта цепочка конечная.
\end{claim}
\begin{proof}
Прежде всего мы можем переименовать переменные в $x_1,\ldots,x_n$ так, что они идут в порядке убывания $x_1>x_2>\ldots>x_n$.
Теперь предположим, что существует бесконечная цепочка убывающих мономов.
Давайте напишем явно степени всех мономов на картинке ниже
\[
\text{Этап I}\quad
\xymatrix@R=10pt@C=10pt{
	{x_1^{k_1(1)}}&{x_2^{k_2(1)}}&{\ldots}&{x_n^{k_n(1)}}\\
	{x_1^{k_1(2)}}\ar@{}[u]|{\rotatebox{90}{$\leqslant$}}&{x_2^{k_2(2)}}&{\ldots}&{x_n^{k_n(2)}}\\
	{\vdots}\ar@{}[u]|{\rotatebox{90}{$\leqslant$}}&{\vdots}&{}&{\vdots}\\
	{x_1^{k_1(r_1)}}\ar@{}[u]|{\rotatebox{90}{$\leqslant$}}&{x_2^{k_2(r_1)}}&{\ldots}&{x_n^{k_n(r_1)}}\\
	{x_1^{k_1(r_1+1)}}\ar@{=}[u]&{x_2^{k_2(r_1+1)}}&{\ldots}&{x_n^{k_n(r_1+1)}}\\
}
\quad\text{Этап II}\quad
\xymatrix@R=10pt@C=10pt{
	{x_1^{k_1}}&{x_2^{k_2(r_1)}}&{\ldots}&{x_n^{k_n(r_1)}}\\
	{x_1^{k_1}}\ar@{=}[u]&{x_2^{k_2(r_1+1)}}\ar@{}[u]|{\rotatebox{90}{$\leqslant$}}&{\ldots}&{x_n^{k_n(r_1+1)}}\\
	{\vdots}\ar@{=}[u]&{\vdots}\ar@{}[u]|{\rotatebox{90}{$\leqslant$}}&{}&{\vdots}\\
	{x_1^{k_1}}\ar@{=}[u]&{x_2^{k_2(r_2)}}\ar@{}[u]|{\rotatebox{90}{$\leqslant$}}&{\ldots}&{x_n^{k_n(r_2)}}\\
	{x_1^{k_1}}\ar@{=}[u]&{x_2^{k_2(r_2+1)}}\ar@{=}[u]&{\ldots}&{x_n^{k_n(r_2+1)}}\\
}
\]
В начале мы смотрим на степени по переменной $x_1$.
По определению лексикографического порядка эти степени должны идти в невозрастающем порядке (возможно в нестрогом), то есть $k_1(1) \geqslant k_1(2)\geqslant \ldots$.
Однако, это последовательность натуральных чисел, она не может быть строго убывающей бесконечно и рано или поздно стабилизируется на некотором числе $r_1$, а именно $k_1(r_1) = k_1(r_1+1) = \ldots$.
Начиная с этого момента все степени у переменной $x_1$ будут одинаковыми и равными, скажем, $k_1$.

Теперь посмотрим на степени по переменной $x_2$ начиная с $r_1$ монома: $m_{r_1}>m_{r_1+1}>\ldots$.
Так как степени при $x_1$ уже не меняются и равны, то мы должны иметь $k_2(r_1)\geqslant k_2(r_1+1)\geqslant \ldots$.
Опять, это невозрастающая цепочка натуральных чисел, а потому она когда-нибудь стабилизируется.
То есть найдется номер $r_2 > r_1$ такой, что $k_2(r_2) = k_2(r_2+1) = \ldots$.
Обозначим эту общую степень по переменной $x_2$ через $k_2$.
Значит для всех мономов $m_i$ с условием $i \geqslant r_2$, степени при $x_1$ и $x_2$ будут равны $k_1$ и $k_2$ соответственно.

Повторяя этот процесс, мы найдем номер $r_3 > r_2$, начиная с которого перестают меняться степени $x_3$.
Потом находим $r_4>r_3$ начиная с которого перестают меняться степени $x_4$ и так далее.
В итоге мы найдем номер $r_n$ такой, что начиная с него все степени всех переменных должны быть одинаковыми.
Но тогда начиная с этого номера цепочка мономов $m_i$ не может быть строго убывающей.
Полученное противоречие завершает доказательство.
\end{proof}


\begin{remark}
Существует популярная ошибка, о которой я хочу поговорить.
Мы только что доказали, что не существует бесконечной строго убывающей цепочки мономов.
Однако, это не значит, что меньше данного монома у нас лишь конечное число мономов.
Действительно, давайте рассмотрим $\mathbb Q[x, y]$ с порядком $x > y$ и соответствующим лексикографическим упорядочиванием.
Тогда существует бесконечно много мономов меньше, чем $x^2$.
Действительно, любой моном $xy^n$ строго меньше, чем $x^2$.
Однако, когда мы пытаемся построить убывающую цепочку, нам надо выбрать конкретный $xy^n$, тогда мы получаем цепочку
\[
x^2 > xy^n > xy^{n-1} > \ldots > xy > x > y^m > y^{m-1} > \ldots > y > 1
\]
Обратите внимание, что в ней мы пропустили бесконечное число членов между $x^2$ и $xy^n$, а так же бесконечное число членов между $x$ и $y^m$.
Таким образом убывающие цепочки начинающиеся с $x^2$ могут быть сколь угодно большой длины, но обязательно конечны.
\end{remark}


\subsection{Редукция}

Теперь после разговора о порядках, давайте определим процедуру редукции.
Мы это сделаем в два шага: в начале определим элементарную редукцию, а потом уже и общий случай.
Давайте напомню контекст.
У нас зафиксировано некоторое поле $F$, кольцо многочленов $F[x_1,\ldots,x_n]$, и произвольный лексикографический порядок на мономах от $n$ переменных.

\begin{definition}
Предположим $F$ -- поле, мы фиксировали лексикографический порядок на мономах в $F[x_1,\ldots,x_n]$ и $f\in F[x_1,\ldots,x_n]$ -- произвольный ненулевой многочлен.
Тогда многочлен $f$ может быть записан в виде
\[
f = c_1 m_1 + c_2 m_2 + \ldots + c_k m_k,\quad c_i\in F,\;m_i\;\text{-- мономы такие, что}\; m_1 > m_2 >\ldots>m_k
\]
Терм $c_1m_1$ называется старшим термом многочлена $f$ и будет обозначаться $T(f)$.
Моном $m_1$ называется старшим мономом $f$ и обозначается $M(f)$.
Коэффициент $c_1$ называется старшим коэффициентом многочлена $f$ и будет обозначаться $C(f)$.
По определению $T(f) = C(f) M(f)$.
Часть $c_2m_2 + \ldots + c_k m_k$ называется хвостом $f$ и будет обозначаться $f_0$.
Таким образом, любой многочлен $f$ может быть записан как $f = T(f) + f_0$.
\end{definition}


\begin{definition}
Предположим $g\in F[x_1,\ldots,x_n]$ -- ненулевой многочлен и $f\in F[x_1,\ldots,x_n]$ -- произвольный многочлен.
Предположим
\[
f= c_1m_1 + \ldots + c_i m_i + \ldots + c_k m_k,\quad c_i\in F,\;m_i\;\text{-- мономы такие, что выполнено}\; m_1 > m_2 >\ldots>m_k
\]
и
\[
g = C(g) M(g) + g_0 = T(g) + g_0
\]
Фиксируем моном $m = m_i$ в многочлене $f$ и предположим, что $m$ делится на старший моном $g$, то есть $m = t M(g)$.
Мы определим элементарную редукцию $f$ относительно $g$ следующим образом
\[
f\stackrel{g}{\longrightarrow} f' = f - \frac{c_i}{C(g)}t g
\]
Многочлен $f'$ -- это результат элементарной редукции.
\end{definition}

Если описать редукцию по простому, то она действует следующим образом: мы находим в $f$ моном $m_i$, который делится на $M(g)$ и заменяем его на хвост $g$ умноженный на $- c_i m_i / T(g)$.

\begin{example}
Рассмотрим кольцо $\mathbb Q[x, y, z]$, $f = xyz$, $g_1 = xy - z$, и $g_2 = yz - 1$.
Порядок будет произвольным лексикографическим.
Тогда
\[
f\stackrel{g_1}{\longrightarrow} xyz - z(xy - z) = z^2,
\quad\text{или}\quad
f\stackrel{g_2}{\longrightarrow} xyz - x(yz - 1) = x
\]
\end{example}

\begin{definition}
Предположим $G\subseteq F[x_1,\ldots,x_n]\setminus \{0\}$ -- некоторое множество многочленов и $f, f'\in F[x_1,\ldots,x_n]$ -- многочлены.
Мы скажем, что $f$ редуцируется к $f'$ с помощью $G$, если существует конечная последовательность элементарных редукций как ниже%
\footnote{Многочлены $g_1,\ldots,g_k$ не обязаны быть различными.}
\[
f\stackrel{g_1}{\longrightarrow}f_1\stackrel{g_2}{\longrightarrow}f_2\stackrel{g_3}{\longrightarrow}\ldots \stackrel{g_k}{\longrightarrow}f_k = f'\quad\text{где}\;g_i \in G
\]
Будем писать в этом случае $f\stackrel{G}{\rightsquigarrow} f'$.

Если многочлен $f'$ уже не редуцируем никаким $g\in G$, будем говорить, что $f'$ является остатком $f$ относительно $G$.
\end{definition}


\begin{remarks}
\begin{enumerate}
\item Многочлен $f'$ не редуцируем никаким $g\in G$ тогда и только тогда, когда каждый моном $f'$ не делится ни на какой $M(g)$ для $g\in G$.

\item Важно отметить, что вообще говоря остаток не определен однозначно.
Элементарные редукции $f$ относительно членов $G$ в разном порядке могут привести к разным остаткам.
Вот пример подобного.
Пусть $\mathbb Q[x, y, z]$ и фиксирован лексикографический порядок для $x > y > z$.
Положим $f = xyz$, $g_1 = xy - 1$, $g_2 = yz - 1$, и $G = \{g_1, g_2\}$.
Тогда,
\[
f\stackrel{g_1}{\longrightarrow}xyz - z(xy - 1) = z
\quad\text{и}\quad
f\stackrel{g_2}{\longrightarrow}xyz - x(yz - 1) = x
\]
Тогда по определению $f\stackrel{G}{\rightsquigarrow} z$ и $f\stackrel{G}{\rightsquigarrow} x$.
Более того, многочлены $z$ и $x$ не редуцируемы с помощью $G$.
Значит, это два разных остатка $f$ относительно $G$.
\end{enumerate}
\end{remarks}

Как мы увидели в общем случае остаток многочлена $f$ относительно $G$ не единственный.
Грубо говоря, такое происходит потому что множество $G$ было выбрано не удачно.
Таким образом можно выделить <<хорошие>> множества, для которых остаток будет получаться однозначным.
Такая процедура ведет к понятию базиса Грёбнера.

\begin{definition}
[Базис Грёбнера]
Предположим $F$ -- некоторое поле, $G\subseteq F[x_1,\ldots,x_n]\setminus\{0\}$, и у нас зафиксирован какой-то лексикографический порядок.%
\footnote{Я хочу отметить, что понятие базиса Грёбнера сильно зависит от выбранного порядка.
Если мы изменим порядок на переменных, то множество $G$ может стать базисом Грёбнера или наоборот перестанет им быть.}
Будем говорить, что $G$ является базисом Грёбнера если для любого $f\in F[x_1,\ldots,x_n]$ все его остатки совпадают.
\end{definition}

Давайте поговорим о еще одном важном вопросе связанном с редукцией.
Предположим нам дали многочлен $f$ и множество $G$.
Если $f$ редуцируем относительно $G$, то мы можем сделать элементарную редукцию и получить $f_1$.
Теперь если $f_1$ тоже редуцируем, то мы можем провести элементарную редукцию и получить многочлен $f_2$ и так далее.
Вопрос в том остановится ли этот процесс когда-либо.
Оказывается, что ответ на этот вопрос положительный, как показывает утверждение~\ref{claim::ReductFin}.

\begin{examples}
Мы с вами познакомимся со способами проверки является ли множество $G$ базисом Грёбнера.
Сейчас же я хочу привести пару примеров без доказательства, просто потому что мы пока не готовы это объяснить.
\begin{enumerate}
\item Для любого кольца многочленов $F[x_1,\ldots,x_n]$ с произвольным лексикографическим порядком множество $G = \{g\}$ состоящее из одного ненулевого многочлена всегда является базисом Грёбнера.

\item Пусть $F[x,y,z,w]$ и мы используем какой-нибудь лексикографический порядок.
Тогда множество $G = \{xy - 1, zw + 1\}$ является базисом Грёбнера.
Это происходит в силу того, что старшие мономы элементов $G$ взаимно просты.
\end{enumerate}
\end{examples}


\begin{definition}
Пусть $F$ -- некоторое поле и фиксирован некоторый лексикографический порядок на мономах от $n$ переменных.
Как и раньше, каждый многочлен $f\in F[x_1,\ldots,x_n]$ может быть представлен в виде
\[
f = c_1 m_1 + c_2 m_2 + \ldots + c_k m_k,\quad c_i\in F,\;m_i\;\text{-- мономы такие, что}\; m_1 > m_2 >\ldots>m_k
\]
Мы обозначим $i$-ый по старшинству моном $m_i$ через $M_i(f)$.
В частности $M_1(f)$ -- это старший моном $f$, $M_2(f)$ -- это второй по старшинству моном в $f$ и так далее.
Обратите внимание, что $i$-ый по старшинству моном не обязан существовать в $f$, это происходит если $f$ содержит меньше $i$ мономов.
\end{definition}

\begin{claim}
\label{claim::MkReduct}
Пусть $F$ -- некоторое поле, мы работаем с многочленами $F[x_1,\ldots,x_n]$, и некоторый лексикографический порядок задан.
Пусть $f, f',g\in F[x_1,\ldots,x_n]$ такие, что  $f\stackrel{g}{\longrightarrow}f'$ и $M_1(f) = M_1(f'), \ldots, M_{k-1}(f) = M_{k-1}(f')$, и пусть мономы $M_k(f)$ и $M_k(f')$ существуют.
Тогда, $M_k(f) \geqslant M_k(f')$.
\end{claim}
\begin{proof}
Предположим, что $f = c_1 m_1 + \ldots + c_{k-1} m_{k-1} + c_km_k +\ldots + c_s m_s$, где $c_i\in F$ и $m_i$ -- мономы расположенные в убывающем порядке.
Многочлен $f'$ получен из $f$ с помощью одной элементарной редукции.
Это означает, что есть какой-то моном $m_i$ делящийся на $M(g)$, который мы заменили на линейную комбинацию более младших мономов в $f'$.
По предположению первые $k-1$ моном в $f$ и $f'$ совпадают.
А это значит, что ни один из них не был редуцирован.
То есть $i\geqslant k$.
Если $i>k$ то, $k$-ый моном в $f'$ такой же как и $k$-ый моном в $f$.
Это значит, что в этом случае $M_k(f) = M_k(f')$.
Если $i = k$, то мы заменили $m_k$ на линейную комбинацию более младших мономов.
Например результат может выглядеть так
\[
\xymatrix@R=8pt@C=10pt{
	{f}\ar@{<->}[rr]&{}&{m_1}&{\ldots}&{m_{k-1}}&{m_k}&{}&{m_{k+1}}&{}&{\ldots}&{m_s}&{}\\
	{f'}\ar@{<->}[rr]&{}&{m_1}&{\ldots}&{m_{k-1}}&{}&{m'_k}&{m'_{k+1}}&{m'_{k_2}}&{\ldots}&{}&{m'_{s'}}\\
}
\]
Но тогда моном $m'_k = M_k(f')$ строго младше монома $m_k = M_k(f)$, а это означает что надо.
\end{proof}

\begin{claim}
\label{claim::ReductFin}
Пусть $F$ -- некоторое поле, мы работаем с кольцом $F[x_1,\ldots,x_n]$, на котором зафиксирован какой-то лексикографический порядок.
Тогда для любого многочлена $f\in F[x_1,\ldots,x_n]$ и любого подмножества $G\subseteq F[x_1,\ldots,x_n]\setminus\{0\}$, любая последовательность элементарных редукций $f$ с помощью $G$ конечна.
\end{claim}
\begin{proof}
Предположим, что верно противное и найдется бесконечная последовательность элементарных редукций.
\[
f\stackrel{g_1}{\longrightarrow}f_1\stackrel{g_2}{\longrightarrow}f_2\stackrel{g_3}{\longrightarrow}\ldots \stackrel{g_k}{\longrightarrow}f_k\stackrel{g_{k+1}}{\longrightarrow}\ldots
\]
Давайте построим бесконечную убывающую цепочку мономов, что и даст нужное противоречие.
Посмотрим на цепочку старших мономов $M_1(f),M_1(f_1), M_1(f_2),\ldots$.
Применяя утверждение~\ref{claim::MkReduct} в случае $k = 1$, мы видим, что $M_1(f) \geqslant M_1(f_1)\geqslant M_1(f_2) \geqslant \ldots$.
По утверждению~\ref{claim::LexWellOrder},  эта цепочка обязана стабилизироваться, то есть найдется $r_1$ такое, что $M_1(f_{r_1}) = M_1(f_{r_1+1})=\ldots$.
Значит начиная с номера $r_1$ все редукции имеют одинаковый старший моном, обозначим его через $m_1$.
Так как последовательность редукций бесконечно, у нас должен существовать второй по старшинству моном во всех редукциях с момента $r_1$.
Опять воспользуемся утверждением~\ref{claim::MkReduct} но теперь для случая $k = 2$.
Мы получим $m_1 > M_2(f_{r_1})\geqslant M_2(f_{r_1 + 1})\geqslant \ldots$.
Эта убывающая цепочка мономов должна стабилизироваться по утверждению~\ref{claim::LexWellOrder}.
Значит найдется номер $r_2 > r_1$ такой, что $m_1 > M_2(f_{r_2}) = M_2(f_{r_2+1})=\ldots$.
То есть начиная с номера $r_2$ все редукции имеют одинаковый старший моном $m_1$ и одинаковый второй по старшинству моном, который обозначим через $m_2$.
В итоге у нас получилась цепочка мономов $m_1 > m_2$.
Так как цепочка бесконечная, то у нас обязательно существует третий моном во всех редукциях с номера $r_2$.

Значит мы как и выше можем повторить это рассуждение для третьего по величине монома.
Потом для четвертого и так далее.
В результате мы построим цепочку мономов $m_1 > m_2 > m_3 > \ldots > m_s> \ldots$.
Построенная цепочка противоречит утверждению~\ref{claim::LexWellOrder}, что заканчивает доказательство.
\end{proof}


