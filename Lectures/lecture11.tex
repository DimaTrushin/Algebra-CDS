\ProvidesFile{lecture11.tex}[Lecture 11]


\subsection{Доказательство критерия Бухбергера}

Теперь я хочу расплатиться с долгами за предыдущую лекцию и доказать много чего недоказанного.
Начать я хочу с критерия Бухбергера.
Его доказательство основано на вспомогательном утверждение, которое называется Diamond Lemma.

\begin{claim}
[The Diamond Lemma]
Пусть $F$ -- поле, мы фиксировали лексикографический порядок на мономах от $n$ переменных и $G\subseteq F[x_1,\ldots,x_n]\setminus\{0\}$ -- некоторое множество.
Тогда следующие условия эквивалентны

\begin{tabular}{cc}
{
\parbox{12cm}{
\begin{enumerate}
\item $G$ является базисом Грёбнера.

\item Для любого многочлена $f\in F[x_1,\ldots,x_n]$ и любых элементарных редукций $f\stackrel{g_1}{\longrightarrow}f_1$ и $f\stackrel{g_2}{\longrightarrow}f_2$ (где $g_1,g_2\in G$) найдется $f'\in F[x_1,\ldots,x_n]$ такой, что $f_1\stackrel{G}{\rightsquigarrow}f'$ и $f_2\stackrel{G}{\rightsquigarrow}f'$.%
\footnote{Тут мы подразумеваем, что найдется хотя бы одна цепочка редукций от $f_1$ к $f'$ и хотя бы одна цепочка редукций от $f_2$ к $f'$.}
\end{enumerate}
}}&{
\parbox{4cm}{
\[
\xymatrix@R=15pt@C=15pt{
	{}&{\textcolor{red}{f}}\ar[dl]_{g_1}\ar[dr]^{g_2}&{}\\
	{\textcolor{OliveGreen}{f_1}}\ar[d]&{}&{\textcolor{OliveGreen}{f_2}}\ar[d]\\
	{}\ar@{}[d]|{\vdots}&{}&{}\ar@{}[d]|{\vdots}\\
	{}\ar[dr]&{}&{}\ar[dl]\\
	{}&{f'}&{}\\
}
\]
}
}
\end{tabular}
\end{claim}
\begin{proof}
(1)$\Rightarrow$(2).
Пусть $r_1$ -- остаток $f_1$ относительно $G$ и $r_2$ -- остаток $f_2$ относительно $G$.
В частности $f_1 \stackrel{G}{\rightsquigarrow} r_1$ и $f_2 \stackrel{G}{\rightsquigarrow} r_2$.
По определению $r_1$ и $r_2$ так же являются остатками $f$.
Так как $G$ является базисом Грёнера все остатки $f$ равны между собой.
Значит $r_1 = r_2$.
А потому $f' = r_1 = r_2$ удовлетворяет всем требованиям.

(2)$\Rightarrow$(1).
Нам надо показать, что любой многочлен $f\in F[x_1,\ldots,x_n]$ имеет только один остаток относительно $G$.
Предположим противное, что найдется многочлен $f$, который имеет хотя бы два разных остатка.
Давайте рассмотрим все элементарные редукции $f$ относительно всех элементов $G$.
Тогда возможно два случая:
\begin{enumerate}
\item Все элементарные редукции имеют единственный остаток относительно $G$.

\item Существует хотя бы одна редукция, у которой есть хотя бы два разных остатка.
\end{enumerate}
Я хочу показать, что мы можем выбрать многочлен $f$ имеющий хотя бы два разных остатка, но при этом все его элементарные редукции имеют по одному остатку.
Если $f$ уже такой, то ничего доказывать не надо.
Потому можно предположить, что хотя бы одна элементарная редукция $f_1$ многочлена $f$ имеет несколько разных остатков.
Если $f_1$ удовлетворяет свойству (1), то мы нашли нужный многочлен.
Иначе у него есть элементарная редукция $f_2$, которая имеет несколько разных остатков.
То есть мы имеем цепочку редукций
\[
f\stackrel{g_1}{\longrightarrow}f_1\stackrel{g_2}{\longrightarrow}f_2\stackrel{g_3}{\longrightarrow}f_3\stackrel{g_4}{\longrightarrow}\ldots
\]
По утверждению~\ref{claim::ReductFin}, такая цепочка ограничена.
Причина, по которой мы не можем продолжить эту цепочку -- все редукции какого-то многочлена $f_k$ имеют ровно один остаток.
Давайте для наглядности изобразим этот процесс на картинке ниже.
Красным цветом отмечены многочлены имеющие несколько разных остатков.
Зеленым цветом отмечены многочлены имеющие единственный остаток.
\[
\xymatrix@R=15pt@C=15pt{
	{}&{}&{\textcolor{red}{f}}\ar[dll]\ar[d]\ar[dr]\ar[drrr]&{}&{}&{}\\
	{\textcolor{red}{*}}&{\ldots}&{\textcolor{red}{f_1}}\ar[dll]\ar[d]\ar[dr]\ar[drrr]&{\textcolor{OliveGreen}{*}}&{\ldots}&{\textcolor{OliveGreen}{*}}\\
	{\textcolor{red}{*}}&{\ldots}&{\textcolor{red}{f_2}}\ar[dll]\ar[d]\ar[dr]\ar[drrr]&{\textcolor{OliveGreen}{*}}&{\ldots}&{\textcolor{OliveGreen}{*}}\\
	{}&{\ldots}&{}\ar@{}[d]|(.3){\vdots}&{}&{\ldots}&{}\\
	{}&{}&{\textcolor{red}{f_k}}\ar[dll]\ar[d]\ar[dr]\ar[drrr]&{}&{}&{}\\
	{\textcolor{OliveGreen}{*}}&{\ldots}&{\textcolor{OliveGreen}{*}}&{\textcolor{OliveGreen}{*}}&{\ldots}&{\textcolor{OliveGreen}{*}}\\
}
\]
Теперь мы можем предположить, что $f$ имеет хотя бы два разных остатка, но при этом все его элементарные редукции имеют один единственный остаток.
Предположим, что $f  \stackrel{G}{\rightsquigarrow}  r_1$ и $f  \stackrel{G}{\rightsquigarrow}  r_2$ -- два разных остатка.
По определению каждая цепочка редукции начинается с какой-то элементарной редукции.
Потому найдутся многочлены $g_1, g_2\in G$ и многочлены $f_1, f_2\in F[x_1,\ldots,x_n]$ такие, что 
\[
f  \stackrel{g_1}{\longrightarrow}f_1  \stackrel{G}{\rightsquigarrow} r_1
\quad\text{and}\quad
f  \stackrel{g_2}{\longrightarrow}f_2  \stackrel{G}{\rightsquigarrow} r_2
\]
С другой стороны, по нашему предположению, существует многочлен $f'$ такой, что $f_1$ и $f_2$ редуцируются к нему с помощью $G$.
Давайте доредуцируем $f'$ до остатка $r'$.
Подытожим, что произошло на следующей картинке:
\[
\xymatrix@R=15pt@C=15pt{
	{}&{}&{\textcolor{red}{f}}\ar[dl]_{g_1}\ar[dr]^{g_2}&{}&{}\\
	{}&{\textcolor{OliveGreen}{f_1}}\ar[dl]\ar[d]&{}&{\textcolor{OliveGreen}{f_2}}\ar[d]\ar[dr]&{}\\
	{}\ar@{}[d]|{\vdots}&{}\ar@{}[d]|{\vdots}&{}&{}\ar@{}[d]|{\vdots}&{}\ar@{}[d]|{\vdots}\\
	{}\ar[d]&{}\ar[dr]&{}&{}\ar[dl]&{}\ar[d]\\
	{r_1}&{}&{r'}&{}&{r_2}\\
}
\]
По выбору $f$, $r_1 \neq r_2$ -- два разных остатка.
С другой стороны, многочлены $f_1$ и $f_2$ имеют единственный остаток.
Потому, каждый остаток $f_1$ одинаковый.
Но $r_1$ и $r'$ являются остатками $f_1$.
Потому, $r_1 = r'$.
Аналогичный аргумент показывает, что $r_2 = r'$.
Значит $r_1 = r' = r_2$ -- противоречие.
Это противоречие возникло из нашего предположения, что найдется многочлен $f$ имеющий хотя бы два разных остатка.
Значит это предположение было не верно, что и доказывает требуемое.
\end{proof}

\begin{claim}
[Критерий Бухбергера]
Пусть $F$ -- поле, мы фиксировали лексикографический порядок на мономах от $n$ переменных, и $G\subseteq F[x_1,\ldots,x_n]\setminus\{0\}$ -- некоторое множество.
Тогда следующие условия эквиваленты:
\begin{enumerate}
\item $G$ является базисом Грёбнера.

\item Для любых $g_1,g_2\in G$, любая редукция $S_{g_1\,g_2}$ относительно $G$ ведет к нулевому остатку.

\item Для любых $g_1,g_2\in G$, существует редукция $S_{g_1\,g_2}$ относительно $G$ ведущая к нулевому остатку.
\end{enumerate}
\end{claim}
\begin{proof}
(1)$\Rightarrow$(2).
Предположим $g_1,g_2\in G$, $g_1 = c_1 m_1 + g'_1$, $g_2 = c_2m_2 + g'_2$, и наименьшее общее кратное старших мономов $m_1$ и $m_2$ является $m = t_1 m_1 = t_2 m_2$.
Положим $f = c_2 t_1 g_1$.
Тогда $f$ редуцируется к нулю с помощью $g_1$.
С другой стороны, $f$ редуцируется к  $S_{g_1\,g_2}$ с помощью $g_2$.
Действительно,
\[
f\stackrel{g_2}{\longrightarrow} f' = c_2 t_1 g_1 - c_1 t_2 g_2 = S_{g_1\,g_2}
\]
Значит остатки $f'$ относительно $G$ являются остатками $f$.
Но $f$ редуцируется к нулю и $G$ является базисом Грёбнера.
Значит $f'$ имеет только нулевые остатки, что и означает, что любая редукция $f'$ приводит к нулю.

(2)$\Rightarrow$(3).
Эта импликация ясна.

(3)$\Rightarrow$(1).
Нам надо показать, что $G$ является базисом Грёбнера.
Для этого нам достаточно проверить, что выполняется второе условие из Diamond Lemma.
Предположим, что нам задан произвольный многочлен $f\in F[x_1,\ldots,x_n]$ и две редукции $f \stackrel{g_1}{\longrightarrow}f_1$ и $f\stackrel{g_2}{\longrightarrow}f_2$, где $g_1,g_2\in G$.
Нам надо показать, что $f_1$ и $f_2$ редуцируются к одному и тому же многочлену $f'$ с помощью $G$.
Предположим $g_1 = c_1 m_1 + g_1'$ и $g_2 = c_2 m_2 + g_2'$, где $m_1$ и $m_2$ являются старшими мономами в $g_1$ и $g_2$ соответственно.
Предположим, что редукции имеют вид $f_1 = f - a_1 t_1 g_1$ и $f_1 = f - a_2 t_2 g_2$, где $a_1$ и $a_2$ -- некоторые коэффициенты, а $t_1$ и $t_2$ -- мономы.
Утверждение~\ref{claim::ReductionProps} говорит, что нам достаточно показать, что $f_1 - f_2$ редуцируется к нулю относительно $G$, то есть надо показать, что $a_2t_2g_2 - a_1 t_1 g_1 \stackrel{G}{\rightsquigarrow}0$.

Давайте посмотрим на старшие мономы многочленов $t_1g_1$ и $t_2 g_2$.
Возможны два случая: либо мономы различны, либо они равны.
Рассмотрим эти случаи попорядку.

\paragraph{случай 1}

Не теряя общности, можно предположить, что $M(t_2 g_2) > M(t_1 g_1)$.
Тогда многочлен $a_2t_2g_2 - a_1 t_1 g_1$ зависит от монома $M(t_2 g_2)$, точнее он будет его старшим мономом.
А раз этот моном явно присутствует и делится на $M(g_2)$, то мы можем проделать редукцию относительно $g_2$, а именно
\[
a_2t_2g_2 - a_1 t_1 g_1\stackrel{g_2}{\longrightarrow} a_2t_2g_2 - a_1 t_1 g_1 - a_2t_2 g_2 = - a_1 t_1 g_1 \stackrel{g_1}{\longrightarrow} 0
\]

\paragraph{случай 2}

Теперь рассмотрим случай $m = M(t_1 g_1) = M(t_2 g_2)$.
Посмотрим дополнительно на старшие коэффициенты многочленов $a_1t_1 g_1$ $a_2 t_2 g_2$.
Коэффициенты будут $a_1 c_1$ и $a_2 c_2$ соответственно.
Теперь у нас есть два подслучая: старшие коэффициенты либо равны, либо нет.

\paragraph{случай 2а}

Предположим $a_2 c_2 = a_1 c_1$.
Тогда $\lambda = a_2 / c_1 = a_1 / c_2$.
Давайте покажем, что многочлен $a_2t_2g_2 - a_1 t_1 g_1$ пропорционален S-многочлену от $g_1$ и $g_2$.
Пусть $d$ -- наибольший общий делитель $t_1$ и $t_2$.
Тогда, $t_1 = d t_1'$ и $t_2 = d t_2'$.
Значит,
\begin{gather*}
a_2t_2g_2 - a_1 t_1 g_1 = \lambda d \left(c_1 t_2' g_2 - c_2 t_1' g_1\right) = \lambda d S_{g_2\,g_1}
\end{gather*}
По предположению $S_{g_2\, g_1}$ как-то редуцируется к нулю.
Тогда утверждение~\ref{claim::ReductionProps} пункт~(1), гласит, что $\lambda d S_{g_1\,g_1}$ так же редуцируется к нулю.

\paragraph{случай 2б}

Теперь рассмотрим случай $a_2 c_2 \neq a_1 c_1$.
В этот раз оном $m = M(t_2 g_2) = M(t_1 g_1)$ не сокращается в разности $a_2t_2g_2 - a_1 t_1 g_1$.
Давайте редуцируем его с помощью $g_1$, получим
\[
a_2t_2g_2 - a_1 t_1 g_1 \stackrel{g_1}{\longrightarrow}
a_2t_2g_2 - a_1 t_1 g_1 - \frac{a_2 c_2 - a_1 c_1}{c_1}t_1 g_1 = a_2 t_2 g_2 - \frac{a_2 c_2}{c_1}t_1 g_1
\]
Но теперь, мы оказались в \textbf{случае 2а} потому что $a_2 c_2 = \frac{a_2 c_2}{c_1}c_1$.
Значит последний многочлен редуцируется к нулю по предыдущему случаю и все доказано.
\end{proof}

\subsection{Остановка алгоритма Бухбергера}

Начать я хочу со вспомогательной леммы про мономы.

\begin{claim}
Пусть у нас дана последовательность мономов от $n$ переменных: $m_1, m_2, \ldots, m_n,\ldots$.
Предположим, что для любого $i$ моном $m_i$ не делится на более ранние мономы $m_j$ при $j < i$.
Тогда данная последовательность мономов конечна.
\end{claim}
\begin{proof}
Сделаем одно общее замечание.
Если мы перейдем от последовательности $m_i$ к любой подпоследовательности, то условие, что каждый элемент не делится на все предыдущие сохраняется.
Теперь предположим, что последовательность $m_i$ бесконечная и попытаемся прийти к противоречию.
Далее доказательство будем вести по индукции по количеству переменных.
Если у нас всего одна переменная и $m_1 = x_1^m$, то чтобы $m_i$ не делилось на $m_1$ в них степень по $x_1$ должна быть меньше $m$.
А значит у нас не может быть бесконечной последовательности (противоречие с нашим предположением).

Теперь рассмотрим случай $n$ переменных.
Так как $m_i$ не делится на $m_1$, то найдется переменная (зависящая от $i$) по которой степень в $m_i$ меньше, чем степень в $m_1$.
Так как последовательность мономов бесконечная, а переменных конечное число, то для какой-то переменной найдется бесконечное число мономов.
Будем считать что это переменная $x_n$.
Перейдем к этой подпоследовательности, тогда получим, что для любого монома $m_i$ его степень по $x_n$ меньше, чем степень $m_1$ по $x_n$.
Значит у нас всего встречается конечное число степеней по $x_n$.
Опять переходя к подпоследовательности можно считать, что степени всех мономов по $x_n$ одинаковая и равна $d$.
Значит все мономы выглядят так $m_i = x_1^{k_1(i)}\ldots x_{n-1}^{k_{n-1}(i)} x_n ^d$.
При этом вспомним, что $m_i$ не делится на более ранние мономы $m_j$ при $j < i$.
Построим последовательность $m'_i = x_1^{k_1(i)}\ldots x_{n-1}^{k_{n-1}(i)}$.
Тогда ясно, что в этой последовательности никакой моном не делится на более ранние мономы и эта последовательность от меньшего числа переменных.
Тогда по индукции она должна быть конечной, но мы построили бесконечную последовательность.
Полученное противоречие завершает доказательство.
\end{proof}


\begin{claim}
Алгоритм Бухбергера останавливается.
\end{claim}
\begin{proof}
Давайте вспомним как работает алгоритм Бухбергера.
Мы стартуем с конечного множества $G \subseteq F[x_1,\ldots, x_n]\setminus\{0\}$, мы фиксируем лексикографический порядок на мономах.
Теперь алгоритм действует следующим образом
\begin{enumerate}
\item Инициализация $G_0 = G$.

\item Для всех $g_i,g_j\in G_0$ найдем $S_{g_i, g_j}$ и отредуцируем его к остатку $r_{ij}$ с помощью  $G_0$.

\item Если все $r_{ij} = 0$, то $G_0$ ответ.
Иначе положим $G_0 = G_0 \cup \{r_{ij}\mid r_{ij}\neq 0\}$ и перейдем к шагу $2$.
\end{enumerate}
Давайте предположим, что алгоритм не останавливается.
Тогда после первого цикла мы добавили хотя бы один остаток в $G_0$, обозначим его $r_1$.
И после второго цикла мы добавили хотя бы один остаток, назовем его $r_2$.
И так далее.
В итоге мы получим бесконечную последовательность многочленов $r_1,r_2,\ldots, r_n,\ldots$.
Посмотрим на их старшие мономы $M(r_1), M(r_2), \ldots, M(r_n),\ldots$.
Если мы покажем, что никакой моном в этой последовательности не делится на предыдущий, мы получим противоречие с предыдущей леммой, что докажет наше утверждение.
Рассмотрим какой-нибудь многочлен $r_i$.
Тогда на $i$-ом шаге алгоритма многочлены $r_1,\ldots, r_{i-1}$ уже лежат в $G_0$ и мы добавляем к ним $r_i$.
При этом $r_i$ является остатком относительно $G_0$.
В частности он не редуцируется относительно $r_1,\ldots, r_{i-1}$.
Но тогда старший моном $r_i$ не может делиться на старшие мономы $r_1,\ldots,r_{i-1}$.
Действительно, в противном случае мы могли бы отредуцировать старший моном в $r_i$, что невозможно.
\end{proof}
