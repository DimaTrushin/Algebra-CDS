\ProvidesFile{lecture05.tex}[Lecture 5]


\section{Кольца и поля}

\subsection{Определения}

До этого мы с вами изучали множество с одной операцией.
Оказывается, что для многих приложений этого не достаточно и надо рассмотреть множество с двумя операциями.
Это ведет нас к понятию кольца.

\begin{definition}
Я собираюсь определить кольцо $(R, +, \cdot)$ или более кратко $R$.
\begin{itemize}
\item\textbf{Данные:} 
\begin{enumerate}
\item Множество элементов $R$.

\item Операция $+ \colon R\times R\to R$ называемая сложением.

\item Операция $\cdot \colon R\times R\to R$ называемая умножением.
\end{enumerate}

\item\textbf{Аксиомы:}
\begin{enumerate}
\item $(R, +)$ является абелевой группой.

\item Умножение дистрибутивно относительно сложения  с обеих сторон:
\[
a(b + c) = ab + ac \quad\text{and}\quad (a + b) c = ac + bc\quad \text{для всех }a, b, c\in R
\]

\item Умножение ассоциативно: $(ab)c = a(bc)$ для всех $a, b, c\in R$.

\item Умножение имеет нейтральный элемент обозначаемый $1$.
\end{enumerate}
\end{itemize}
В случае выполнения аксиом выше говорят, что $(R, +, \cdot)$ -- ассоциативное кольцо с единицей.
Мы же будем называть такие объекты просто кольцами, так как не будем рассматривать неассоциативные кольца или кольца в которых нет единицы.
Как и раньше, мы будем опускать обозначения операций и говорить, что $R$ является кольцом, подразумевая, что на $R$ имеются нужные операции.
Нейтральный элемент по сложению будет обозначаться $0$ и называется нулем в кольце.
Если $a\in R$ -- произвольный элемент, то обратный по сложению обозначается $-a$.
Для любых элементов $a, b\in R$, выражение $a + (-b)$ будет для краткости записываться так $a - b$.

Если в дополнение а аксиомам выше выполняется
\begin{itemize}
\item[]
\begin{enumerate}
\setcounter{enumi}{4}
\item Умножение коммутативно: $ab = ba$ для всех $a, b\in R$.
\end{enumerate}
\end{itemize}
То кольцо называется коммутативным.
А если мы добавим еще две аксиомы
\begin{itemize}
\item[]
\begin{enumerate}
\setcounter{enumi}{5}
\item Всякий ненулевой элемент обратим по умножению: для любого $a\in R\setminus\{0\}$, существует элемент $b\in R$ такой, что $ab = ba = 1$.%
\footnote{Стоит обратить внимание, что не достаточно проверять только одно из условий $ab = 1$ или $ba = 1$ в случае если кольцо не коммутативно.}

\item $1 \neq 0$.
\end{enumerate}
\end{itemize}
То кольцо называется полем.
В этом случае обратный элемент к $a$ обозначается  $a^{-1}$.
\end{definition}

\begin{examples}
\begin{enumerate}
\item Пусть $R = \{*\}$ -- множество состоящее из одного элемента.
Тогда существует только одна бинарная операция на таком множестве.
Мы будем использовать эту операцию как сложение и как умножение на множестве $R$: $+\colon R\times R\to R$ задано по правилу $* + * = *$ и $\cdot \colon R\times R\to R$ задано по правилу $* \cdot * = *$.
Тогда $R$ -- коммутативное кольцо.
Его единственный элемент $*$ является одновременно и нулем (нейтральным по сложению) и единицей (нейтральным по умножению).
Любой ненулевой элемент является обратимым, потому что в кольце нет ненулевых элементов, а значит аксиома~(6) выполнена.
Но при этом $1 = 0$ в кольце.
Такое кольцо называется нулевым.

Давайте покажем, что если в кольце $1 = 0$, то все элементы равны нулю.
То есть кольцо является нулевым.
Действительно, пусть $S$ -- кольцо в котором $1 = 0$.
Возьмем произвольный $x\in S$, тогда $x = x \cdot 1 = x \cdot 0$.
Теперь надо показать, что $x \cdot 0 = 0$ в любом кольце.
Это делается следующим образом.
По определению нуля имеем $0 + 0 = 0$.
Теперь умножим это равенство на $x$ и получим $x\cdot (0 + 0) = x \cdot 0$.
Раскроем левую часть по дистрибутивности $x \cdot 0 + x\cdot 0 = x\cdot 0$.
А теперь прибавим обратный к $x \cdot 0$ к обеим частям и получим, что $ x\cdot 0 = 0$.

Еще обратите внимание, что для нулевого кольца выполняются все аксиомы поля, кроме последней седьмой.
По сути эта аксиома нужна для того, чтобы исключить из списка полей нулевое кольцо.

\item Целые числа с обычными сложением и умножением $(\mathbb Z, +, \cdot)$ являются коммутативным кольцом.

\item Кольцо матриц с матричными сложением и умножением $(\operatorname{M}_n(\mathbb R), +, \cdot)$ является кольцом.

\item Вещественные числа с обычными сложением и умножением $(\mathbb R, +, \cdot)$ являются полем.

\item Множество остатков по модулю натурального числа $n$ с операциями сложения и умножения по модулю $n$ $(\mathbb Z_n, +, \cdot)$ является коммутативным кольцом.

\item Если в предыдущем примере модуль $p$ является простым, то кольцо $(\mathbb Z_p, +,\cdot)$ является полем.
Давайте поясним это.
Ясно, что $\mathbb Z_p$ коммутативное кольцо, в котором $1 \neq 0$.
Потому надо лишь проверить аксиому~(6).
Для этого возьмем $a\in \mathbb Z_p\setminus\{0\}$, то есть $1 \leqslant a < p$.
Так как число $p$ простое, последнее означает, что $a$ и $p$ взаимно просты.
А значит по расширенному алгоритму Евклида найдутся такие $u, v\in \mathbb Z$, что $u a + v p = 1$.
Теперь рассмотрим последнее равенство по модулю $p$ и получим $u a = 1 \pmod p$.
Что и означает обратимость элемента $a$.

\item Пусть $A$ -- некоторое коммутативное кольцо и $x$ -- переменная.
Тогда, $A[x] = \{a_0 + a_1 x + \ldots + a_n x^n \mid n\in \mathbb N\cup\{0\},\; a_i\in A\}$ -- множество всех многочленов с коэффициентами из $A$.
Множество $A[x]$ с обычными операциями сложения и умножения многочленов является коммутативным кольцом.
\end{enumerate}
\end{examples}

\begin{remark}
Если нам дано кольцо $R$, то в нем выполняются некоторые естественные свойства, которые не включены в список аксиом, но которые полезно знать.
Я хочу здесь привести небольшой список тех свойств, которые полезно иметь в виду:
\begin{enumerate}
\item Для любого элемента $x\in R$ выполнено $0 x = x 0  = 0$.

\item Для любых элементов $x, y\in R$ выполнено $x - (-y) = x + y$.

\item Для любого элемента $x\in R$ выполнено $(-1) x = - x$.

\item Для любых обратимых элементов $x,y\in R$ выполнено $(xy)^{-1} = y^{-1}x^{-1}$.
\end{enumerate}
\end{remark}

\begin{definition}
Пусть $R$ -- кольцо.
Я собираюсь определить понятие подкольца $T\subseteq R$.
\begin{itemize}
\item\textbf{Данные:} 
\begin{enumerate}
\item Подмножество $T\subseteq R$.
\end{enumerate}

\item\textbf{Аксиомы:}
\begin{enumerate}
\item $(T, +)\subseteq (R, +)$ является подгруппой.

\item $T$ замкнуто относительно умножения.

\item $T$ содержит $1$.
\end{enumerate}
\end{itemize}
\end{definition}

\begin{examples}
\begin{enumerate}
\item Множество целых чисел $\mathbb Z\subseteq \mathbb R$ является подкольцом в поле вещественных чисел.

\item Верхнетреугольные матрицы являются подкольцом в кольце квадратных матриц.

\item Скалярные матрицы являются подкольцом кольца квадратных матриц.
\end{enumerate}
\end{examples}

\subsection{Элементы кольца}

Существуют разные подходы к изучению колец.
Самый простой -- поэлементный, то есть мы смотрим на то какого сорта элементы присутствуют в кольце.
Давайте разберем, какие интересные классы элементов имеются.

\begin{definition}
Пусть $R$ -- некоторое кольцо и $x\in R$.
\begin{itemize}
\item Элемент $x$ называется обратимым, если он обратим относительно операции умножения, то есть  $y\in R$ такой, что $xy = yx = 1$.
В этом случае $y$ обозначается $x^{-1}$.
Множество всех обратимых элементов в кольце $R$ обозначается $R^*$.

\item Элемент $x$ называется левым делителем нуля, если найдется ненулевой элемент $y\in R$ такой, что $xy = 0$.
Аналогично, $x$ называется правым делителем нуля, если найдется ненулевой $y\in R$ такой, что $yx = 0$.
Множество левых и правых делителей нуля обозначается  $D_l(R)$ и $D_r(R)$ соответственно.
Множество всех делителей нуля это $D(R) = D_l(R) \cup D_r(R)$.

\item Элемент $x$ называется нильпотентом, если $x^n = 0$ для некоторого $n\in \mathbb N$.
Множество всех нильпотентов кольца $R$ будем обозначать $\operatorname{nil}(R)$.

\item Элемент $x$ называется идемпотентом если $x^2 = x$.
Множество всех идемпотентов кольца $R$ обозначается $E(R)$.
\end{itemize}
\end{definition}

\begin{examples}
\begin{enumerate}
\item Пусть $R = \mathbb Z$ -- кольцо целых чисел.
Тогда, $\mathbb Z^* = \{\pm 1\}$, $D(\mathbb Z) = 0$, $\operatorname{nil}(\mathbb Z) =  0$, $E(\mathbb Z) = \{1, 0\}$.

\item Пусть $R = \mathbb R$ -- поле вещественных чисел.
Тогда, $\mathbb R^* = \mathbb R\setminus\{0\}$, $D(\mathbb R) = 0$, $\operatorname{nil}(\mathbb R) = 0$, $E(\mathbb R) = \{1, 0\}$.

\item Пусть $R = \operatorname{M}_n(\mathbb R)$ -- кольцо квадратных матриц.
Тогда, $\operatorname{M}_n(\mathbb R)^* = \operatorname{GL}_n(\mathbb R)$, $D(\operatorname{M}_n(\mathbb R))$ -- множество вырожденных матриц, $\operatorname{nil}(\operatorname{M}_n(\mathbb R)$ является множеством матриц с нулевыми комплексными собственными значениями, $E(\operatorname{M}_n(\mathbb R))$ -- множество проекторов, оно описывается как множество матриц вида $C^{-1}DC$, где $D$ диагональная с элементами $1$ и $0$ на диагонали.

\item Пусть $R = \mathbb Z_n$ и $n = p_1^{k_1}\ldots p_r^{k_r}$, где $p_i$ -- различные простые числа и $k_i > 0$.
Тогда, $\mathbb Z_n^* = \{k\in \mathbb Z_n\mid (k, n) = 1\}$, $D(\mathbb Z_n) = \{k\in \mathbb Z_n\mid (k, n) \neq 1\}$, $\operatorname{nil}(\mathbb Z_n) = \{k\in \mathbb Z_n\mid p_1|k,\ldots,p_r|k\}$.
По Китайской теореме об остатках, существуют числа $e_i\in \mathbb Z_n$ такие, что $e_i = 1 \pmod{p_i^{k_i}}$ и $e_i = 0\pmod{p_j^{k_j}}$ если $j\neq i$.
Тогда, $E(\mathbb Z_n)$ состоит из всех сумм вид $\sum_{t} e_{i_t}$.
Пустая сумма соответствует нулю, а сумма всех -- единице.
\end{enumerate}
\end{examples}

\subsection{Идеалы}

Другой подход к изучению колец заключается в изучении специальных подмножеств кольца.
К таким подмножествам относятся подкольца, которые мы уже определили, и идеалы.
Давайте разберемся с понятием идеала.

\begin{definition}
Пусть $(R, +, \cdot)$ -- кольцо.
Я собираюсь определить идеал $I$ в кольце $R$.
\begin{itemize}
\item\textbf{Данные:} 
\begin{enumerate}
\item Подмножество $I\subseteq R$.
\end{enumerate}

\item\textbf{Аксиомы:}
\begin{enumerate}
\item $(I, +)\subseteq (R,+)$ -- подгруппа.

\item для любого $r\in R$ выполнено
\[
r I = \{rx\mid x\in I\} \subseteq I\quad\text{и}\quad Ir = \{xr\mid x\in I\}\subseteq I
\]
\end{enumerate}
\end{itemize}
В этом случае говорят, что $I$ -- идеал в $R$.
Подмножества $0$ и $R$ всегда являются идеалами и называются тривиальными идеалами кольца $R$.

Если выполняется первая аксиома и условие $r I = \{rx\mid x\in I\}$ для любого $r\in R$, то говорят, что $I$ левый идеал.
Аналогично, если выполняется первая аксиома и условие $ Ir = \{xr\mid x\in I\}$, то говорят, что $I$ -- правый идеал.
В коммутативных кольцах нет разницы между левыми и правыми идеалами.
\end{definition}

Стоит отметить, что в некоммутативном кольце не достаточно проверять только одно включение из двух $rI \subseteq I$ и $Ir \subseteq I$.


\begin{claim}
\label{claim::ZIdeals}
Пусть $R = \mathbb Z$, тогда все идеалы имеют вид $n\mathbb Z$ для некоторого неотрицательного $n\in \mathbb Z$.
\end{claim}
\begin{proof}
Пусть $I\subseteq \mathbb Z$ -- некоторый идеал.
Тогда $(I, +)$ как минимум является подгруппой в $(\mathbb Z, +)$.
По утверждению~\ref{claim::Zsubgroups}, мы уже знаем, что $I = n\mathbb Z$ для некоторого неотрицательного $n\in \mathbb Z$.
С другой стороны, возьмем произвольную подгруппу  $I = n\mathbb Z$ и число $k\in \mathbb Z$.
Тогда,
\[
k I = \{k x \mid x\in n\mathbb Z\} = \{kn m\mid m\in \mathbb Z\} = kn\mathbb Z\subseteq n\mathbb Z
\]
То есть любая подгруппа по сложению является идеалом.
\end{proof}


\begin{claim}
\label{claim::ZnIdeals}
Пусть $R = \mathbb Z_n$, тогда любой идеал единственным образом представляется в виде $k\mathbb Z_n$ для некоторого $k|n$.
\end{claim}
\begin{proof}
Прежде всего давайте покажем, что множество $k \mathbb Z_n$ является идеалом.
По утверждению~\ref{claim::Znsubgroups}, $k \mathbb Z_n$ -- подгруппа по сложению в $\mathbb Z_n$.
Потому мы только должны показать, что для любого $a\in \mathbb Z_n$ и любого $x\in k \mathbb Z_n$ их произведение $a x \pmod n$ тоже в $k \mathbb Z_n$.
Положим $a x = r \pmod n$.
Тогда $a x = q n + r$.
Так как $k|x$ и $k | n$, то $r$ делит $k$.
Последнее означает, что $r$ принадлежит $k \mathbb Z_n$ и все доказано.

В обратную сторону, пусть $I\subseteq \mathbb Z_n$ -- некоторый идеал.
Тогда как минимум он является подгруппой в $\mathbb Z_n$ по сложению.
По утверждению~\ref{claim::Znsubgroups}, любая подгруппа по сложению в $\mathbb Z_n$ имеет вид $k \mathbb Z_n$ для некоторого $k | n$, и все доказано.
\end{proof}

\subsection{Гомоморфизмы и кольца}

Мы использовали понятие гомоморфизма групп чтобы <<сравнивать>> разные группы, а понятие изоморфизма позволяло нам сказать, что две группы одинаковые.
Давайте распространим эти определения на случай колец.

\begin{definition}
Пусть $(R, +, \cdot)$ и $(S, +, \cdot)$ -- кольца.
Я собираюсь определить гомоморфизм колец $\phi\colon R\to S$.
\begin{itemize}
\item\textbf{Данные:} 
\begin{enumerate}
\item Отображение $\phi\colon R\to S$.
\end{enumerate}

\item\textbf{Аксиомы:}
\begin{enumerate}
\item $\phi(a + b) = \phi(a) + \phi(b)$ для всех $a, b\in R$.

\item $\phi(ab) = \phi(a)\phi(b)$ для всех $a,b\in R$.

\item $\phi(1) = 1$.
\end{enumerate}
\end{itemize}
В этом случае будем говорить, что $\phi$ -- гомоморфизм из кольца $R$ в кольцо $S$.
Если дополнительно мы имеем
\begin{itemize}
\item[]
\begin{enumerate}
\setcounter{enumi}{3}
\item $\phi$ биективно
\end{enumerate}
\end{itemize}
тогда $\phi$ называется изоморфизмом.
В этом случае кольца $R$ и $S$ называются изоморфными.
\end{definition}

Хочу сделать замечание, что это не самое общее определение гомоморфизма колец, но именно оно подходит для наших нужд больше всего.

\begin{remarks}
\begin{enumerate}
\item Отметим, что если $\phi\colon R\to S$ -- гомоморфизм колец, то как минимум $\phi\colon (R,+) \to (S, +)$ является гомоморфизмом абелевых групп.
В частности, $\phi(0) = 0$ и $\phi(-a) = - \phi(a)$ по утверждению~\ref{claim::HomGrProp}.

\item Если $R$ и $S$ -- изоморфные кольца, то это по по сути означает, что кольца $R$ и $S$ одинаковые.
Мы уже обсуждали как понимать изоморфизм в случае групп (см.~обсуждение после определения~\ref{def::IsomorphismGr}).
Кратко изоморфизм можно рассматривать как правило переименования элементов $R$ в имена элементов $S$ и при этом переименовании сложение и умножение на $R$ превращается в сложение и умножение на $S$.
Изоморфные кольца имеют одинаковые свойства.
\end{enumerate}
\end{remarks}

\begin{examples}
\begin{enumerate}
\item Отображение  $\mathbb Z\to \mathbb Z_n$ по правилу $k\mapsto k \pmod n$ является гомоморфизмом колец.

\item Отображение $\mathbb R\to \operatorname{M}_n(\mathbb R)$ по правилу $\lambda \mapsto \lambda E$, где $E$ -- единичная матрица, является гомоморфизмом колец.

\item Отображение $\mathbb R[x]\to \mathbb C$ по правилу $f(x) \mapsto f(i)$, где $i^2 = -1$, является гомоморфизмом колец.

\item Отображение $\mathbb C\to \operatorname{M}_n(\mathbb R)$ по правилу $a + bi \mapsto \left(\begin{smallmatrix}{a}&{-b}\\{b}&{a}\end{smallmatrix}\right)$ является гомоморфизмом колец.
\end{enumerate}
\end{examples}


\begin{claim}
[Китайская теорема об остатках]
Пусть $n$ и $m$ -- взаимно простые натуральные числа, то есть $(n,m) = 1$.
Тогда отображение
\[
\Phi \colon \mathbb Z_{mn} \to \mathbb Z_m \times \mathbb Z_n,\quad k\mapsto (k\!\!\mod m, k\!\!\mod n)
\]
является изоморфизмом колец.
\end{claim}
\begin{proof}
По сути мы это уже с вами доказали, только не осознали этого.
Мы уже знаем, что $\Phi\colon (\mathbb Z_{mn}, +) \to (\mathbb Z_m\times\mathbb Z_n, +)$ является изоморфизмом абелевых групп по утверждению~\ref{claim::Chinese}.
Кроме того, мы уже проверили, что $\Phi$ сохраняет умножение и единицу при доказательстве утверждения~\ref{claim::ChineseMult}.
А это ровно то, что надо было сделать для доказательства этого утверждения.
\end{proof}

\begin{definition}
Пусть $\phi\colon R\to S$ -- гомоморфизм колец.
Тогда
\begin{itemize}
\item Ядро $\phi$ -- это $\ker\phi = \{r\in R\mid \phi(r) = 0\}\subseteq R$.

\item Образ $\phi$ -- это $\Im \phi = \{\phi(r) \mid r\in R\} = \phi(R)\subseteq S$.
\end{itemize}
\end{definition}

\begin{claim}
\label{claim::RingHomProp}
Пусть $\phi\colon R\to S$ -- гомоморфизм колец.
Тогда
\begin{enumerate}
\item $\Im\phi\subseteq S$ -- подкольцо.

\item $\ker \phi\subseteq R$ -- идеал.

\item Отображение $\phi$ сюръективно тогда и только тогда, когда $\Im\phi = S$.

\item Отображение $\phi$ инъективно тогда и только тогда, когда $\ker \phi = \{0\}$.
\end{enumerate}
\end{claim}
\begin{proof}
1) Мы уже знаем, что $\Im \phi$ является аддитивной подгруппой в $(S, + )$ по утверждению~\ref{claim::HomProp}.
По определению гомоморфизма $1 = \phi(1) \in \Im \phi$.
Значит осталось лишь показать, что образ замкнут относительно умножения.
Действительно, если $x,y\in \Im\phi$, тогда $x = \phi(a)$ и $y = \phi(b)$ для некоторых $a,b\in R$.
Тогда,
\[
xy = \phi(a) \phi(b) = \phi(ab) \in \Im \phi
\]

2) Мы уже знаем, что $\ker \phi$ является подгруппой в $(R, +)$ по утверждению~\ref{claim::HomProp}.
Значит нам лишь надо показать, что она устойчива по умножению на любой элемент кольца $R$.
Пусть $x\in \ker \phi$ и $r\in R$, мы должны показать, что  $rx, xr \in \ker \phi$, то есть $\phi(rx) = 0$ и $\phi(xr) = 0$.
Действительно, $\phi(rx) = \phi(r)\phi(x) = \phi(r) 0 = 0$ и аналогично $\phi(xr) = 0$.

3) Этот пункт выполнен по определению.

4) Так как $\phi\colon (R,+)\to (S,+)$ -- гомоморфизм групп, то результат следует из утверждения~\ref{claim::HomProp}.
\end{proof}

