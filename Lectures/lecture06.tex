\ProvidesFile{lecture06.tex}[Lecture 6]


\section{Многочлены от одной переменной}

\subsection{Определение}

Пусть $F$ -- поле.
Многочлен $f$ от переменной $x$ -- это картинка следующего вида
\[
f = a_0 + a_1 x + \ldots + a_n x^n,\quad \text{где } a_i \in F
\]
Здесь операции $+$ и $\cdot$ -- это просто символы.
Таким образом мы берем какие-то элементы $a_i$ из поля $F$ и составляем строку символов на листе бумаги как выше.
Формально многочлен -- это просто последовательность его коэффициентов $(a_0,\ldots,a_n)$.
Мы так же можем считать, что у такого многочлена есть коэффициенты $a_{n+1},a_{n+2},\ldots$ и они все равны нулю.
То есть можно считать, что у многочлена счетное число коэффициентов, но только конечное число из этих коэффициентов не равна нулю.
Эта точка зрения очень удобна, особенно когда надо писать какие-то формулы в общем виде.

Пусть у нас заданы два многочлена
\[
f = a_0 + a_1 x + \ldots + a_n x^n\text{ и } g = b_0 + b_1 x + \ldots + b_m x^m
\]
мы скажем, что многочлены $f$ и $g$ равны, если $a_k = b_k$ для всех $k\in \mathbb N$.%
\footnote{Здесь я использую замечание про степени многочленов, то есть предполагаю что $a_k = 0$ для $ k > n$ и аналогично $b_k = 0$ для $k > m$.}
Мы можем складывать и умножать многочлены используя следующие правила
\begin{gather*}
f = \sum_{k=0}^n a_k x^k\quad\text{и}\quad g = \sum_{k=0}^m b_k x^k\\
f + g = \sum_{k \geqslant 0} (a_k + b_k) x^k\quad\quad
f g = \sum_{k\geqslant 0}\Bigl( \sum_{u + v = k} a_u b_v\Bigl) x^k
\end{gather*}
Множество всех многочленов с коэффициентами в $F$ от переменной $x$ обозначается через $F[x]$.
Прямая проверка показывает, что $F[x]$ с операциями сложения и умножения образуют коммутативное кольцо.

\begin{remark}
Давайте сделаем одно важное замечание.
Каждый многочлен $f\in F[x]$ определяет функцию $\hat f\colon F\to F$ по правилу $a \mapsto f(a)$.
А именно, если $f = a_0 + a_1 x + \ldots + a_n x^n$, то $f(a) = a_0 + a_1 a + \ldots + a_n a^n$.
Однако, в общем случае $f$ не определяется однозначно полученной функцией $\hat f$.
Действительно, предположим $F = \mathbb Z_2$.
Тогда многочлены $f_n = x^n$ определяют одну и ту же функцию $\hat f_n \colon \mathbb Z_2 \to \mathbb Z_2$ отправляющую $0\mapsto 0$ и $1\mapsto 1$.
Так что все полиномы $f_n$ различны, но определяют одну и ту же функцию.
Этот пример объясняет почему мы дали такое странное определение многочленам.
Мы хотим, чтобы многочлен однозначно определялся именно своими коэффициентами, а для этого на них надо смотреть как на формальные картинки, а не как на функции.
Такой формальный подход не мешает при этом по каждому многочлену построить функцию.
\end{remark}

Если $f\in F[x]$ представлен в виде $f = a_0 + a_1 x + \ldots + a_n x^n$, где $a_n \neq 0$, то есть $n$ -- старший индекс с ненулевым коэффициентом в $f$.
Тогда $n$ называется степенью многочлена $f$ и обозначается $\deg f$.
Коэффициент $a_n$ называется старшим коэффициентом многочлена.
Степень нулевого многочлена надо определить отдельно, так как в нем нет ненулевых коэффициентов.
Мы будем предполагать, что $\deg (0) = -\infty$.
Многочлен $f\in F[x]$ принадлежит полю $F$ тогда и только тогда, когда $\deg (f) \leqslant 0$, при этом $f$ ненулевая константа тогда и только тогда, когда $\deg f = 0$.%
\footnote{Здесь под константой понимаетя элемент поля $F$.}

\begin{claim}
\label{claim::Degree}
Пусть $F$ -- поле.
Тогда для любых $f, g \in F[x]$, имеем $\deg(fg) = \deg (f) + \deg(g)$.%
\footnote{Тут мы придерживаемся правила, что $-\infty + a = a + (-\infty) = -\infty$.}
\end{claim}
\begin{proof}
Если хотя бы один из многочленов нулевой, обе части равенства содержат $-\infty$ и доказывать нечего.
Поэтому можно считать, что $f$ и  $g$ ненулевые.
Пусть 
\[
f = a_n x^n + a_{n-1}x^{n-1} + \ldots + a_1 x + a_0\quad \text{и} \quad g = b_mx^m + b_{m-1}x^{m-1} + \ldots + b_1 x + b_0
\]
такие, что $a_n\neq 0$ и $b_m\neq0$.
Тогда
\[
fg = a_n b_m x^{m+n} + h(x)
\]
где $\deg h < m + n$.
Так как $F$ -- поле, $a_n b_m \neq 0$.
Значит $\deg fg = n + m = \deg f + \deg g$.
\end{proof}

\begin{claim}
\label{claim::PolyZeroDiv}
Пусть $F$ -- поле.
Единственный делитель нуля в $F[x]$ -- нулевой многочлен.
\end{claim}
\begin{proof}
Если $f, g\in F[x]$, $f\neq 0$, и $g\neq 0$, тогда $\deg(f) \geqslant 0$ и $\deg(g) \geqslant 0$.
Значит $\deg(fg) = \deg(f) + \deg(g)\geqslant 0$.
В частности $fg \neq 0$.
\end{proof}

\begin{claim}
\label{claim::PolyInvert}
Пусть $F$ -- поле.
Многочлен $f\in F[x]$ обратим тогда и только тогда, когда $f\in F^*$.
\end{claim}
\begin{proof}
Если $f$ обратим, то $fg = 1$ для некоторого $g\in F[x]$.
Тогда, $0 = \deg(1) = \deg(fg) = \deg(f) + \deg(g)$.
Так как степени ненулевых многочленов неотрицательны, отсюда следует, что $\deg(f) = \deg(g) = 0$.
Последнее означает, что $f$ и $g$ ненулевые константы.
\end{proof}


\subsection{Алгоритм Евклида}

Если $F$ -- некоторое поле, то в кольце $F[x]$ определено деление с остатком.
Если $f, g \in F[x]$ и $g\neq 0$, то мы можем разделить $f$ на $g$ с остатком.
Последнее означает, что существуют единственные $q, r\in F[x]$ такие, что $f = q g + r$ и $\deg(r) < \deg (g)$.
Многочлен $q$ называется частным, а $r$ остатком.

Предположим $f, g\in F[x]$, будем говорить, что $f$ делит $g$, если $g = fh$ для некоторого $h\in F[x]$.
Стоит отметить, что любой многочлен делит $0$.
Так же, если $a\in F^*$, то $f$ делит $af$ и наоборот, так как $F$ -- поле.

\begin{definition}
Пусть $F$ -- поле.
Многочлен $f\in F[x]$ называется унитальным, если его старший коэффициент равен единице $1$.
\end{definition}

\begin{definition}
Пусть $F$ -- поле и $f, g\in F[x]$ -- некоторые многочлены.
Многочлен $d\in F[x]$ называется наибольшем общим делителем $f$ и $g$, если
\begin{enumerate}
\item $d$ делит и $f$ и $g$.

\item если $h$ делит одновременно $f$ и $g$, то $h$ делит $d$.

\item $d$ унитальный если не равен нулю.
\end{enumerate}
\end{definition}


\begin{claim}
\label{claim::PolyIdeals}
Пусть $F$ -- поле и $I\subseteq F[x]$ -- некоторый идеал.
Тогда $I = f F[x] = \{fh\mid h\in F[x]\}$ для некоторого $f\in F[x]$.
\end{claim}
\begin{proof}
Если $I$ состоит только из нулевого многочлена, то $I = 0 F[x]$.
Пусть теперь $f\in I$ -- ненулевой многочлен минимальной степени в $I$.
Возьмем произвольный $h\in I$ и разделим $h$ с остатком на $f$, получим $h = qf + r$, где $\deg r < \deg f$.
Тогда, $r = h - qf \in I$ и имеет меньшую степень, чем $f$.
Так как $f$ ненулевой многочлен минимальной степени в $I$, то $r$ обязан быть нулевым.
А это означает, что все многочлены из $I$ делятся на $f$, что и требовалось.
\end{proof}


Небольшое замечание по поводу обозначений.
Идеал $fF[x]$ обычно обозначается $(f)$ для краткости.

\begin{claim}
\label{claim::PolyGCD}
Пусть $F$ -- некоторое поле и $f,g\in F[x]$.
Тогда
\begin{enumerate}
\item Существует наибольший общий делитель $d$ многочленов $f$ и $g$.
Кроме того, существуют такие многочлены $u,v\in F[x]$, что $d = uf + v g$.

\item Наибольший общий делитель $f$ и $g$ единственный.
\end{enumerate}
\end{claim}
\begin{proof}
(1) Рассмотрим следующее множество многочленов
\[
I = \{a f + b g \mid a, b \in F[x]\}\subseteq F[x]
\]
Это подмножество является идеалом в $F[x]$.
По утверждению~\ref{claim::PolyIdeals}, найдется унитальный многочлен $r\in I$ такой, что $I = r F[x]$.
Так как $r\in I$, то $r = uf + vg$ для некоторых $u, v\in F[x]$.
Из того, что $f, g \in I = r F[x]$ следует, что $f = s r$ и $g = t r$ для некоторых $s, t \in F[x]$.

Теперь давайте докажем, что $r$ является наибольшим общим делителем.
Равенства $f = s r$ и $g = t r$ означают, что $r$ делит $f$ и $g$, то есть $r$ -- общий делитель.
Пусть теперь $h$ делит $f$ и $g$, тогда $h$ делит оба слагаемых в правой части следующего равенства $r = uf + vg$.
Значит $h$ делит $r$.
Мы проверили, что все свойства наибольшего общего делителя выполнены для $r$.


(2) Предположим, что у нас есть два наибольших общи делителя $d_1$ и $d_2$.
Тогда $d_1$ делит $d_2$ потому что $d_2$ -- наибольший общий делитель.
И наоборот, $d_2$ делит $d_1$, потому что $d_1$ -- наибольший общий делитель.
Значит
\[
d_1 = a d_2\quad d_2 = b d_1
\]
В частности, $d_1 = ab d_1$.
Значит, $d_1(1 - ab)= 0$ в кольце $F[x]$.
Но в кольце $F[x]$ нет ненулевых делителей нуля по утверждению~\ref{claim::PolyZeroDiv}.
Значит $1 - ab = 0$ и потому $1 = ab$.
Следовательно $a$ и $b$ -- обратимые элементы поля $F$.
\end{proof}

\begin{remarks}
\begin{itemize}
\item я хочу явно проговорить некоторые крайние случаи в определении наибольшего общего делителя.
Например, если $f = g = 0$, то наибольший общий делитель будет $0$.
Действительно, в этом случае любой многочлен делит $f$ и $g$.
А $0$ -- это делитель $f$ и $g$ делимый всеми многочленами.

\item Если $f \neq 0$ и $g = 0$, тогда наибольший общий делитель -- это $f$ с нормализованным старшим коэффициентом, потому что в этом случае $f$ делит $g$.
\end{itemize}
\end{remarks}


\begin{claim}
Пусть $F$ -- поле и $f, g, h\in F[x]$ -- некоторые многочлены.
Тогда $(f, g) = (f, g - hf)$.
\end{claim}
\begin{proof}
Действительно, множество делителей у пары $\{f, g\}$ такое же, как и у пары $\{f, g-hf\}$.
В частности максимальные элементы (относительно порядка делимости) тоже совпадают, то есть совпадают наибольшие общие делители.
\end{proof}

Последнее утверждение позволяет нам считать наибольший общий делитель эффективно с помощью алгоритма Евклида.

\paragraph{Дано:}

Два многочлена $f, g\in F[x]$.
Здесь $F$ -- поле.

\paragraph{Результат:}

Наибольший общий делитель $d\in F[x]$ многочленов $f$ и $g$.

\paragraph{Алгоритм}

Используются две временные переменные $u, v\in F[x]$.

\begin{enumerate}
\item Инициализируем $u = f$, $v = g$ в случае $\deg f \geqslant \deg g$ и $u = g$, $v = f$ в противном случае.

\item Пока $v \neq 0$ выполняем следующее:
\begin{enumerate}
\item Делим $u$ на $v$ с остатком $u = q v + r$.

\item делаем замену $u = v$, $v = r$.
\end{enumerate}

\item Когда $v = 0$, $u$ становится наибольшим общим делителем $f$ и $g$.
\end{enumerate}


\subsection{Однозначное разложение на множители}

\begin{definition}
\begin{itemize}
\item Многочлен $ f\in F[x]\setminus F$ называется приводимым если существуют многочлены $g,h\in F[x]$ такие, что $f = gh$, $0<\deg (g) < \deg (f)$ и $0 < \deg(h) < \deg(f)$.

\item Многочлен $f\in F[x]\setminus F$ называется неприводимым, если для любых $g,h\in F[x]$ таких, что $f = gh$, либо $g$ либо $h$ является ненулевой константой.
\end{itemize}
\end{definition}

Надо отметить, что все ненулевые многочлены бьются на три класса: 1) обратимые многочлены, то есть $F^*$, 2) приводимые многочлены, 3) неприводимые многочлены.

\begin{claim}
[UFD]
\label{claim::PolyUFD}
Пусть $F$ -- поле.
Тогда каждый элемент $f\in F[x]\setminus F$ представляется в форме $f = a p_1^{k_1}\ldots p_n^{k_n}$, где $a\in F$ -- ненулевая константа, $k_i$ -- положительные целые числа, и $p_i$ -- различные унитальные неприводимые многочлены.
При этом такое представление единственное с точностью до перестановки множителей.
\end{claim}

Я не очень хочу доказывать это утверждение, доказательство для многочленов полностью повторяет доказательство для целых чисел.
Ключевым техническим утверждением является утверждение~\ref{claim::PolyGCD}, а именно нам важно, что наибольший общий делитель можно представить в виде линейной комбинации исходных многочленов.
Тем не менее, я хочу доказать следующий частный случай общего результата.

\begin{claim}
Пусть $F$ -- поле, $f, g\in F[x]$ -- два взаимно простых многочлена и $h\in F[x]$ делится на $f$ и на $g$.
Тогда, $h$ делится на $fg$.
\end{claim}
\begin{proof}
Так как $f$ и $g$ взаимно просты, то $1 = uf + vg$ для некоторых $u,v\in F[x]$ по утверждению~\ref{claim::PolyGCD} пункт~(1).
Умножая это равенство на $h$ получим $h = u hf + v hg$.
Так как $g$ делит $h$, $gf$ делит $uhf$.
Так как $f$ делит $h$, $fg$ делит $vhg$.
Значит, $fg$ делит $h$.
\end{proof}

\subsection{Кольцо остатков}

Теперь мы готовы познакомиться с очень важной конструкцией в алгебре -- кольцо полиномиальных остатков.

Пусть $F$ -- поле и $f\in F[x]$ -- некоторый многочлен.
Я собираюсь определить кольцо $F[x]/(f)$.
Чтобы определить кольцо, я должен определить множество и две операции на нем -- сложение и умножение.
А после этого надо проверить все необходимые аксиомы кольца.
Начнем с тривиального случая.
Пусть $f = 0$, тогда по определению $F[x]/(f)$ -- это кольцо многочленов $F[x]$ с обычными операциями.
Теперь интересный случай, когда $f \neq 0$:
\begin{itemize}
\item $F[x]/(f) = \{g \in F[x]\mid \det g < \deg f\}$ -- множество остатков от деления на многочлен $f$.

\item $+\colon F[x]/(f)\times F[x]/(f) \to F[x]/(f)$ -- обычное сложение многочленов.

\item $\cdot \colon F[x]/(f)\times F[x]/(f) \to F[x]/(f)$ -- умножение по модулю многочлена $f$, а именно: для любых $g, h\in F[x]/(f)$, мы определяем умножение, как  $gh \pmod{f}$.
Последнее означает, что мы должны в начале посчитать обычное произведение многочленов $gh$, а потом найти остаток от деления на $f$, то есть $gh = q f + r$.
В этом случае произведением $g$ и  $h$ является $r$.
\end{itemize}

\begin{claim}
Если $F$ -- поле и $f\in F[x]$ -- некоторый многочлен, тогда множество $F[x]/(f)$ с введенными на нем операциями является коммутативным кольцом.
\end{claim}
\begin{proof}
Если $f = 0$, это ясно, так как $F[x]/(f) = F[x]$ по определению.
Предположим, что $f \neq 0$.
Множество $F[x]/(f) = \{g\in F[x]\mid \deg g < \deg f\}$ с операцией сложения является абелевой группой, так как это подгруппа в $(F[x], +)$.

Теперь мы мы должны показать: 1) дистрибутивный закон, 2) ассоциативность умножения, 3) существование нейтрального элемента по умножению, 4) коммутативность умножения.

1) Если $g,h,p\in F[x]/(f)$, нам надо показать, что
\[
(g+h)p\!\!\mod{f} = gp\!\!\mod{f} + hp\!\!\mod{f}\quad\text{and}\quad
g(h+p)\!\!\mod{f} = gh\!\!\mod{f} + gp\!\!\mod{f}
\]
Мы покажем первое равенство.
Второе будет следовать из коммутативности.
Давайте разделим $gp$ и $hp$ на $f$ с остатком и получим
\[
gp = q_1 f + r_1,\;\deg r_1 < \deg f,\;\text{и}\;hp = q_2 f + r_2,\;\deg r_2 < \deg f
\]
Теперь, правая часть по определению совпадает с $r_1 + r_2$.
С другой стороны, выражение
\[
(g+h)p = (q_1 + q_2)f + r_1 + r_2
\]
является делением $(g+h)p$ на $f$ с остатком равным $r_1 + r_2$.
Следовательно, левая часть равна тому же самому.

2) Если $g,h,p\in F[x]/(f)$, тогда 
\[
(g\cdot  (h\cdot  p\!\!\mod{f}))\!\!\mod{f}= (g\cdot  h\cdot  p)\!\!\mod{f} = ((g \cdot h\!\!\mod{f}) \cdot p)\!\!\mod{f}
\]
Я позволю себе оставить эту проверку читателю, как упражнение в абстрактной чепухе.

3) Многочлен $1$ является нейтральным элементом по умножению.

4) Коммутативность произведения по модулю $f$ следует из его определения.
\end{proof}

\begin{remarks}
\begin{itemize}
\item Обратите внимание, хотя мы можем рассматривать  $F[x]/(f)$ как подмножество в $F[x]$ (даже как абелеву подгруппу по сложению), однако, так делать не следует.
Основная причина в том, что это вложение не согласовано с умножением (когда $f \neq 0$).
Последнее означает, что $F[x]/(f)$ НЕ является подкольцом в $F[x]$.

\item Отображение $F[x]\to F[x]/(f)$ по правилу $g\mapsto g\!\!\mod{f}$ является сюръективным гомоморфизмом.
\end{itemize}
\end{remarks}

\begin{claim}
\label{claim::PolyRemIdeals}
Пусть $F$ -- поле, $f\in F[x]$ -- многочлен и $I\subseteq F[x]/ (f)$ -- некоторый идеал.
Тогда существует многочлен $g\in F[x]$ делящий $f$ такой, что $I = (g) = \{g h\!\!\mod{f}\mid h\in F[x]\}$.
\end{claim}
\begin{proof}
Случай $f = 0$ разобран в утверждении~\ref{claim::PolyIdeals}.
Теперь мы предположим, что $f \neq 0$.
Если $I$ состоит только из нулей, то $I = (f)$ и все доказано.

Пусть $h\in I$ -- ненулевой многочлен минимально возможной степени в $I$.
Тогда для любого $g\in I$, разделим $g$ на $h$ с остатком и получим $g = qh + r$ где $\deg r < \deg h$.
Также $r = g - qh\in I$.
Так как $h$ был ненулевым многочленом минимально возможной степени в $I$, такое может быть только если $r = 0$.
Значит $h$ делит любой $g\in I$.
последнее означает, что $I = (h)$.

Теперь мы должны показать, что $h$ делит $f$.
Давайте разделим $f$ на $h$ с остатком, получим $f = qh + r$, где $\deg r < \deg h$.
Это означает, что $r = -qh$ в кольце $F[x]/(f)$.
В частности $r\in I$ и имеет степень меньше, чем $h$.
Такое возможно только если $r = 0$, что завершает доказательство.
\end{proof}


%
%\begin{claim}\label{claim::PolyQuotField}
%Let $F$ be a field, $f\in F[x]\setminus F$ be a polynomial. The ring $F[x]/(f)$ is a field if and only if $f$ is irreducible nonzero polynomial.
%\end{claim}
%\begin{proof}
%Suppose that $f$ is reducible. Then $f = gh$, where $\deg g < \deg f$ and $\deg h < \deg f$. Then $h\neq 0$ in $F[x]/(f)$ as well as $g\neq 0$ in $F[x]/(f)$. But $gh = f = 0$ in $F[x]/(f)$. Hence, $g$ and $h$ are nonzero zero divisors. But zero divisors are not invertible. The latter contradicts to the definition of a field.
%
%If $f$ is irreducible we should show that any nonzero $g\in F[x]/(f)$ is invertible. Since $\deg g < \deg f$ and $g\neq 0$, $f$ and $g$ are coprime. Hence, $1 = (g, f)$. By Claim~\ref{claim::PolyGCD} item~(3), it follows that $1 = ug + vf$ for some $u,v\in F[x]$. But the latter means that $1 = ug\pmod f$ and hence, $u = g^{-1}$ in $F[x]/(f)$.
%\end{proof}
%
%\begin{remarks}
%\begin{itemize}
%
%\item It is also worth mentioning that the element $x\in F[x]/(p)$ is a root of the polynomial $p$ in the field $F[x]/(p)$. Indeed, $p(x) = 0$ in $F[x]/(p)$ be definition.
%
%\item Let us consider the field of real numbers $\mathbb R$. Then the polynomial $x^2 + 1\in \mathbb R[x]$ is irreducible. Hence $\mathbb R[x]/(x^2 +1)$ is a field and the element $x$ becomes a root of $x^2 + 1$. Explicitly elements of $\mathbb R[x]/(x^2 + 1)$ are of the form $a + bx$, where $a, b\in \mathbb R$ and we also know that $x^2 = -1$. Hence, this is the usual model for the complex numbers.
%
%\item If $\deg f = n$, then the elements $1, x, \ldots, x^{n-1}$ form a basis of $F[x]/(f)$ over the field $F$. In particular, $\dim_F F[x]/(f) = \deg f$.
%\end{itemize}
%\end{remarks}

А вот еще одна версия Китайской теоремы об остатках, только теперь для кольца полиномиальных остатков.

\begin{claim}
[Китайская теорема об остатках]
Пусть $f, g\in F[x]$ -- два взаимно простых многочлена, то есть $(f, g) = 1$.
Тогда отображение
\[
\Phi\colon F[x]/(fg) \to F[x]/(f)\times F[x]/(g)\quad h \mapsto (h\!\!\mod{f},h\!\!\mod{g})
\]
является изоморфизмом колец.
\end{claim}
\begin{proof}
В начале проверим, что отображение является гомоморфизмом колец.
Мы должны показать, что оно сохраняет сложение, умножение и единицу кольца.
Это делается прямым вычислением и я позволю себе пропустить это вычисление.

Теперь покажем, что отображение инъективно.
По утверждению~\ref{claim::RingHomProp} достаточно проверить, что ядро $\Phi$ состоит только из нуля.
Предположим $h\in \ker \Phi$.
Это значит, что $h = 0 \pmod f$ и $h = 0 \pmod g$, то есть $h$ делится на $f$ и $g$.
Так как $f$ и $g$ взаимно просто, последнее означает, что $h$ делится на $fg$.
Но $\deg h < \deg fg$.
Значит такое может быть только если $h = 0$.

Теперь мы хотим показать сюръективность.
Так как $F$ -- поле, мы можем рассматривать $F[x]/(fg)$и $F[x]/(f)\times F[x]/(g)$ как векторные пространства над $F$.
Кроме того, $\Phi$ является линейным отображением потому что по определению $\Phi(\lambda) = \lambda$ и значит $\Phi(\lambda f) = \Phi(\lambda)\Phi(f) = \lambda \Phi(f)$.
Так как $\Phi$ инъективно, то для доказательства сюръективности достаточно показать, что оба пространства имеют одинаковую размерность.
Ясно, что размерность $\dim_F F[x]/(fg) = \deg(fg)$.
С другой стороны, с точки зрения векторных пространств $F[x]/(f)\times F[x]/(g)$ является прямой суммой $F[x]/(f)$ и $F[x]/(g)$.
А значит размерность $F[x]/(f)\times F[x]/(g)$ совпадает с $\deg f + \deg g$.
Теперь результат следует из утверждения~\ref{claim::Degree}.
\end{proof}