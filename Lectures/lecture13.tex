\ProvidesFile{lecture13.tex}[Lecture 13]


\subsection{Теоремы о гомоморфизме}

В этом разделе я хочу обсудить две теоремы, позволяющие эффективно работать с факторами.

\begin{claim}
[Первая теорема о гомоморфизме]
\label{claim::HomoThmGroups}
Пусть $\phi\colon G\to H$ -- гомоморфизм групп.
Тогда существует канонический изоморфизм $\bar \phi \colon G/\ker \phi \to \Im \phi$ по правилу $g\ker \phi \mapsto \phi(g)$.
\end{claim}
\begin{proof}
Давайте я поясню формулу, по которой работает отображение $\bar \phi$.
Оно говорит.
Возьмем какой-нибудь класс эквивалентности из $x\in G/\ker \phi$ и выберем в нем представителя $g$.
После чего рассмотрим $\phi(g)\in H$ и положим его в качестве образа класса эквивалентности $x$.
В чем проблема этого определения?
Проблема в том, что если мы выберем другого представителя, то образ может оказаться другим элементом из $H$, а значит не определен корректно.
Потому в начале нам надо доказать, что $\bar\phi$ действительно корректное отображение.

Возьмем произвольный смежный класс $x\in G/\ker \phi$.
Выберем в нем представителя $g\in x$.
Тогда мы знаем, что $x = g \ker \phi$.
Значит любой другой элемент из этого класса имеет вид $gh$, где $h\in \ker \phi$.
Давайте рассмотрим его образ $\phi(gh) = \phi(g) \phi(h) = \phi(g)$.
Таким образом мы видим, что все представители смежного класса идут в один и тот же элемент $\phi(g)$ для выбранного в самом начале элемента $g\in x$.
Значит отображение корректно.
Кроме того, по определению ясно, что это отображение сюръективно, потому что мы любой элемент вида $\phi(g)$ получим как образ смежного класса $g\ker \phi$.

Теперь давайте покажем, что $\bar \phi$ является гомоморфизмом групп.
Для этого нам надо выбрать два произвольных смежных класса в $G/\ker \phi$, скажем $g_1 \ker \phi$ и $g_2\ker \phi$, и проверить, что
\[
\bar\phi(g_1 \ker \phi g_2 \ker \phi) \stackrel{?}{=} \bar \phi(g_1 \ker \phi) \bar \phi(g_2 \ker \phi)
\]
Давайте посчитаем левую часть
\[
\bar\phi(g_1 \ker \phi g_2 \ker \phi) = \bar\phi(g_1 g_2 \ker \phi) = \phi(g_1g_2)
\]
С другой стороны правая часть будет
\[
\bar \phi(g_1 \ker \phi) \bar \phi(g_2 \ker \phi) = \phi(g_1) \phi(g_2) = \phi(g_1 g_2)
\]
Получили одно и то же, значит $\bar \phi$ действительно гомоморфизм.

Теперь проверим, что это инъективный гомоморфизм.
Для этого достаточно проверить, что $\ker \phi$ состоит только из нейтрального элемента (по утверждению~\ref{claim::HomProp} пункт~(4)).
Возьмем $g\ker \phi \in \ker \bar \phi$.
Тогда по определению $\bar\phi(g \ker \phi) = \phi(g) = e\in H$.
Значит $g\in \ker \phi$.
Но тогда $g\ker \phi = \ker \phi$.
Но этот смежный класс является нейтральным элементом в фактор группе.
То есть инъективность доказана.
\end{proof}

По сути относиться к первой теореме надо так.
Она позволяет описать любую фактор группу.
А именно.
Мы знаем, что подгруппа нормальна тогда и только тогда, когда она является ядром некоторого гомоморфизма.
Для ядер гомоморфизма нормальность мы доказывали в утверждении~\ref{claim::HomProp} пункт~(2).
А для нормальных подгрупп нужный гомоморфизм доставляется отображением факторизации, как показывала конструкция прошлой лекции.
Потому, чтобы описать фактор $G/ N$ надо подобрать гомоморфизм $f\colon G\to H$ так, чтобы $\ker f = N$.
Тогда $G/N$ просто совпадает с $\Im f$ с точностью до изоморфизма.

\paragraph{Примеры}

\begin{enumerate}
\item Рассмотрим группы $\mathbb Z$ и $\mathbb Z_n$ и отображение $\phi\colon \mathbb Z\to \mathbb Z_n$ по правилу $k \mapsto k \pmod n$.
Тогда легко видеть, что это отображение является гомоморфизмом групп.
Этот гомоморфизм сюръективен, так как в любой остаток переходит число равное этому остатку.
А ядро гомоморфизма есть $n\mathbb Z$.
Значит по теореме о гомоморфизме имеем $\mathbb Z/n \mathbb Z = \mathbb Z_n$ при этом $k+n\mathbb Z \mapsto k \pmod n$.

\item Рассмотрим отображение $\sgn \colon \operatorname{S}_n\to \{\pm 1\} = \mathbb Z_2$, которое каждой перестановке сопоставляет ее знак.
Мы знаем, что это отображение уважает умножение, то есть гомоморфизм групп.
Более того образом будет вся правая группа, а ядро -- четные перестановки $\operatorname{A}_n$.
Значит по теореме о гомоморфизме $\operatorname{S}_n / \operatorname{A}_n= \mathbb Z_2$.

\item Рассмотрим отображение $\mathbb C^* \to \mathbb C^*$ по правилу $ z \mapsto \frac{z}{|z|}$.
Тогда образом будет $S^1 = \{z\in \mathbb C^* \mid |z| = 1\}$ -- окружность радиуса один с центром в нуле на комплексной прямой.
А ядром будет множество положительных вещественных чисел $\mathbb R_{>0}$.
Действительно, $z$ идет в $z / |z| =1 $ равносильно тому, что $z = |z|\in \mathbb R$.
То есть $z$ положительное вещественное число.
Таким образом по теореме о гомоморфизме имеем $\mathbb C^* / \mathbb R_{>0} = S^1$.

\item Рассмотрим отображение $\mathbb C^* \to \mathbb R^*$ по правилу $z \mapsto |z|$.
Тогда образом является множество положительных вещественных чисел $\mathbb R_{>0}$.
А ядро -- $S^1$ комплексные числа по модулю равные $1$.
То есть $\mathbb C^* / S^1 = \mathbb R_{>0}$.
\end{enumerate}

Теперь я хочу обсудить еще одну версию теоремы о гомоморфизме, которая позволяет строить гомоморфизмы факторгрупп.

\begin{claim}
[Вторая теорема о гомоморфизме]
\label{claim::HomoThmGroups2}
Пусть $G$ -- некоторая группа, $N\subseteq G$ -- нормальная подгруппа и $\phi\colon G\to H$ -- некоторый гомоморфизм.
Тогда существует гомоморфизм $\gamma \colon G/N \to H$ такой, что $\phi = \gamma \circ \pi$ тогда и только тогда, когда $N\subseteq \ker \phi$.
Изобразим все на диаграмме ниже
\[
\xymatrix{
	{G}\ar[r]^{\phi}\ar[d]_{\pi}&{H}&\\
	{G/N}\ar@{-->}[ru]_{\gamma}&{}
}
\]
Более того в случае существования $\gamma$ обязательно единственное и задано по правилу $\gamma(gN) = \phi(g)$, а ядро $\gamma$ равно $\ker \phi / N$.
\end{claim}
\begin{proof}
В начале покажем, что если $\gamma$ существует, то $N\subseteq \ker \gamma$.
Действительно, пусть $h\in N$, и мы знаем, что $\phi = \gamma \circ \pi$.
Применим к $h$ последнее равенство, получим $\phi(h) = \gamma(\pi(h)) = \gamma(e) = e$.
То есть действительно любой элемент $N$ должен идти в нейтральный элемент $H$ под действием $\phi$.

В обратную сторону.
Пусть $N\subseteq \ker \phi$.
Давайте покажем, что существует требуемое $\gamma$.
Прежде всего определим $\gamma$ по следующему правилу $\gamma(gN) = \phi(g)$.
Нам надо показать, что это правило корректно, то есть что оно не зависит от выбора представителя смежного класса.
Давайте возьмем другой элемент из смежного класса, он имеет вид $gh$, где $h\in N$.
Тогда $\phi(gh) = \phi(g) \phi(h) = \phi(g)$, так как $N$ лежит в ядре $\phi$.
Значит действительно образ не зависит от выбора представителя.

Теперь проверим, что это отображение является гомоморфизмом групп.
Для этого нам надо выбрать два произвольных смежных класса в $G/N$, скажем $g_1 N$ и $g_2N$, и проверить, что
\[
\gamma(g_1  N g_2  N) \stackrel{?}{=} \gamma(g_1  N) \gamma (g_2  N)
\]
Давайте посчитаем левую часть
\[
\gamma(g_1  N g_2  N) = \gamma(g_1 g_2 N) = \phi(g_1g_2)
\]
С другой стороны правая часть будет
\[
\gamma(g_1N) \gamma(g_2 N) = \phi(g_1) \phi(g_2) = \phi(g_1 g_2)
\]
Получили одно и то же, значит $\gamma$ действительно гомоморфизм.

Теперь надо проверить, что ядро этого гомоморфизма будет $\ker \phi / N$.
Это значит, что ядро должно состоять из смежных классов $gN$ таких, что $g\in \ker \phi$.
Давайте проверим это, пусть $gN$ лежит в $\ker \gamma$.
Тогда $\gamma(gN) = e$, то есть $\phi(g) = e$.
Значит $g\in \ker \phi$, что требовалось.

Осталось показать, что такое $\gamma$ единственное.
Но это простое следствие сюръективности $\pi$.
Действительно, если мы возьмем $g\in G$ и хотим обеспечить равенство $\phi(g) = \gamma(\pi(g))$, то это значит, что элемент $\pi(g)$ должен перейти в элемент $\phi(g)$.
Но $\pi(g)= gN$ по определению.
Значит мы обязаны положить образом $gN$ элемент $\phi(g)$.
\end{proof}

\subsection{Свободные абелевы группы и их факторы}

Оказывается к абелевым группам можно попытаться применить понятия из линейной алгебры такие как базис.
Однако выясняется, что не всякая абелева группа обладает базисом.
Тем не менее, те группы, что обладают базисом устроены очень просто и их можно изучать напрямую.
А все остальные группы -- это их факторы.
В этом смысле мы можем подобраться к изучению произвольных абелевых групп, причем этот подход очень даже конструктивный.
Напомню, что для абелевых групп я буду использовать аддитивную нотацию, то есть операция будет обозначаться в виде сложения $+$, а возведение в степень является умножением на целое число.

\begin{definition}
Пусть $G$ -- абелева группа и $g_1,\ldots, g_n\in G$ -- некоторые элементы.
\begin{enumerate}
\item Будем говорить, что элементы $g_1,\ldots, g_n$ линейно независимы, если для любого набора целых чисел $k_1,\ldots,k_n\in\mathbb Z$ из условия $k_1 g_1+ \ldots + k_n g_n = 0$ в $G$ следует, что все коэффициенты $k_i$ равны нулю в $\mathbb Z$.

\item Будем говорить, что элементы $g_1,\ldots, g_n$ являются порождающими $G$, если для любого элемента $g\in G$ найдется набор целых чисел $k_1,\ldots, k_n\in \mathbb Z$ такой, что $g = k_1 g_1 +\ldots + k_n g_n$.

\item Будем говорить, что набор $g_1,\ldots, g_n$ является базисом $G$, если этот набор одновременно линейно независим и порождающий.
\end{enumerate}
\end{definition}


\begin{definition}
Пусть $G$ -- абелева группа и $g_1,\ldots, g_n$ -- некоторые ее элементы.
Введем следующее обозначение
\[
\langle g_1,\ldots, g_n \rangle = \{ k_1 g_1 + \ldots + k_n g_n \mid k_i \in \mathbb Z \}
\]
То есть мы определили множество всех линейных комбинаций элементов $g_1,\ldots, g_n$ с целочисленными коэффициентами.
Это множество является подгруппой в $G$ и называется подгруппой порожденной $g_1,\ldots, g_n$.
Это наименьшая подгруппа содержащая все элементы $g_i$.
Кроме того, $g_1,\ldots, g_n$ являются порождающими для этой подгруппы.
\end{definition}

\paragraph{Примеры}

\begin{enumerate}
\item Пусть $G = \mathbb Z^n$ и $g_i = (0,\ldots, 0,1,0,\ldots,0)$, где $1$ стоит на $i$-ом месте.
Тогда набор элементов $g_1,\ldots, g_n$ является базисом.

\item Пусть $G = \mathbb Z/n\mathbb Z$, где $n \neq 0$.
Тогда любой элемент $G$ линейно зависим.
Действительно, для любого $g\in G$ верно, что $n g = 0$, при этом $n \neq 0$.
Элемент $1 + n\mathbb Z$ является порождающим.
Таким образом в абелевых группах в отличие от векторных пространств бывает так, что в группе нет линейно независимых порождающих.

\item Пусть $G = (\mathbb Q, + )$.
Тогда любой ненулевой элемент $G$ будет линейно независим.
Действительно, если $0\neq q \in \mathbb Q$ и $0 \neq n \in \mathbb Z$, то $n q \neq 0 $  в $\mathbb Q$.
Однако, любые два элемента из $G$ уже линейно зависимы.
Действительно, пусть $q_1 = \frac{a}{b}$ и $q_2 = \frac{c}{d}$.
Тогда $cb q_1 - ad q_2 = 0$ и при этом набор $(cd, -ad)$ не нулевой.
То есть максимальное линейно независимое множество содержит один элемент.
И этот элемент не может быть порождающим.
Потому что если вы рассмотрите $\langle q\rangle = \{n \frac{a}{b} \mid n\in \mathbb Z \text{ и }q = \frac{a}{b}\}$, то в знаменателях этих элементов вы получите только те простые числа, которые входят в знаменатель $q$.
То есть это еще один пример абелевой группы без базиса, но содержащей линейно независимые множества.
\end{enumerate}

\begin{claim}
Пусть $G$ -- абелева группа и $g_1,\ldots, g_n$ -- базис $G$.
Тогда $G$ изоморфна $\mathbb Z^{n}$.
\end{claim}
\begin{proof}
Давайте рассмотрим отображение $\phi \colon \mathbb Z^n \to G$ по правилу $(a_1,\ldots,a_n) \mapsto a_1 g_1+ \ldots + a_n g_n$.
Так как $g_1,\ldots, g_n$ порождающие, то отображение $\phi$ по определению сюръективно.
Теперь покажем инъективность.
Пусть так получилось, что для двух наборов $(a_1,\ldots,a_n)$ и $(b_1,\ldots,b_n)$ их образы совпали, то есть
\[
a_1 g_1 + \ldots + a_n g_n = b_1 g_1 + \ldots + b_n g_n
\]
Перенесем все в одну сторону, получим
\[
(a_1 - b_1) g_1 + \ldots + (a_n - b_n) g_n = 0
\]
Так как $g_1,\ldots, g_n$ линейно независимы, то все коэффициенты должны быть нулевыми.
А это значит, что наборы $(a_1,\ldots,a_n)$ и $(b_1,\ldots,b_n)$ одинаковые.
Теперь надо показать, что $\phi$ уважает операцию сложения.
Для этого надо показать тождество
\[
\phi
\left(
\begin{pmatrix}
a_1\\\vdots\\a_n\\
\end{pmatrix}
+
\begin{pmatrix}
b_1\\\vdots\\b_n\\
\end{pmatrix}
\right)
=
\phi
\begin{pmatrix}
a_1\\\vdots\\a_n\\
\end{pmatrix}
+
\phi
\begin{pmatrix}
b_1\\\vdots\\b_n\\
\end{pmatrix}
\]
Но это делается прямым вычислением.
Левая и правая части по определению вычисляются в выражение $(a_1 + b_1) g_1 + \ldots + (a_n + b_n) g_n$.
\end{proof}

\paragraph{Замечание}

Пусть $G$ -- абелева группа с конечным числом порождающих $g_1,\ldots, g_n$.
Тогда отображение $\pi \colon \mathbb Z^n \to G$ по правилу $(a_1,\ldots,a_n) \mapsto a_1 g_1 + \ldots + a_n g_n$ является сюръективным.
Тогда по первой теореме о гомоморфизме $G$ изоморфно $\mathbb Z^n / \ker \pi$.
При этом $\ker \pi$ -- это некоторая подгруппа в $\mathbb Z^n$.
Таким образом мы можем описать любую абелеву группу с конечным числом порождающих, как фактор свободной группы $\mathbb Z^n$.

\subsection{Конечные абелевы группы}

Ранее мы с вами использовали факт (см.~утверждение~\ref{claim::FAGStruct}), что любая конечная абелева группа $G$ изоморфна прямому произведению циклических групп.
Здесь я хочу объяснить, как доказывается этот факт с помощью свободных групп и их факторов.

Пусть $G$ -- конечная абелева группа.
Тогда в ней конечное число элементов, а значит конечное число порождающих (можно в качестве порождающих взять все элементы группы $G$, этого заведомо хватит).
Но тогда, в силу замечания выше, мы можем представить группу $G$ в виде $\mathbb Z^n / N$, где $N\subseteq \mathbb Z^n$ -- некоторая подгруппа.
Так как $N$ так же конечная, то в ней тоже найдется конечное число порождающих, пусть это будут $h_1,\ldots,h_m\in N$.
Тогда $N = \langle h_1,\ldots, h_m\rangle$.
То есть мы представили 
\[
G = \mathbb Z^n / \langle h_1,\ldots, h_m \rangle
\]

Давайте представим на секунду, что нам так повезло, что у группы $N$ нашлись следующие порождающие
\[
h_1 = (d_1,0,\ldots, 0), h_2 = (0, d_2,0,\ldots, 0),\ldots, h_n = (0,\ldots,0,d_n)
\]
Давайте покажем, что в этом случае $G$ изоморфна $\mathbb Z/d_1 \mathbb Z \times \ldots \times \mathbb Z/d_n \mathbb Z$.
Действительно, рассмотрим гомоморфизм 
\[
\phi \colon \mathbb Z^n \to \mathbb Z/d_1 \mathbb Z \times \ldots \times \mathbb Z/d_n \mathbb Z,\quad (a_1,\ldots,a_n) \mapsto (a_1 \pmod{d_1},\ldots, a_n \pmod{d_n})
\]
Ясно, что это сюръективный гомоморфизм.
Но что будет его ядром?
Его ядро будет состоять из последовательностей $(k_1 d_1, \ldots, k_n d_n)$ для произвольных $k_i \in \mathbb Z$.
То есть это в точности подгруппа порожденная $h_1,\ldots,h_n$.
Значит по теореме о гомоморфизме в этом случае $G = \mathbb Z^n / \ker \phi$ изоморфна прямому произведению циклических групп.

Теперь давайте посмотрим на общую ситуацию, когда $h_i$ выглядят произвольным образом и наша группа имеет вид
\[
G = \mathbb Z^n / \langle h_1,\ldots, h_m \rangle
\]
Что можно сделать в этом случае?
Обратите внимание на две вещи: мы можем по разному выбрать порождающие в группе $\langle h_1,\ldots, h_m\rangle$ и мы можем по разному выбрать базис в группе $\mathbb Z^n$.
Если мы выбираем порождающие или базис по другому, то мы не меняем самих групп, а значит мы не изменим фактор, но при этом получим более удобное представление для фактора.

Давайте поймем, какие манипуляции с базисом и порождающими нам доступны.
Сначала порождающие:
\begin{enumerate}
\item Заменим один из образующих $h_i$ на $h_i + m h_j$, где $m\in \mathbb Z$ и $i\neq j$, а остальные образующие оставим те же самые.

\item Поменяем местами два образующих.

\item Умножим один из образующих на $-1$.
\end{enumerate}
Можно увидеть методом пристального взгляда, что это не меняет порождаемую подгруппу.
А теперь базис
\begin{enumerate}
\item Заменим один из базисных векторов $e_i$ на $e_i + m e_j$, где $m\in \mathbb Z$ и $i\neq j$, а остальные базисные векторы оставим те же самые.

\item Поменяем местами два базисных вектора.

\item Умножим один из базисных векторов на $-1$.
\end{enumerate}
Опять же, призывая на помощь метод пристального взгляда, можно увидеть, что после этих преобразований базис остается базисом.
Теперь посмотрим, как это все выглядит в координатах.
Если мы сложим все порождающие $h_1,\ldots, h_m$ по строкам в целочисленную матрицу, то преобразования над образующими дают нам целочисленные элементарные преобразования строк этой матрицы, а преобразования базиса дают нам целочисленные элементарные преобразования столбцов этой матрицы.
Но можно проверить, что любую целочисленную матрицу можно диагонализовать с помощью таких преобразований.
Это значит что общий случай сводится к разобранному частному случаю.
Что завершает доказательство.