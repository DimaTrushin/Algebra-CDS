\ProvidesFile{lecture12.tex}[Lecture 12]


\section{Факторы}

В этом разделе я хочу поговорить о некоторых важных абстракциях, которые используются в алгебре и которые я успешно обходил все это время.
Начать я хочу с конструкции фактора.
Эта конструкция очень простая, но требует некоторых психологических усилий по привыканию к ней и потому основная сложность в ее освоении будет не техническая, а идейно-психологическая.

\subsection{Фактор множества}

Обычно изложение начинают с фактор групп.
Однако, для лучшего понимания, я предлагаю начать с еще более простого объекта -- фактор множества.
Концепция в этом случае следующая.
Предположим у нас есть некоторое множество $X$ и мы хотим <<склеить>> какие-то его элементы между собой.
По сути, чтобы <<склеить>> элементы, надо ввести новое равенство на множестве $X$, которое бы перестало различать склеенные элементы.
В математике для такого нового равенства есть более приличное название -- отношение эквивалентности.
Напомню, что бинарное отношение $\sim$ на $X$ является отношением эквивалентности, если выполнены следующие три свойства
\begin{enumerate}
\item $x\sim x$ для любого $x\in X$.

\item $x\sim y$ влечет $y \sim x$ для любых $x,y\in X$.

\item Если $x \sim y$ и $y \sim z$, то $x \sim z$ для любых $x,y, z\in X$.
\end{enumerate}
В этом случае множество $X$ разбивается в непересекающееся объединение классов эквивалентности.
Множество классов эквивалентности обозначается $X/{\sim}$.
То есть можно считать так, что мы для каждого класса эквивалентности заводим точку, на которой пишем имя этого класса, а потом все эти точки сваливаем в одну кучу под названием $X/{\sim}$.

Если у нас есть множество $X$ с отношением эквивалентности $\sim$, то мы можем построить естественным образом отображение $\pi\colon X\to X/{\sim}$, где каждый элемент $x$ идет в его класс эквивалентности.%
\footnote{Существует много обозначений для класса эквивалентности $x$, например: $[x]$, $\bar x$ или $\operatorname{cl}(x)$ и т.д.}
Это отображение будет сюръективно, так как каждый класс состоит из точек $X$.
Более того, по этому отображению можно восстановить исходное отображение эквивалентности по следующему правилу.
Заметим, что $x\sim y$ тогда и только тогда, когда $\pi(x) = \pi(y)$.
Пусть теперь наоборот у нас есть какое-то сюръективное отображение $\phi\colon X\to Y$.
Тогда введем отношение эквивалентности $x\sim y$ тогда и только тогда, когда $\phi(x) = \phi(y)$.
Легко видеть, что мы действительно получили отношение эквивалентности.%
\footnote{Я все же рекомендую проверить три свойства тем, кто хочет разобраться во всех деталях.}
По определению отношения эквивалентности каждый класс эквивалентности однозначно соответствует точкам $Y$, а именно для $y\in Y$ множество $\phi^{-1}(y)$ будет классом эквивалентности.
Таким образом можно считать, что $Y$ -- это и есть множество классов эквивалентности, а отображение $\phi$ есть не что иное, как сопоставление каждому элементу из $X$ его класса эквивалентности.
Таким образом нет разницы между изучением сюръективных отображений из $X$ или отношений эквивалентности на $X$.

\paragraph{Примеры}

\begin{enumerate}
\item Пусть $\mathbb Z$ выбрано в качестве множества и отношение эквивалентности будет $x\sim y$ тогда и только тогда, когда $x$ и $y$ имеют одинаковую четность.
Тогда у нас только два класса эквивалентности: четные числа и нечетные числа.
Назовем эти классы <<$0$>> и <<$1$>>, для четных и нечетных соответственно.
Тогда $\mathbb Z/{\sim} = \{\text{<<$0$>>},\text{<<$1$>>}\}$.
Отображение $\pi \colon \mathbb Z\to \mathbb Z/{\sim}$ отображает четные числа в <<$0$>>, а нечетные в <<$1$>>.

\item Пусть теперь $\mathbb R$ выбрано в качестве множества.
Зададим отношение эквивалентности следующим образом: $x\sim y$ тогда и только тогда, когда $x - y \in \mathbb Z$.
Тогда классом эквивалентности точки $x$ является множество чисел на прямой отличающееся от $x$ на целое число, то есть $x + \mathbb Z$.
Ясно, что каждый класс эквивалентности имеет только одну точку пересечения с полуинтервалом $[0, 1)$.
Но такой полуинтервал можно отождествить с окружностью радиуса $1$ по правилу $x\mapsto (\cos(2\pi x), \sin(2\pi x))$.
Тогда множество $\mathbb R/{\sim}$ можно описать как полуинтервал $[0, 1)$.
Но можно получить еще более красивое описание в виде окружности радиуса $1$ на плоскости.
Тогда отображение $\pi \colon \mathbb R\to \mathbb R/{\sim}$ задается по правилу $x \mapsto (\cos(2\pi x), \sin(2\pi x))$.
\end{enumerate}

\subsection{Фактор группы}

Теперь давайте посмотрим, что получится, если мы постараемся склеить элементы группы $(G, \cdot)$.
Если мы будем склеивать просто так, не думаю про операцию, то можно считать, что у нас вообще нет никакой операции и мы просто работаем с множеством $G$.
Чтобы почувствовать влияние операции надо, чтобы отношение эквивалентности было согласовано с операцией.

\begin{definition}
Пусть $G$ -- группа и $\sim$ -- отношение эквивалентности на $G$.
Отношение $\sim$ называется конгруэнцией, если для любых $x,y,u,v\in G$ таких, что $x\sim y$ и $u\sim v$ следует, что $x \cdot u \sim y \cdot v$.
\end{definition}

\paragraph{Операция на классах эквивалентности}

Если у нас есть конгруэнция на группе $G$, то мы можем естественным образом снабдить множество классов эквивалентности $G/{\sim}$ операцией, которая превращает это множество в группу, а отображение $\pi \colon G\to G/{\sim}$ в гомоморфизм групп.
Действительно, пусть у нас есть два класса эквивалентности $\alpha,\beta\in G/{\sim}$, тогда выберем из каждого из них по точке, скажем $x\in \alpha$ и $y\in \beta$, где $x,y\in G$.
Теперь перемножим $xy\in G$.
Это произведение попадает в какой-то класс $\gamma$.
Тогда надо положить $\alpha\cdot \beta=\gamma$.
Единственная проблема этого определения -- надо проверить, что оно не зависит от выбора элементов $x$ и  $y$.
Давайте предположим, что мы выбрали другие элементы $x'\in \alpha$ и $y'\in \beta$.
Но в этом случае $x\sim x'$ и $y\sim y'$ по определению классов эквивалентности.
Тогда в силу того, что $\sim$ конгруэнция, $x\cdot y \sim x' \cdot y'$.
А значит, элементы $xy$ и $x'y'$ лежат в одном и том же классе эквивалентности $\gamma$.
А это и означает, что определение не зависит от выбора точек в классах эквивалентности.
По-другому про это можно думать так: давайте перемножим все точки $\alpha$ на все точки $\beta$, то есть рассмотрим множество $\alpha \cdot \beta$.
Тогда это множество целиком попадает внутрь $\gamma$, то есть $\alpha \cdot \beta \subseteq \gamma$.
Пусть $x,y\in G$ -- какие-то элементы.
Давайте обозначим их классы эквивалентности через $\bar x$ и $\bar y$ соответственно.
Тогда правило умножения определенное выше в этих обозначениях выглядит так: $\bar x  \bar y = \overline{xy}$.

\begin{claim}
Пусть $G$ -- некоторая группа и $\sim$ -- конгруэнция на $G$.
Тогда множество $G/{\sim}$ с введенной выше операцией является группой.
\end{claim}
\begin{proof}
% TO DO
Нам надо проверить три аксиомы группы.

1) \textbf{Ассоциативность.}
Нам надо выбрать произвольные три класса из $G/{\sim}$.
Для этого можно выбрать три элемента $x,y,z\in G$ и рассмотреть их классы $\bar x, \bar y, \bar z\in G/{\sim}$.
В этом случае нам надо проверить, что
\[
(\bar x \bar y) \bar z = \bar x (\bar y \bar z)
\]
Вычислим обе части пользуясь нашим определением.
Видим, что $(\bar x \bar y) \bar z = \overline{xy}\bar z = \overline{(xy)z}$.
Аналогично $\bar x (\bar y \bar z) = \bar x \overline{yz} = \overline{x(yz)}$.
Но так как элементы $(xy)z$ и $x(yz)$ равны в силу ассоциативности, то и их классы тоже равны.
Что доказывает ассоциативность умножения в $G/{\sim}$.

2) \textbf{Наличие нейтрального элемента.}
Пусть $e\in G$ -- нейтральный элемент.
Давайте покажем, что класс $\bar  e$ будет выполнять роль единичного элемента в $G/{\sim}$.
Действительно для любого $x\in G$ выполнено
\[
\bar x \bar e = \overline{xe} = \bar x\text{ и }\bar e \bar x = \overline{ex} = \bar x
\]

3) \textbf{Обратимость каждого элемента.}
Нам надо проверить, что для каждого элемента $x\in G$ его класс $\bar x$ обратим в $G/{\sim}$.
Для этого достаточно предъявить обратный.
Давайте проверим, что элемент $\overline{x^{-1}}$ будет обратным.
Действительно,
\[
\bar x \overline{x^{-1}} = \overline{x x^{-1}} = \bar e \text{ и }\overline{x^{-1}} \bar x  = \overline{x^{-1} x} = \bar e
\]
\end{proof}

Теперь переключимся на вот какой вопрос.
Не задавались ли вы вопросом: а почему мы задали произведение классов эквивалентности именно по такому правилу?
Оказывается, что ни по какому другому правилу его задать нельзя.
Это следует из следующего наблюдения.

\begin{claim}
Пусть $G$ -- группа, $X$ -- множество и $\pi \colon G\to X$ -- сюръективное отображение.
Тогда существует не более одной операции на $X$ такой, что $X$ превращается в группу, а $\pi$ в гомоморфизм групп.
\end{claim}
\begin{proof}
Если таких операций нет, то победа.
Пусть есть две такие операции, скажем $\cdot$ и $\circ$.
Покажем, что они обязаны совпадать.
Рассмотрим две точки $x, y\in X$.
Так как $\pi$ сюръективно, то у них существуют прообразы в $G$, то есть найдутся такие $g,h\in G$, что $\pi(g) x$ и $\pi(h) = y$.
Тогда
\[
x\cdot y = \pi(g) \cdot \pi(h) = \pi(g h)
\]
где $gh$ -- произведение в $G$.
С другой стороны
\[
x \circ y = \pi(g) \circ \pi(h) = \pi(gh)
\]
То есть обе операции вычисляются по одному и тому же правилу.
Значит они совпадают.
\end{proof}

\paragraph{Замечание}

Давайте я обсужу причину, почему операция в предыдущем утверждении может не существовать.
В обозначениях доказательства для перемножения элементов $x,y\in X$ мы должны положить $x\cdot y = \pi(gh)$, где $\pi(g) = x$ и $\pi(h) = y$.
Но предположим, что у нас есть несколько прообразов у $x$ и несколько прообразов у $y$.
Например есть $g',h'\in G$ такие, что $\pi(g') = x$ и $\pi(h') = y$.
Тогда у нас должно быть $x\cdot y = \pi(g'h')$.
Но отображение $\pi$ может быть задано так, что $\pi(gh) \neq \pi(g'h')$.
Именно в этом случае мы получим противоречие с существованием операции $\cdot$.
На самом деле, эта причина является единственной для несуществования операции на $X$.

Чтобы эффект выше был более понятным, я продемонстрирую его на примере.
Пусть у нас $G = (\mathbb Z, +)$.
Отношение эквивалентности на $G$ зададим так, что $0 \sim 1$, а все остальные элементы эквивалентны только самим себе.
В качестве отображения возьмем $\pi \colon \mathbb Z\to \mathbb Z/{\sim}$ -- отображение в классы эквивалентности.
Пусть класс $\{0, 1\}$ называется A, класс $\{2\}$ называется B, а класс $\{3\}$ называется C.
Тогда если мы хотим перемножить A и B, то с одной стороны
\[
\text{A}\cdot \text{B} = \pi(0) \cdot \pi(2) = \pi(0+2) = \pi(2) = \text{B}
\]
А с другой
\[
\text{A}\cdot \text{B} = \pi(1) \cdot \pi(2) = \pi(1+2) = \pi(3) = \text{C}
\]
Что приводит к противоречию, произведение не может равняться одновременно двум разным элементам.


\subsection{Конгруэнции и нормальные подгруппы}

Оказывается, что в случае группы существует биекция между конгруэнциями и нормальными подгруппами.
То есть любую конгруэнцию можно описать с помощью нормальной подгруппы.
Давайте я объясню, как это соответствие работает.

Пусть $G$ -- некоторая группа и $\sim$ -- конгруэнция на $G$.
Тогда построим подмножество $N_\sim = \{g\in G \mid g \sim e\}$.
Я утверждаю, что получится обязательно нормальная подгруппа в $G$ (я пока отложу доказательство).
Наоборот, пусть $N\subseteq G$ -- нормальная подгруппа, тогда определим отношение $\stackrel{N}{\sim}$ следующим образом $x \stackrel{N}{\sim}y$ тогда и только тогда, когда $xy^{-1}\in N$.
Я утверждаю, что получится конгруэнтность на $G$ (пока откладываем доказательство).
Более того, эти операции взаимно обратные, то есть они дают биекцию между множеством всех конгруэнций на $G$ и множеством всех нормальных подгрупп в $G$.

\paragraph{Проверка}

1) В начале проверим, что подмножество $N_\sim$ действительно дает нормальную подгруппу.
Для этого надо проверить три аксиомы группы и условие нормальности.
Идем по-порядку
\begin{enumerate}
\item По условию $N_\sim = \{g\in G\mid g\sim e\}$.
Надо проверить, что $e$ лежит в $N_\sim$.
Но по определению $e \sim e$, значит лежит.

\item Пусть $x\in N_\sim$, то есть $x\sim e$.
Надо проверить, что $x^{-1}\sim e$.
По определению выполнено $x^{-1}\sim x^{-1}$.
Тогда по определению конгруэнции $x x^{-1} \sim e x^{-1}$.
То есть $e \sim x^{-1}$, а значит и $x^{-1}\sim e$, что и требовалось.

\item Пусть теперь $x,y\in N_\sim$ и надо показать, что $xy \in N_\sim$.
То есть $x\sim e$ и $y\sim e$.
Но тогда $x y \sim e e = e$, что и требовалось.

\item Теперь надо взять любой $x\in N_\sim$ и $y\in G$ и проверить, что $yxy^{-1}\in N_\sim$.
Так как $x\sin e$ и $\sim$ конгруэнция получаем, что $yxy^{-1}\sim y e y^{-1} = e$, что и требовалось.
\end{enumerate}

2) Теперь проверим, что $\stackrel{N}{\sim}$ является конгруэнцией.
Для этого надо проверить три аксиомы отношения эквивалентности и еще одну аксиому согласованности с операцией.
Пойдем по-порядку
\begin{enumerate}
\item Надо проверить, что $x \stackrel{N}{\sim}x$.
Действительно $xx^{-1} = e \in N$.

\item Надо проверить, что $x\stackrel{N}{\sim}y$ влечет $y \stackrel{N}{\sim}x$.
Нам дано, что $xy^{-1}\in N$.
А надо показать, что $yx^{-1}\in N$.
В этом случае обратный элемент тоже лежит в $N$, то есть
$(xy^{-1})^{-1} = yx^{-1}\in N$, что и требовалось.

\item Теперь пусть $x\stackrel{N}{\sim}y$ и $y\stackrel{N}{\sim}z$.
Надо показать, что $x\stackrel{N}{\sim}z$.
Нам дано $xy^{-1}\in N$ и $yz^{-1}\in N$, тогда и произведение лежит в $N$.
То есть $xz^{-1} =  xy^{-1} yz^{-1}\in N$, что и требовалось.

\item До этого мы не пользовались нормальностью $N$, нам было достаточно, что $N$ является подгруппой.
Для доказательства, что $\stackrel{N}{\sim}$ согласована с операцией в группе, нам понадобится нормальность.
Пусть $x\stackrel{N}{\sim}y$ и $u\stackrel{N}{\sim}v$, нам надо показать, что $x u \stackrel{N}{\sim} y v$.
Нам дано, что $xy^{-1}\in N$ и $uv^{-1}\in N$, надо показать, что $xu(yv)^{-1}\in N$.
Рассмотрим требуемое выражение
\[
xu(yv)^{-1} = xu v^{-1} y^{-1} = x(y^{-1}y)u v^{-1} y^{-1} = (xy^{-1})y(u v^{-1}) y^{-1}
\]
Тогда $xy^{-1}$ лежит в $N$ по условию, $uv^{-1}$ лежит в $N$.
Так как $y\in G$ и $N$ нормальна, то $y(uv^{-1})y^{-1}$ тоже лежит в $N$.
А значит и все выражение лежит в $N$.
\end{enumerate}

3) Теперь проверим, что эти операции взаимно обратные.
Для этого надо проверить следующие равенства
\[
\sim \quad\mapsto\quad N_\sim\quad \mapsto\quad \stackrel{N_\sim}{\sim} = \sim\
\quad\quad
\text{и}
\quad\quad
N \quad\mapsto \quad\stackrel{N}{\sim} \quad\mapsto\quad N_{\stackrel{N}{\sim}} = N
\]
Проверим первое равенство.
Как проверить, что два отношения равны, надо показать, что два элемента эквивалентны в смысле $\stackrel{N_\sim}{\sim}$ тогда и только тогда, когда они эквиваленты в смысле $\sim$.
Пусть $x\stackrel{N_\sim}{\sim} y$.
Это равносильно тому, что $xy^{-1}\in  N_\sim$, что по определению означает, что $xy^{-1}\sim e$.
Но так как $\sim$ конгруэнтность, мы можем рассмотреть $y\sim y$ и умножить на предыдущее равенство.
Тогда $xyy^{-1}\sim e y$, то есть $x\sim y$.
Обратно, если $x\sim y$, то рассмотрим $y^{-1}\sim y^{-1}$ и перемножим.
Получим $xy^{-1}\sim y y^{-1} = e$.
Значит действительно $x\stackrel{N_\sim}{\sim} y$ равносильно $x\sim y$.
Теперь второе равенство.
По определению
\[
N_{\stackrel{N}{\sim}} = \{g\in G \mid g \stackrel{N}{\sim} e\} = \{g\in G \mid g e^{-1}\in N\} = N
\]

\paragraph{Описание фактора в терминах нормальной подгруппы}

Пусть теперь $G$ -- некоторая группа и $N\subseteq G$ -- нормальная подгруппа.
Возьмем отношение $\stackrel{N}{\sim}$ и покажем, что классы эквивалентности совпадают в точности со смежными классами группы $N$.
Фиксируем $x\in G$ и рассмотрим $\{y \in G\mid y \stackrel{N}{\sim} x\}$.
Тогда
\[
\{y \in G\mid y \stackrel{N}{\sim} x\} = \{y\in G \mid yx^{-1}\in N\} = \{y\in G\mid y \in N x\} = N x = x N
\]
То есть класс эквивалентности элемента $x$ совпадает с его смежным классом (не важно левым или правым, так как $N$ нормальная).

Множество смежных классов $G$ по $N$ обозначается через $G/N$.
Таким образом $G/N = G/{\stackrel{N}{\sim}}$.
Потому в случае групп фактор группу по нормальной подгруппе определяют с помощью нормальной подгруппы и обозначаются $G/N$.
Давайте я резюмирую, что произошло.
Мы определяем $G/N$ как множество смежных классов $N$.
На этом множестве определена операция $gN \cdot hN = gh N$.
Нейтральным элементом является $N$.
А обратный вычисляется по правилу $(gN)^{-1} = g^{-1}N$.
При этом отображение $\pi \colon G\to G/N$ по правилу $g \mapsto gN$ является сюръективным гомоморфизмом групп.

\paragraph{Пример}

В качестве группы возьмем $\mathbb Z$, а в качестве нормальной подгруппы $n\mathbb Z$.
Тогда множество смежных классов имеет вид
\[
\mathbb Z / n\mathbb Z = \{n\mathbb Z, 1 + n \mathbb Z, \ldots, n-1 + n\mathbb Z\}
\]
Таким образом мы можем отождествить множество смежных классов с остатками $\mathbb Z_n$ по модулю $n$.
Более того, можно проверить, что операция на остатках совпадает с операцией на смежных классах, то есть $\mathbb Z/ n \mathbb Z = \mathbb Z_n$ -- изоморфизм групп.