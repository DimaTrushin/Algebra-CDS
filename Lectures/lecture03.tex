\ProvidesFile{lecture03.tex}[Lecture 3]


\subsection{Гомоморфизмы и изоморфизмы}

Существует много различных групп.
Кроме того, мы с вами изучим методы по построению новых групп из уже имеющихся.
В такой ситуации очень полезно иметь механизм для сравнения групп.
Как понять, что мы построили уже знакомую нам группу?
Чтобы ответить на этот вопрос, мы должны объяснить, что значит, что две группы одинаковые.
То есть нам нужен способ сравнивать группы между собой.
Тут на помощь нам приходят гомоморфизмы (способ сравнивать группы) и изоморфизмы (способ говорить, что две группы одинаковые).
Давайте начнем с определений.

\begin{definition}
Пусть $G$ и $H$ -- группы.
Определим гомоморфизм $\varphi\colon G\to H$.
\begin{itemize}
\item \textbf{Данные} отображение $\varphi\colon G\to H$.

\item \textbf{Аксиома} $\varphi(g_1g_2) = \varphi(g_1) \varphi(g_2)$ для всех $g_1,g_2\in G$.
\end{itemize}
В таком случае $\varphi$ называется гомоморфизмом из $G$ в $H$.
\end{definition}

\begin{remark}
Давайте я явно проговорю определение.
Нам даны две группы $(G, \circ)$ и $(H,\cdot)$.
Гомоморфизм $\varphi \colon G\to H$ -- это отображение такое, что $\varphi(g_1 \circ g_2) = \varphi(g_1) \cdot \varphi(g_2)$.
В левой части равенства мы берем элементы $g_1$ и $g_2$ из группы $G$ и перемножаем их с помощью операции из $G$ и потом отправляем результат в $H$.
В правой части, мы сначала отправляем элементы $g_1$ и $g_2$ в группу $H$ и только потом перемножаем образы с помощью операции из $H$.
\end{remark}

\begin{examples}
\begin{enumerate}
\item Пусть $G = (\mathbb Z, +)$ и $H = (\mathbb Z_n, +)$, тогда отображение $\pi\colon \mathbb Z\to \mathbb Z_n$ по правилу $k \mapsto k \pmod n$ является гомоморфизмом.

\item Пусть $G = S_n$ -- группа перестановок и $H = \mu_2 = \{\pm 1\}$ снабжено умножением.
Тогда отображение $\sgn \colon S_n \to \mu_2$ сопоставляющее каждой перестановке ее знак (четные идут в $1$, а нечетные в $-1$) является гомоморфизмом.

\item Пусть $G = (\operatorname{GL}_n(\mathbb R),\cdot)$ и $H = (\mathbb R^*, \cdot)$ -- множество ненулевых вещественных чисел с операцией умножения.
Тогда отображение $\det \colon \operatorname{GL}_n(\mathbb R) \to \mathbb R^*$ по правилу $A \mapsto \det(A)$ является гомоморфизмом.

\item Пусть $G = (\mathbb R, +)$ и $H = (\mathbb R^*, \cdot)$.
Тогда отображение $\exp\colon \mathbb R\to \mathbb R^*$ по правилу $x \mapsto e^x$ является гомоморфизмом.

\item Пусть $G = (\mathbb Z, +)$, $H$ -- произвольная группа и $h\in H$ -- произвольный элемент.
Тогда отображение $\phi\colon \mathbb Z \to H$ по правилу $k \mapsto h^k$ является гомоморфизмом.

\item Пусть $G = (\mathbb Z_n, +)$, $H$ -- произвольная группа и $h\in H$ -- произвольный элемент такой, что $h^n = 1$.
Тогда отображение $\phi\colon \mathbb Z_n \to H$ по правилу $k \mapsto h^k$ является гомоморфизмом групп.
\end{enumerate}
\end{examples}

Давайте докажем некоторые свойства гомоморфизмов.

\begin{claim}
\label{claim::HomGrProp}
Пусть $\varphi\colon G\to H$ -- некоторый гомоморфизм групп.
Тогда
\begin{enumerate}
\item $\varphi(1) = 1$, то есть нейтральный элемент $G$ идет в нейтральный элемент $H$.

\item $\varphi(g^{-1}) = \varphi(g)^{-1}$ для любого $g\in G$.
\end{enumerate}
\end{claim}
\begin{proof}
1) Мы знаем, что $1 = 1 \cdot 1$.
Давайте применим $\varphi$ к этому равенству.
Тогда мы получим
\[
\varphi(1) = \varphi(1 \cdot 1) = \varphi(1) \varphi(1) \in H
\]
Теперь умножим это равенство на $\varphi(1)^{-1}$, будет $1 = \varphi(1)$.


2) Пусть $g\in G$ -- некоторый элемент.
Тогда $g g^{-1} = 1$.
Теперь применим $\varphi$ к этому равенству и получим
\[
\varphi(g) \varphi(g^{-1}) = \varphi(g g^{-1}) = \varphi(1) = 1
\]
Умножая полученное равенство слева на $\varphi(g)^{-1}$, мы получаем $\varphi(g^{-1})  = \varphi(g)^{-1}$.
\end{proof}

\begin{definition}
\label{def::IsomorphismGr}
Пусть $G$ и $H$ -- группы.
Определим изоморфизм $\varphi \colon G\to H$.
\begin{itemize}
\item \textbf{Данные} гомоморфизм $\varphi \colon G\to H$.

\item \textbf{Аксиома} $\varphi$ является биекцией.
\end{itemize}
В этом случае $\varphi$ называется изоморфизмом между $G$ и $H$.
Если найдется изоморфизм между $G$ и $H$, то группы $G$ и $H$ называются изоморфными.
\end{definition}

Давайте я поясню немного определение.
В начале давайте разберемся, что значит, что у нас есть биекция множеств $\varphi \colon X\to Y$.
Предположим $X = \{1, 2, 3\}$, $Y = \{a, b, c\}$ и $\varphi$ устроено так: $1 \mapsto a$, $2\mapsto b$ и $3\mapsto c$.
Тогда можно думать про эту биекцию следующим образом.
Множество $X$ -- это множество имен элементов, а множество $Y$ -- это множество других имен тех же самых элементов.
Тогда на биекцию можно смотреть, как на процедуру переименования.
То есть можно считать, что $Y$ это то же самое множество элементов, что и $X$, только с другими названиями элементов.

Теперь пусть у нас задан изоморфизм групп $\varphi\colon G\to H$.
Такой гомоморфизм как минимум биекция, а значит мы можем считать, что подлежание множества $G$ и $H$ на самом деле одинаковые.
Кроме того, условие $\varphi(g_1g_2) = \varphi(g_1) \varphi(g_2)$ означает, что при нашем отождествлении $G$ и $H$ операция на $G$ превращается в операцию на $H$.
То есть мы можем думать, что $H$ -- это то же самое множество, что и $G$ с той же самой операцией, что на $G$.
Или другими словами, мы считаем, что группа $H$ -- это та же самая группа, что и $G$, но только с другим множеством имен и другим обозначением операции.
Однако, по существу это одна и та же группа.
Как следствие, изоморфные группы имеют одинаковые свойства.

\begin{examples}
\begin{enumerate}
\item Пусть $G = (\mathbb Z_n, +)$ и $H = \mu_n\subseteq \mathbb C$ -- множество комплексных корней из единицы степени $n$ с операцией умножения.
Давайте фиксируем примитивный корень $\xi \in\mu_n$.
Тогда отображение $\mathbb Z_n \to \mu_n$ по правилу $k \mapsto \xi^k$ является изоморфизмом.

\item Пусть $G = (\mathbb Z, +)$ и
\[
H = \left\{\left.
\begin{pmatrix}
{1}&{n}\\
{0}&{1}
\end{pmatrix}
\;\right|\;
n\in \mathbb Z
\right\}
\]
с операцией умножения.
Тогда отображение $\varphi\colon \mathbb Z\to H$ по правилу $k \mapsto \left(\begin{smallmatrix}{1}&{k}\\{0}&{1}\end{smallmatrix}\right)$ является изоморфизмом.

\item Пусть $G = (\mathbb C, +)$ и $H = (\mathbb R^2, +)$.
Тогда отображение $\varphi \colon \mathbb C\to \mathbb R^2$ по правилу $z \mapsto (\Re z, \Im z)$ является изоморфизмом.

\item Пусть $G = (\mathbb C^*, \cdot)$ и
\[
H = \left\{\left.
\begin{pmatrix}
{a}&{-b}\\
{b}&{a}
\end{pmatrix}
\;\right|\;
a, b\in \mathbb R\text{ такие что }a^2 + b^2 \neq 0
\right\}
\]
с операцией умножения.
Тогда отображение $\varphi \colon \mathbb C^*\to H$ по правилу $a + bi \mapsto \left(\begin{smallmatrix}{a}&{-b}\\{b}&{a}\end{smallmatrix}\right)$ является изоморфизмом.

\item Claim~\ref{claim::CyclicClass} говорит, что циклическая группа $G = \langle g \rangle$ изоморфна $\mathbb Z$ или $\mathbb Z_n$ в зависимости от порядка образующего.
Если $\ord g = \infty$, тогда $G\simeq \mathbb Z$.
Если $\ord g = n$, тогда $G\simeq \mathbb Z_n$.
\end{enumerate}
\end{examples}

С каждым гомоморфизмом мы можем ассоциировать специальные подгруппы: ядро и образ.

\begin{definition}
Пусть $\varphi\colon G\to H$ -- некоторый гомоморфизм групп.
Тогда
\begin{enumerate}
\item ядро $\varphi$ -- это $\ker \varphi = \{g\in G \mid \varphi(g) = 1\} \subseteq G$.

\item образ$\varphi$ -- это $\Im \varphi = \{\varphi(g)\mid g\in G\} = \varphi(G) \subseteq H$.
\end{enumerate}
\end{definition}

Стоит отметить, что ядро является подмножеством в $G$, а образ -- в $H$.

\begin{claim}
\label{claim::HomProp}
Пусть $\varphi\colon G\to H$ -- гомоморфизм групп.
Тогда
\begin{enumerate}
\item $\Im \varphi \subseteq H$  является подгруппой.

\item $\ker \varphi \subseteq G$ является нормальной подгруппой.

\item Отображение $\varphi$ сюръективно тогда и только тогда, когда $\Im \varphi = H$.

\item Отображение $\varphi$ инъективно тогда и только тогда, когда $\ker \varphi  = \{1\}$.
\end{enumerate}
\end{claim}
\begin{proof}
1) Давайте проверим, что все свойства подгруппы выполняются.
Во-первых, $1 = \varphi(1) \in \Im \varphi$, значит нейтральный элемент лежит в образе.
Во-вторых, $\varphi(g_1)\varphi(g_2) = \varphi(g_1 g_2) \in \Im\varphi$, то есть образ замкнут относительно операции.
В-третьих, $\varphi(g)^{-1} = \varphi(g^{-1}) \in \Im\varphi$, то есть с каждым элементом образ содержит его обратный.

2) В начале проверим, что ядро -- подгруппа.
Во-первых, $\varphi(1) = 1$, значит $1 \in \ker \varphi$ по определению.
Во-вторых, если $x, y\in \ker \varphi$, тогда $\varphi(xy) = \varphi(x) \varphi(y) = 1\cdot 1 = 1$, то есть $xy\in \ker\varphi$.
В-третьих, если $x\in \ker\varphi$, то $\varphi(x^{-1}) = \varphi(x)^{-1} = 1^{-1} = 1$.
Значит $x^{-1}\in \ker\varphi$.
Мы только что проверили, что $\ker \varphi$ является подгруппой.
Теперь надо показать, что $g \ker \varphi = \ker \varphi g$ для всех $g\in G$.
По утверждению~\ref{claim::normal_crit}, достаточно проверить, что $g\ker \varphi g^{-1}\subseteq \ker \varphi$ для каждого $g\in G$.
То есть мы должны показать, что $\varphi(g \ker \varphi g^{-1}) = 1$ для каждого $g\in G$.
Действительно, пусть $h\in \ker \varphi$, тогда
\[
\varphi(g h g^{-1}) = \varphi(g) \varphi(h) \varphi(g^{-1}) = \varphi(g) \cdot 1 \cdot \varphi(g^{-1}) = \varphi(g) \varphi(g^{-1}) = \varphi(g g^{-1}) = \varphi(1) = 1
\]

3) Это условие тривиально по определению.

4) Предположим, что $\varphi$ инъективно и $x\in \ker\varphi$.
Это значит, что $\varphi(x) = 1$.
С другой стороны, мы всегда имеем $\varphi(1) = 1$.
Значит, $x$ и $1$ идут в один и тот же элемент $1$.
По инъективности получаем $x = 1$.

Теперь предположим, что $\ker \varphi = \{1\}$.
Рассмотрим два элемента $x, y\in G$ таких, что $\varphi(x) = \varphi(y)$.
Умножим это равенство на $\varphi(x)^{-1}$ и получим
\[
1 = \varphi(y) \varphi(x)^{-1} = \varphi(y) \varphi(x^{-1}) = \varphi(yx^{-1})
\]
Значит $yx^{-1}\in\ker\varphi = \{1\}$.
Поэтому $y x^{-1} = 1$.
Тогда $y = x$, что завершает доказательство.
\end{proof}

\subsection{Произведение групп}

Вообще говоря нам не очень хочется каждый раз строить группы с нуля.
Хочется иметь механизм по построению новых групп из уже заданных.
Существует много подобных процедур в алгебре.
Мы собираемся изучить одну из таких.

\begin{definition}
Пусть $G$ и $H$ -- некоторые группы.
Определим новую группу $G\times H$ следующим образом
\begin{enumerate}
\item Как множество это декартово произведение подлежащих множеств групп $G$ и $H$: $G\times H = \{(g, h)\mid g\in G,\;h\in H\}$.

\item Операция
\[
\cdot \colon (G\times H)\times (G\times H) \to G\times H
\]
задана по правилу
\[
(g_1, h_1) (g_2, h_2) = (g_1 g_2, h_1 h_2),\quad g_1, g_2,\in G, \;h_1, h_2\in H
\]
\end{enumerate}
Группа $G\times H$ называется произведением групп $G$ и $H$.
\end{definition}

По-хорошему нам надо бы показать, что $G\times H$ действительно является группой.
Мы только что определили все необходимые данные для группы, но осталось проверить аксиомы.
Давайте я напомню их
\begin{itemize}
\item Операция ассоциативна
\[
(g_1, h_1)\Bigl( (g_2, h_2) (g_3, h_3) \Bigl) = \Bigl((g_1, h_1) (g_2, h_2)\Bigl) (g_3, h_3)
\]

\item Существует нейтральный элемент, $1 = (1, 1)$.

\item У каждого элемента есть обратный,  $(g, h)^{-1} = (g^{-1}, h^{-1})$.
\end{itemize}
Все свойства проверяются прямым вычислением.
Если у нас есть несколько групп $G_1,\ldots,G_k$, мы можем определить произведение $G_1\times \ldots \times G_k$ аналогично тому, как мы определили произведение двух групп.

\subsection{Конечные абелевы группы}

Теперь я хочу сосредоточиться на очень важном классе групп, класс конечных абелевых групп.

\begin{definition}
Конечная абелева группа -- это коммутативная (абелева) группа $G$ с конечным числом элементов.
\end{definition}

Само по себе определение -- не большой сюрприз, название говорит само за себя.
Однако обратите внимание на следующий результат.

\begin{claim}
\label{claim::FAGStruct}
Пусть $G$ -- конечная абелева группа, тогда $G$ изоморфна группе вида $\mathbb Z_{n_1}\times \ldots \times \mathbb Z_{n_k}$.
\end{claim}

Я не буду доказывать этот результат.
Доказательство не сложно, но требует некоторой технической работы, на которую у нас нет времени.
Кроме того, само доказательство не проливает свет на какие-либо свойства конечных абелевых групп и потому не так интересно.
На мой взгляд куда важнее научиться понимать как использовать этот результат.
Давайте начнем с некоторых примеров.

\begin{examples}
\begin{enumerate}
\item Пусть $G = \mathbb Z_8^*$ с операцией умножения.
Очевидно, что это конечная абелева группа, а значит она должна представляться в виде произведения циклических групп.
Действительно, давайте проверим, что
\[
\mathbb Z_8^* \simeq \mathbb Z_2\times \mathbb Z_2
\]
Отображение задающее биекцию и уважающее операции можно задать так:
\[
1 \leftrightarrow (0,0),\; 3 \leftrightarrow (1,0),\; 5\leftrightarrow(0,1),\;7\leftrightarrow(1,1)
\]
Это не единственный способ отождествить эти две группы.
Например, вот другой изоморфизм:
\[
1 \leftrightarrow (0,0),\; 3 \leftrightarrow (1,0),\; 7\leftrightarrow(0,1),\;5\leftrightarrow(1,1)
\]
Я не собираюсь описывать все изоморфизмы, самое главное, что мы видим, что таких изоморфизмов много.
Так же обратите внимание, что группа не является циклической, так как в ней нет элемента порядка $4$.

\item Пусть $G = \mathbb Z_9^*$ с операцией умножения.
Это так же конечная абелева группа.
В этом случае мы имеем:
\[
\mathbb Z_9^* \simeq \mathbb Z_6
\]
вот пример двух разных изоморфизомв
\[
\begin{aligned}
\mathbb Z_6 &\to \mathbb Z_9^*\\
k &\mapsto 2^k
\end{aligned}
\quad\text{и}\quad
\begin{aligned}
\mathbb Z_6 &\to \mathbb Z_9^*\\
k &\mapsto 5^k
\end{aligned}
\]
Так же заметим, что в этом случае группа является циклической.
Элементы $2$ и $5$ являются различными образующими группы.
Изоморфизмы выше соответствуют одному из выбору образующего.
\end{enumerate}
\end{examples}


\begin{claim}
[Китайская теорема об остатках]
\label{claim::Chinese}
Пусть $m, n\in \mathbb N$ -- два взаимно простых натуральных числа, то есть $(m, n) = 1$.
Тогда отображение
\[
\Phi\colon \mathbb Z_{mn} \to \mathbb Z_m \times \mathbb Z_n,\quad
k \mapsto (k\!\!\mod m,\;k\!\!\mod n)
\]
является изоморфизмом групп.
\end{claim}
\begin{proof}
В начале мы должны показать, что отображение является гомоморфизмом.
Надо показать, что $\Phi(k + d) = \Phi(k) + \Phi(d)$, то есть
\begin{gather*}
\Phi(k + d) = ( (k+d)\!\!\mod m,\;(k+d)\!\!\mod n) = ((k\!\!\mod m) + (d\!\!\mod m),\;(k\!\!\mod n) + (d\!\!\mod n)) =\\
= (k\!\!\mod m ,\;k\!\!\mod n) + (d\!\!\mod m ,\;d\!\!\mod n) = \Phi(k) + \Phi(d)
\end{gather*}

Теперь я утверждаю, что гомоморфизм инъективен.
Утверждение~\ref{claim::HomProp} пункт~(4) гласит, что достаточно проверить, что ядро гомоморфизма содержит только нейтральный элемент.
По определению, имеем
\[
\ker \Phi = \{k\in \mathbb Z_{mn}\mid k = 0\pmod m,\; k = 0\mod n\}
\]
Значит $k\in \ker \Phi$ тогда и только тогда, когда $m$ делит $k$ и $n$ делит $k$.
Так как $m$ и $n$ взаимно просты, последнее означает, что  $mn$ делит $k$.
А значит, $k = 0$ в $\mathbb Z_{mn}$.

Чтобы показать, что $\Phi$ является изоморфизмом, нам надо показать сюръективность.
Давайте посчитаем количество элементов в обеих группах.
По определению $|\mathbb Z_{mn}| = mn$.
С другой стороны, $|\mathbb Z_m\times \mathbb Z_n| = |\mathbb Z_m| \cdot |\mathbb Z_n| = m n$.
Значит $\Phi$ -- это инъективное отображение между множествами одинакового размера.
А отсюда получаем, что оно обязательно сюръективно.
\end{proof}

В предыдущем утверждении явно сказано, как отображать элементы из $\mathbb Z_{mn}$ в элементы из $\mathbb Z_m\times \mathbb Z_n$.
Однако, стоит сказать, как строится обратное отображение.
Так как $m$ и $n$ взаимно просты, мы имеем $1 = um + vn$ для некоторых $u, v\in \mathbb Z$ по расширенному алгоритмы Евклида.
Теперь рассмотрим элемент $a_1 = um = 1 - vn$.
Ясно, что $a_1 \mapsto (0, 1)$ под действием отображения $\Phi$.
Аналогично, элемент $a_2 = vn = 1 - um$ идет в $(1, 0)$.
Значит, элемент $(a, b)$ соответствует элементу $a a_1 + b a_2 \pmod{mn}$ в группе $\mathbb Z_{mn}$.

\begin{examples}
\begin{enumerate}
\item В случае $m = 3$ и $n = 2$, мы имеем $\mathbb Z_6\simeq \mathbb Z_3\times\mathbb Z_2$.
Здесь элемент $1$ идет в $(1, 1)$.
Значит $(1, 1)$ является образующим циклической группы $\mathbb Z_3\times \mathbb Z_2$.
Так как $1 = 3 - 2$, мы видим, что $3$ идет в $(0, 1)$ и $-2$ идет в $(1,0)$  (обратим внимание, что $- 2 = 4$ в $\mathbb Z_6$).
Потому, обратное отображение задано по правилу $(a, b)\mapsto -2a + 3b = 4a + 3b\pmod 6$.

\item Группа $\mathbb Z_2\times \mathbb Z_2$ не является циклической.
Значит, не существует изоморфизма между ней и группой $\mathbb Z_4$.

\item Еще один пример различного представления абелевой группы в виде произведения циклических
\[
\mathbb Z_{30} \simeq \mathbb Z_6 \times \mathbb Z_5 \simeq \mathbb Z_3 \times\mathbb Z_{10}\simeq\mathbb Z_2 \times \mathbb Z_{15} \simeq\mathbb Z_2\times\mathbb Z_3\times \mathbb Z_5
\]
Так что, все пять конструкций дают одну и ту же циклическую группу.

\item В общем случае, если $m = p_1^{k_1}\ldots p_r^{k_r}$, где $p_i$ -- простые, имеем
\[
\mathbb Z_{m} = \mathbb Z_{p_1^{k_1}}\times \ldots \times \mathbb Z_{p_r^{k_r}}
\]
\end{enumerate}
\end{examples}

Как мы видели выше, одна и та же абелева группа может быть записана совершенно разными способами.
Как же быстро понять, что два представления задают одну и ту же группу?
Ответ содержится в следующем утверждении.

\begin{claim}
\label{claim::FAGClass}
Пусть $G$ -- конечная абелева группа.
Тогда
\begin{enumerate}
\item $G$ единственным образом представляется в следующем виде
\[
G = \mathbb Z_{d_1}\times \ldots \times \mathbb Z_{d_k},\quad\text{где}\; 1 < d_1|d_2|\ldots|d_k\text{ натуральные}
\]

\item С точностью до перестановки множителей $G$ единственным образом представляется в следующем виде
\[
G = \mathbb Z_{p_1^{k_1}}\times \ldots \times \mathbb Z_{p_r^{k_r}},\quad \text{где}\; p_i \text{ -- не обязательно различные простые числа},\; k_i\text{ -- натуральные}
\]
\end{enumerate}
\end{claim}

Важно упомянуть, что простые $p_i$ могут повторяться во втором представлении, например $\mathbb Z_2\times \mathbb Z_4$ -- один из возможных случаев.

\begin{examples}
\begin{enumerate}
\item Пусть $G = \mathbb Z_2 \times \mathbb Z_6$ и $H = \mathbb Z_{12}$.
Обе эти группы представлены в первой форме.
Так как такое представление единственно, то $G$ и $H$ не изоморфны.

\item Пусть $G = \mathbb Z_2 \times \mathbb Z_6$ и $H = \mathbb Z_2\times \mathbb Z_2\times \mathbb Z_3$.
Мы видим, что $G$ представлено в первой форме, а $H$ представлено во второй.
Давайте пересчитаем $G$ во второй форме, используя Китайскую теорему об остатках:
\[
G = \mathbb Z_2 \times \mathbb Z_6 = \mathbb Z_2 \times (\mathbb Z_2 \times \mathbb Z_3) = H
\]
Значит группы изоморфны.
\end{enumerate}
\end{examples}

