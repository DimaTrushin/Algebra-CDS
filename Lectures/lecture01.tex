\ProvidesFile{lecture01.tex}[Lecture 1]


\newpage

\section{Бинарные операции}

В математике часто изучаются разные структуры.
Обычно это множества снабженные дополнительной структурой.
В алгебре обычно множества снабжаются разного рода операциями.
Простейший тип операций -- бинарные операции, то есть операции с двумя аргументами.
Давайте обсудим какие бывают бинарные операции и после перейдем к определению самой простой алгебраической структуры -- группы.

\subsection{Определения}

\begin{definition}
Пусть $X$ -- некоторое множество.
Бинарная операция на $X$ -- это отображение $\circ\colon X\times X\to X$ по правилу $(x, y)\mapsto x\circ y$ для всех $x, y\in X$.
\end{definition}

В этом случае $\circ$ -- это имя операции.
Проще говоря, операция -- это правило, которое съедает два элемента из $X$ и выплевывает один новый элемент, называемый $x\circ y$, из того же множества $X$.
Новый элемент $x\circ y$ обычно называется произведением элементов $x$ и $y$.%
\footnote{Операция может быть какой угодно, например, на множестве целых чисел можно рассматривать сложение, взятие максимума, или что-либо другое, но с абстрактной точки зрения результат операции все равно называется произведением элементов.
Не забывайте, что математика -- это искусство обозначать одинаковые вещи по разному и разные вещи одинаково.}

Обратите внимание, что у бинарных операций есть функциональный стиль обозначения, когда имя операции пишется не между аргументами, а в виде имени функции перед аргументами.
Давайте я повторю определение операции в функциональном стиле.

\begin{definition}
Пусть $X$ -- некоторое множество.
Бинарная операция на $X$ -- это отображение $\mu\colon X\times X\to X$ по правилу $(x, y)\mapsto \mu(x, y)$ для всех $x, y\in X$.
\end{definition}

Это не новое определение, это всего лишь переобозначение предыдущего.
Я буду предпочитать операторное обозначение.

\begin{examples}
Бинарные операции:
\begin{enumerate}
\item Сложение целых чисел.
В операторной форме
\[
+\colon \mathbb Z\times \mathbb Z\to \mathbb Z,\quad (m,n) \mapsto m + n
\]
В функциональной форме
\[
\operatorname{add}\colon \mathbb Z\times \mathbb Z\to \mathbb Z,\quad (m,n) \mapsto \operatorname{add}(m,n) = m+n
\]
Так как мы привыкли к сложению в форме $m + n$, мы хотим, чтобы общее определение было похоже на привычную нам запись.
С другой стороны, многие языки программирования допускают оба вида нотаций.
Но по сути $\operatorname{add}(m,n)$ и $m+n$ это одно и то же.

\item Умножение целых чисел.
В операторной форме
\[
\cdot\colon \mathbb Z\times \mathbb Z\to \mathbb Z,\quad (m,n) \mapsto m \cdot n
\]
В функциональном стиле
\[
\operatorname{mult}\colon \mathbb Z\times \mathbb Z\to \mathbb Z,\quad (m,n) \mapsto \operatorname{mult}(m,n) = m\cdot n
\]

\item Максимум целых чисел.
В операторной форме
\[
\vee\colon \mathbb Z\times \mathbb Z\to \mathbb Z,\quad (m,n) \mapsto m \vee n
\]
В функциональной форме
\[
\operatorname{max}\colon \mathbb Z\times \mathbb Z\to \mathbb Z,\quad (m,n) \mapsto \operatorname{max}(m,n) = m\vee n
\]
На всякий случай поясню, что $\max(m,n) = m \vee n$, то лишь разные обозначения максимума.

\item Минимум целых чисел.
В операторной форме
\[
\wedge\colon \mathbb Z\times \mathbb Z\to \mathbb Z,\quad (m,n) \mapsto m \wedge n
\]
В функциональной форме
\[
\operatorname{min}\colon \mathbb Z\times \mathbb Z\to \mathbb Z,\quad (m,n) \mapsto \operatorname{min}(m,n) = m\wedge n
\]
Как и выше $\min(m,n) = m \wedge n$ это разные обозначения минимума.

\item Просто случайная дурацкая бинарная операция на целых числах
\[
\phi\colon \mathbb Z\times \mathbb Z\to \mathbb Z,\quad (m,n)\mapsto m^2 - n^2
\]
\end{enumerate}
\end{examples}

Давайте резюмируем, что бинарная операция на $X$ -- это любое отображение вида $f\colon X\times X\to X$.
И вы вольны задать его как вам вздумается по любому правилу.
Но разные операции будут иметь разные свойства, какие-то операции будут лучше, чем другие.
Давайте теперь обсудим какие же есть свойства у операций.

\subsection{Свойства}

Можно рассматривать множество различных свойств операций.
Я хочу обсудить лишь те, которые нам понадобятся в дальнейшем для определения группы.

\subsubsection{Ассоциативность}

\begin{definition}
Операция $\circ \colon X\times X\to X$ называется ассоциативной, если для любых элементов $x, y, z\in X$ выполнено $(x\circ y)\circ z = x\circ (y\circ z)$.
\end{definition}

Если у вас есть бинарная операция $\circ$ на множестве $X$, то вы можете посчитать произведение трех элементов $x$, $y$, $z$ двумя разными способами:
\begin{itemize}
\item сначала посчитаем произведение $w = x\circ y$ и потом вычислим $w \circ z = (x\circ y) \circ z$.

\item сначала посчитаем произведение $u = y \circ z$ и потом вычислим $x \circ u = x \circ (y\circ z)$.
\end{itemize}
Если операция взята произвольно, то может случится, что эти два способа дают разные результаты для каких-то значений $x$, $y$ и $z$.
Ассоциативность означает, что не важен порядок, в котором вы вычисляете операции.
Кроме того, если $(x\circ y) \circ z = x \circ (y \circ z)$ для всех $x,y,z\in X$, то на самом деле не имеет значения, как вы расставляете скобки в произвольных произведениях.
Например, все следующие выражения равны $(x\circ y)\circ (z\circ w)$, $x\circ (y\circ (z\circ  w))$ and $((x\circ  y)\circ z)\circ w$, а значит мы можем убрать все скобки и просто записать $x\circ  y \circ z \circ w$.
Поэтому для ассоциативных операций обычно не используют скобки, так как они не существенны.

\begin{examples}
Ниже примеры ассоциативных и не ассоциативных операций.
\begin{enumerate}
\item Целочисленное сложение ассоциативно.
\[
+\colon \mathbb Z\times \mathbb Z\to \mathbb Z,\quad (m,n)\mapsto m+n
\]
Если $m,n,k\in \mathbb Z$, то мы знаем, что $(m + n) + k = m + (n + k)$.

\item Вычитание целых чисел не ассоциативно.
\[
-\colon \mathbb Z\times \mathbb Z\to \mathbb Z,\quad (m,n)\mapsto m-n
\]
Тогда равенство $(m - n) - k = m - (n - k)$ не выполняется для всех целых чисел.
Действительно, если взять $m = n = 0$ и $k = 1$, то левая часть равенство будет $-1$, а правая -- $1$.
Так что, $(0 - 0) - 1 \neq 0 - (0 - 1)$.
\end{enumerate}
\end{examples}

\subsubsection{Нейтральный элемент}

\begin{definition}
Пусть $\circ \colon X\times X\to X$ -- некоторая операция на $X$.
Элемент $e\in X$ называется нейтральным если для каждого элемента $x\in X$ выполнены равенства $x\circ e = x$ и $e \circ x = x$.
\end{definition}

По простому, нейтральный элемент $e\in X$ -- это такой элемент, который ничего не меняет по умножению в смысле операции.

\begin{examples}
Нейтральный элемент может существовать, а может и не существовать.
\begin{enumerate}
\item Целочисленное сложение имеет нейтральный элемент.
\[
+\colon \mathbb Z\times \mathbb Z\to \mathbb Z,\quad (m,n)\mapsto m+n
\]
Ясно, что элемент $e = 0$ удовлетворяет всем требованиям на нейтральный элемент.
Действительно, для всех $m\in \mathbb Z$ имеем $m + 0 = m$ и $0 + m = m$.

\item Целочисленное вычитание не имеет нейтрального элемента.
\[
-\colon \mathbb Z\times \mathbb Z\to \mathbb Z,\quad (m,n)\mapsto m-n
\]
Давайте покажем, что нет элемента $e\in \mathbb Z$ такого, что $e - m = m$ для всех $m\in \mathbb Z$.
Действительно, если такой $e$ существует, то $e = 2m$ для каждого $m\in \mathbb Z$.
Но это не возможно, поскольку для $m = 0$, $e = 0$ а для $m = 1$, $e = 2$, противоречие.
С другой стороны, отметим, что $m - 0 = m$ для всех $m\in \mathbb Z$.
То есть $0$ является нейтральным только с одной стороны для вычитания.
\end{enumerate}
\end{examples}

Последний пример показывает, что вообще говоря не достаточно проверять только одно из условий $x \circ e = x$ или $e \circ x = x$.
Это очень частая ошибка.
Постарайтесь не забыть оба условия.

Правильный вопрос, которым теперь надо задаться: а сколько нейтральных элементов может быть?
Правильный ответ -- не более одного.
Давайте покажем это.

\begin{claim}
Пусть $X$ -- некоторое множество и $\circ \colon X\times X\to X$ -- бинарная операция.
Тогда существует не более одного нейтрального элемента.
\end{claim}
\begin{proof}
Если нейтральных элементов нет, то и доказывать нечего.
Пусть теперь $e$ и $e'$ -- два произвольных нейтральных элемента.
Мы должны показать, что они равны.
Рассмотрим произведение $e \circ e'$.
Так как $e$ является нейтральным элементом,  $e \circ x = x$ для любого $x\in X$.
В частности при  $x = e'$ мы получим, что $e \circ e' = e'$.
С другой стороны, так как $e'$ является нейтральным элементом, то $x \circ e' = x$ для любого $x\in X$.
И значит в частности при $x = e$ имеем $e \circ e' = e$.
То есть $e = e\circ e' = e'$.
\end{proof}

\subsubsection{Обратный элемент}

Я хочу начать с замечания, что это свойство зависит от предыдущего.
А именно, для того чтобы говорить об обратных элементах необходимо, чтобы для операции существовал нейтральный элемент.
Если же нейтрального элемента нет, то нет и способа говорить об обратимых элементах.

\begin{definition}
Пусть $\circ \colon X\times X\to X$ -- некоторая операция с нейтральным элементом $e\in X$.
Элемент $y\in X$ называется обратным к элементу  $x\in X$, если выполнено $x \circ y = e$ и $y \circ x = e$.
\end{definition}

Я напомню, что нейтральный элемент единственный если существует.
Потому элемент $e$ корректно определен в  равенствах выше.

Правильный вопрос, которым надо задаться: а сколько может быть обратных элементов для заданного элемента $x\in X$?
Оказывается, что не больше одного, если операция ассоциативна.

\begin{claim}
Пусть $\circ \colon X\times X \to X$ -- некоторая ассоциативная бинарная операция с нейтральным элементом $e\in X$.
Тогда, для любого $x \in X$ существует не более одного обратного элемента.
\end{claim}
\begin{proof}
Давайте зафиксируем элемент $x\in X$.
Если для него нет обратного, то и доказывать нечего.
Теперь предположим, что $y_1$ и $y_2$ -- это два обратных элемента к $x$.
Последнее означает, что выполнены равенства
\[
\left\{
\begin{aligned}
&x \circ y_1 = e\\
&y_1 \circ x = e
\end{aligned}
\right.
\quad\text{и}\quad
\left\{
\begin{aligned}
&x \circ y_2 = e\\
&y_2 \circ x = e
\end{aligned}
\right.
\]
Теперь рассмотрим произведение $y_1 \circ x \circ y_2$.
Так как $\circ$ ассоциативна, то расстановка скобок не имеет значение, то есть $(y_1 \circ x) \circ y_2 = y_1 \circ (x \circ y_2)$.
Если посчитать левую часть, то получим:
\[
(y_1 \circ x) \circ y_2 = e \circ y_2 = y_2
\]
А для правой части имеем:
\[
y_1 \circ (x \circ y_2) = y_1 \circ e = y_1
\]
Значит $y_2 = (y_1 \circ x) \circ y_2 = y_1 \circ (x \circ y_2) = y_1$ и все доказано.
\end{proof}

Так как в общем случае существует не более одного обратного для элемента $x$, то его принято обозначать через $x^{-1}$.

\begin{examples}
\begin{enumerate}
\item Предположим, что операция -- сложение целых чисел.
\[
+\colon \mathbb Z\times \mathbb Z\to \mathbb Z,\quad (m,n)\mapsto m+n
\]
Нейтральный элемент у нас $0$.
Если $n\in \mathbb Z$, то обратный к нему будет $-n$.
Действительно, $n + (-n) = 0$ и $(-n) + n = 0$.
Значит любой элемент имеет обратный для этой операции.

\item Предположим, что операция -- это умножение целых чисел
\[
\cdot\colon \mathbb Z\times \mathbb Z\to \mathbb Z,\quad (m,n)\mapsto m\cdot n
\]
Нейтральный элемент -- $1$.
Если $n = 1$, то его обратный будет тоже $1$.
Если $n = -1$, то его обратный будет $-1$.
Если же $n\neq \pm1$, то обратного не существует в $\mathbb Z$.
Потому только два элемента обратимы для этой операции.
\end{enumerate}
\end{examples}

\subsubsection{Коммутативность}

\begin{definition}
Бинарная операция $\circ \colon X\times X\to X$ называется коммутативной если для любых $x,y\in X$ выполнено $x \circ y = y\circ x$.
\end{definition}

То есть коммутативность означает, что нам не важен порядок операндов в операции.

\begin{examples}
\begin{enumerate}
\itemЦелочисленное сложение коммутативно.
\[
+\colon \mathbb Z\times \mathbb Z\to \mathbb Z,\quad (m,n)\mapsto m+n
\]
Действительно, для любых $m,n\in \mathbb Z$, мы имеем $m + n = n + m$.

\item Целочисленное вычитание не коммутативно.
\[
-\colon \mathbb Z\times \mathbb Z\to \mathbb Z,\quad (m,n)\mapsto m-n
\]
Коммутативность означает равенство $m - n = n - m$ для всех целых $m,n$.
Ясно, что это не выполнено уже в случае $m = 0$ и $n = 1$.
\end{enumerate}
\end{examples}

\section{Группы}

\subsection{Определение}

Теперь мы готовы к тому, чтобы дать определение одного из самых важных в алгебре объектов -- группы.
Прежде чем сделать это, я хочу пояснить, что мы встретим много абстрактных определений в будущем и все они будут сотканы по единому шаблону.
Давайте я проясню этот шаблон в начале.
В любом абстрактном определении есть две части.
В первой части говорится какие данные нам даны.
А во второй части говорится каким аксиомам эти данные должны удовлетворять.%
\footnote{Если проводить аналогию с программированием, то первая часть описывает интерфейс, а вторая часть -- это контракт на интерфейс.}

\begin{definition}
Определение группы
\begin{itemize}
\item\textbf{Данные:} 
\begin{enumerate}
\item $G$ -- множество.

\item Операция $\circ \colon G\times G\to G$.
\end{enumerate}
\item\textbf{Аксиомы:}
\begin{enumerate}
\item Операция $\circ$ ассоциативна.

\item Операция $\circ$ обладает нейтральным элементом.

\item Каждый элемент $x\in G$ имеет обратный.
\end{enumerate}
\end{itemize}
В этом случае мы будем говорить, что пара $(G, \circ)$ является группой.
Чтобы упростить обозначения, мы будем обычно говорить, что просто $G$ является группой, подразумевая, что на $G$ задана некоторая фиксированная операция.
Если в дополнение к аксиомам выше выполнена следующая аксиома
\begin{itemize}
\item[]
\begin{enumerate}
\setcounter{enumi}{3}
\item Операция $\circ$ коммутативна.
\end{enumerate}
\end{itemize}
То группа $G$ называется абелевой или просто коммутативной.
\end{definition}

Если коротко, то группа -- это множество с <<хорошей>> операцией.
Здесь слово <<хорошая>> означает, что нам не важно как расставлять скобки, у нас есть нейтральный элемент и на любой элемент можно поделить.
Если же в дополнение ко всему не важно в каком порядке стоят аргументы операции, то группа называется абелевой.

\begin{examples}
\begin{enumerate}
\item Целые числа по сложению $(\mathbb Z, +)$ образуют абелеву группу.
Действительно, операция $+$ ассоциативна, нейтральный элемент -- $0$, для каждого числа $n$ есть его обратный $-n$ и порядок аргументов в сложении не важен $n + m = m + n$.
Мы обычно называем эту группу просто $\mathbb Z$ подразумевая, что операция обязательно сложение.

\item Целые числа по умножению $(\mathbb Z, \cdot)$ группу не образуют.
Мы знаем, что операция ассоциативна и есть нейтральный элемент $1$.
И мы уже проверяли, что только $\pm 1$ являются обратимыми элементами.

\item Не нулевые вещественные числа по умножению $(\mathbb  R^*, \cdot)$ образуют абелеву группу.
Действительно, умножение ассоциативно.
Нейтральным элементом будет будет $1$, для всякого элемента $x$ обратным будет $1/x$, и порядок аргументов в умножении не важен $xy = yx$.

\item Пусть $n$ -- положительное целое, тогда множество $\mathbb Z_n = \{0,1, \ldots, n-1\}$ с операцией $a + b \pmod n$ является абелевой группой.
Для простоты операция сложения по модулю $n$ так же обозначается просто $+$.

\item Пусть $n$ -- положительное целое.
Положим $\mathbb Z_n^* = \{m\in \mathbb Z_n \mid (m,n) = 1\}$ (множество всех чисел из $\mathbb Z_n$ взаимно простых с $n$), а операцию зададим как $a \cdot b \pmod n$.
В этом случае мы так же получим абелеву группу.
Для простоты операция в $\mathbb Z_n^*$ обозначается как операция умножения $\cdot$.

\end{enumerate}
\end{examples}

\subsection{Мультипликативная и аддитивная нотации}

В определении группы $G$ мы обозначали операцию $\circ$.
Если надо использовать произведение нескольких элементов, то нам приходится писать $x \circ y \circ z \circ w$.
Это не очень удобно.
Вместо этого есть два более привычных стиля.
А именно, давайте будем обозначать операцию как умножение $\cdot$ или как сложение $+$.
Тогда получаются две разные нотации: мультипликативная и аддитивная.
\begin{center}
\begin{tabular}{c|c|c|c}
{}&{Мультипликативная}&{Аддитивная}\\
\hline
{Операция}&{$\cdot \colon G\times G\to G$}&{$+\colon G\times G\to G$}\\
{На элементах}&{$(x, y)\mapsto xy$}&{$(x,y)\mapsto x+y$}\\
{Нейтральный элемент}&{$1$}&{$0$}\\
{Обратный элемент}&{$x^{-1}$}&{$-x$}\\
{Степерь элемента}&{$x^n = \underbrace{x \cdot\ldots \cdot x}_n$}&{$nx = \underbrace{x + \ldots + x}_n$}\\
\end{tabular}
\end{center}
Обычно мультипликативная нотация используется в случае неабелевых групп или когда свойство коммутативности вообще говоря не известно.
А аддитивная нотация зарезервирована сугубо для абелевых групп.
Я буду в основном использовать мультипликативную нотацию.

Я подчеркну, что указанные нотации -- это всего лишь два разных способа обозначать операцию $\circ$, а не какие-то новые специальные операции.
 То есть мы выбираем обозначение для $\circ$ в виде $\cdot$ или $+$ в зависимости от наших предпочтений.
Не надо путать эти обозначения с операциями сложения и умножения целых чисел.
В случае произвольной группы $G$ путаницы быть не должно, потому что там нет никаких заранее заданных операций сложения и умножения.
Однако, если мы работаем с целыми числами (вещественными, рациональными, комплексными и т.д.), то операции $+$ и $\cdot$ обозначают обычные сложение и умножение.

\subsection{Подгруппы}

\begin{definition}
Пусть $G$ -- некоторая группа.%
\footnote{Строго говоря $(G,\cdot)$, но я буду использовать более короткие обозначения.}
Определим подгруппу $H$ в группе $G$ следующим образом.
\begin{itemize}
\item \textbf{Данные:} 
\begin{enumerate}
\item Подмножество $H\subseteq G$.
\end{enumerate}
\item \textbf{Аксиомы:}
\begin{enumerate}
\item Нейтральный элемент $1$ группы $G$ принадлежит $H$.

\item Если $x,y\in H$, то $x  y\in H$.

\item Если $x\in H$, то $x^{-1}\in H$.
\end{enumerate}
\end{itemize}
В этом случае, мы говорим, что $H$ -- подгруппа в группе $G$.
\end{definition}

Стоит отметить, что если $H$ -- подгруппа в группе $(G,\cdot)$, то $\cdot$ можно ограничить на $H$ и получится операция на $H$.
В этом случае $(H, \cdot)$ удовлетворяет всем аксиомам группы.
Таким образом подгруппа $H$ сама является группой относительно той же самой операции (или точнее относительно ограничения операции), что была на группе $G$.

\begin{examples}
Пусть $G =\mathbb Z$ по сложению.
\begin{enumerate}
\item Если $H\subseteq \mathbb Z$ -- подмножество четных чисел $H = 2\mathbb Z$, то $H$ является подгруппой.

\item Если $H\subseteq \mathbb Z$ -- подмножество нечетных чисел $H = 1 + 2 \mathbb Z$, то $H$ не является подгруппой.
В этом случае $H$ не содержит нейтрального элемента $0$ и не замкнуто относительно операции сложения.
\end{enumerate}
\end{examples}

\subsection{Циклические группы}

Пусть $G$ -- некоторая группа и $g\in G$ -- ее элемент.
Тогда мы можем определить целочисленные степени элемента $g$ по следующим правилам.
\begin{center}
\begin{tabular}{c | c}
{Мультипликативная нотация}&{Аддитивная нотация}\\
\hline
{
$
g^n =
\left\{
\begin{aligned}
&\underbrace{g\cdot \ldots \cdot g}_n,&&n>0\\
&1,&&n=0\\
&\underbrace{g^{-1}\cdot \ldots \cdot g^{-1}}_{-n},&&n<0\\
\end{aligned}
\right.
$
}&{
$
n g =
\left\{
\begin{aligned}
&\underbrace{g+ \ldots + g}_n,&&n>0\\
&0,&&n=0\\
&\underbrace{(-g) + \ldots +(- g)}_{-n},&&n<0\\
\end{aligned}
\right.
$
}\\
\end{tabular}
\end{center}

\begin{claim}
Пусть $G$ -- некоторая группа.
Тогда
\begin{enumerate}
\item Для любых $x,y\in G$ выполнено $(xy)^{-1} = y^{-1}x^{-1}$.

\item Для любого $g\in G$ верно $(g^{-1})^n = (g^n)^{-1} = g^{-n}$.

\item Для любого $g\in G$ и любых $n,m\in \mathbb Z$ верно $g^n g^m = g^{n+m}$.
\end{enumerate}
\end{claim}
\begin{proof}
1) Нам надо показать, что $(xy)^{-1} = y^{-1}x^{-1}$.
С психологической точки зрения удобно обозначить $ y^{-1}x^{-1}$ через $z$.
Если мы покажем, что $(xy)z = z(xy) = 1$, то это будет означать, что $z = (xy)^{-1}$ по определению.
Теперь посчитаем
\[
(xy) z = xy z = xy y^{-1}x^{-1} = x x^{-1} = 1
\]
Аналогично делается и второе равенство.

2) Сначала покажем первое равенство.
Давайте применим предыдущее свойство несколько раз, получим
\[
(g_1\cdot \ldots \cdot g_n)^{-1} = g_n^{-1}\cdot\ldots\cdot g_1^{-1},\;\text{ whenever }g_1,\ldots,g_n\in G
\]
При подстановке $g_1 =\ldots = g_n = g$, получим нужное равенство для $n > 0$.

Если $n = 0$, то по определению $(g^{-1})^0 = 1$.
С другой стороны, $(g^0)^{-1} = 1^{-1} = 1$ потому что обратный к $1$ есть $1$.

Если $n < 0$, то по определению
\[
(g^{-1})^n = \underbrace{(g^{-1})^{-1}\cdot\ldots \cdot (g^{-1})^{-1}}_{-n}
\]
С другой стороны
\[
(g^n)^{-1} = (\underbrace{g^{-1}\cdot \ldots \cdot g^{-1}}_{-n})^{-1} = \underbrace{(g^{-1})^{-1}\cdot\ldots \cdot (g^{-1})^{-1}}_{-n}
\]
где последнее равенство берется из предыдущего пунка утверждения.

Теперь надо проверить второе равенство.
В случае $n > 0$ имеем по определению
\[
(g^{-1})^{n} =  (\underbrace{g^{-1}\cdot \ldots \cdot g^{-1}}_{n})\quad \text{и}\quad
g^{-n} =  (\underbrace{g^{-1}\cdot \ldots \cdot g^{-1}}_{n})
\]
Значит левая часть равна правой.
Если $n = 0$, то обе части равны $1$.
Теперь рассмотрим $n < 0$.
Для удобства изменим степень с $n$ на $-n$ и можно считать, что $n > 0$.
Получаем
\[
(g^{-1})^{-n} =  (\underbrace{(g^{-1})^{-1}\cdot \ldots \cdot (g^{-1})^{-1}}_{n})\quad \text{и}\quad
g^{-(-n)} =  (\underbrace{g\cdot \ldots \cdot g}_{n})
\]
То есть теперь достаточно показать, что $(g^{-1})^{-1} = g$.
А это делается по определению.
Элемент $g$ удовлетворяет равенствам $g g^{-1} = 1$ и $g^{-1} g = 1$, то есть $g$ является обратным к $g^{-1}$, что и требовалось.

3) Мы должны рассмотреть следующие $4$ случая:
\begin{enumerate}
\item $n\geqslant 0$ and $m\geqslant 0$.

\item $n < 0$ and $m\geqslant 0$.

\item $n\geqslant 0$ and $m < 0$.

\item $n < 0$ and $m < 0$.
\end{enumerate}
Пусть у нас первый случай:
\[
g^n g^m = \underbrace{g\cdot \ldots \cdot g}_n\cdot\underbrace{g\cdot \ldots \cdot g}_m = \underbrace{g\cdot \ldots \cdot g}_{n+m} = g^{n+m}
\]
Для удобства рассмотрим $g^{-n} g^{m}$ где $n>0$ and $m\geqslant 0$ во втором случае.
Тогда
\[
g^{-n}g^{m} = \underbrace{g^{-1}\cdot \ldots \cdot g^{-1}}_{n}\cdot\underbrace{g\cdot \ldots \cdot g}_m
\]
Мы сокращаем множители в середине выражения.
Если $n >m$, получим
\[
\underbrace{g^{-1}\cdot \ldots \cdot g^{-1}}_{n - m} = g^{-n + m}
\]
Если $n < m$, имеем
\[
\underbrace{g\cdot \ldots \cdot g}_{m - n} = g^{m - n}
\]
Если $n = m$ получается $1 = g^{m - n}$.

Третий случай по сути является вторым с переставленными множителями.
Значит остается разобрать четвертый случай.
Опять же для удобства будем считать, что нам даны $g^{-n}$ и $g^{-m}$, где $n>0$ и $m > 0$.
Тогда
\[
g^{n-} g^{-m} = \underbrace{g^{-1}\cdot \ldots \cdot g^{-1}}_{n}\underbrace{g^{-1}\cdot \ldots \cdot g^{-1}}_{m} = \underbrace{g^{-1}\cdot \ldots \cdot g^{-1}}_{n + m} = g^{-n - m}
\]
Что и требовалось показать.
\end{proof}


\begin{definition}
Пусть $G$ -- группа и $g\in G$ -- некоторый элемент.
Тогда обозначим множество всех целых степеней $g$ следующим образом
\[
\langle g \rangle = \{ \ldots, g^{-2},g^{-1},1, g, g^2, \ldots\} = \{g^k \mid k\in \mathbb Z\}
\]
Данное подмножество удовлетворяет определению подгруппы в группе $G$.%
\footnote{Действительно, нейтральный элемент содержится в ней.
Это множество замкнуто по умножению в силу свойства~(3) предыдущего утверждения и в силу свойства~(2) предыдущего утверждения с каждым элементом лежит его обратный.}
Эта группа называется циклической подгруппой порожденной $g$.
Элемент $g$ называется порождающим подгруппы $\langle g\rangle$.
\end{definition}

В аддитивной нотации циклическая подгруппа имеет вид
\[
\langle g\rangle = \{\ldots, -2 g, - g , 0, g, 2g, \ldots\} = \{kg\mid k \in \mathbb Z\}
\]
По построению $\langle g\rangle$ -- это самая маленькая подгруппа в $G$ содержащая элемент $g$.



\begin{definition}
Пусть $G$ -- некоторая группа.
Если найдется элемент $g\in G$ такой, что $\langle g\rangle = G$, то группа $G$ называется циклической.
\end{definition}


\begin{examples}
\begin{enumerate}
\item Группа $(\mathbb Z, +)$ является циклической.
Ее образующие $1$ и $-1$.

\item Группа $(\mathbb Z_n, +)$ является циклической.

\item Группа перестановок на $n$ элементах $S_n$ не является циклической при $n > 2$.

\item Группа $(\mathbb R, +)$ не является циклической.
\end{enumerate}
\end{examples}

\begin{definition}
Пусть $G$ -- некоторая группа и $g\in G$ -- ее элемент.
Порядок элемента $g$ -- это минимальное положительное целое число такое, что $g^n = 1$ и $\infty$ если такого числа нет.
Порядок $g$ обозначается $\ord g$.
\end{definition}

\paragraph{Замечания}

\begin{itemize}
\item Обратите внимание, что $g  = 1$ тогда и только тогда, когда $\ord g = 1$.

\item Если мы используем аддитивную нотацию, то есть будем обозначать операцию через $+$, то порядок $g\in G$ -- это такое минимальное положительное целое $n$, что $n g = 0$.
\end{itemize}

\begin{claim}
\label{claim::CyclicClass}
Пусть $G$ -- некоторая группа и $g\in G$ ее элемент.
Тогда есть два возможных случая
\begin{enumerate}
\item Все элементы $g^n$ и $g^m$ различны при различных $n,m\in \mathbb Z$.

\item Существует положительное целое $n$ такое, что степени $1, g, g^2, \ldots, g^{n-1}$ различны.
Более того, степени повторяются по циклу, а именно в ряду
\[
\underbrace{\ldots, g^{-2},g^{-1}},\underbrace{1, g, g^2, \ldots, g^{n-1}}, \underbrace{g^n, g^{n+1},\ldots,g^{2n - 1}}, \underbrace{g^{2n},\ldots}\\
\]
элементы $g^{kn}, g^{1 + kn}, \ldots, g^{n-1 + nk}$ совпадают с элементами $1, g, \ldots, g^{n-1}$ для любого $k\in \mathbb Z$.
В частности, в этом случае
\[
\langle g\rangle =\{1,g,\ldots g^{n-1}\}
\]
При этом $n = \ord g$.
\end{enumerate}
\end{claim}
\begin{proof}
Если $g^n \neq g^m$ для всех различных $m, n\in \mathbb Z$, то доказывать нечего, у нас первый случай.

Давайте предположим, что для каких-то $m\neq n\in \mathbb Z$ у нас выполнено равенство $g^n = g^m$.
Можно считать, что $n > m$.
Тогда умножим обе части равенства на $g^{-m}$ и по правилам перемножения степеней получим $g^{n-m} = 1$.
Значит для некоторого $n > 0$ имеем $g^n = 1$.

Рассмотрим минимальное положительное $n$ такое, что $g^n = 1$.
Я утверждаю, что все степени $1, g, \ldots, g^{n-1}$ различны.
Действительно, если $g^k = g^s$ для некоторых $k,s \in [0, n-1]$ и $k > s$, тогда $g^{k-s} = 1$.
А это значит, что $k - s$ не ноль и строго меньше, чем $n$.
Последнее противоречит выбору $n$.

Теперь проверим, что любая степень $g^N$ совпадает со степенью из списка  $1, g, \ldots, g^{n-1}$.
Для этого поделим $N$ с остатком на $n$, получим $N = qn + r$, где $0 \leqslant r < n$.
Тогда
\[
g^N = g^{qn + r} = (g^n)^q g^r = g^r
\]
Осталось лишь заметить, что выбранное нами $n$ по определению является $\ord g$.
\end{proof}

\paragraph{Замечания}

\begin{itemize}
\item Отметим, что $n$ может быть равным $1$ в случае, когда $g$ совпадает с нейтральным элементом.

\item Из предыдущего утверждения следует, что $\ord g$ совпадает с количеством элементом в подгруппе $\langle g \rangle$.

\end{itemize}

