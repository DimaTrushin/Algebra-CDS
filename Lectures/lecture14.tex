\ProvidesFile{lecture14.tex}[Lecture 14]


\subsection{Фактор кольца}

Мы с вами научились склеивать элементы множеств и групп.
Теперь давайте научимся склеивать элементы колец.
Для этого нам понадобится кольцо $R$ и какое-то отношение эквивалентности $\sim$ на нем.
При этом мы хотим, чтобы это отношение эквивалентности было согласовано с операциями, то есть выполняется следующее.

\begin{definition}
Пусть $R$ кольцо и $\sim$ -- отношение эквивалентности на нем такое, что
\begin{enumerate}
\item Для любых $x\sim y$ и $u \sim v$ следует, что $x + u \sim y + v$.

\item Для любых $x\sim y$ и $u\sim v$ следует, что $x \cdot u \sim y \cdot v$.
\end{enumerate}
Тогда отношение $\sim$ называется конгруэнтностью на $R$.
\end{definition}

Если у нас есть конгруэнтность $\sim$ на кольце $(R, +, \cdot)$, то мы можем забыть про операцию умножения и получим конгруэнтность на абелевой группе $(R, +)$.
Но мы знаем из раздела~\ref{section::CongrNormalSub}, что такая конгруэнтность описывается с помощью подгруппы $I = \{ x\in R \mid x \sim 0\}$.
При этом $(I, +) \subseteq (R, +)$ будет абелевой подгруппой.
Таким образом два элемента $x,y\in R$ эквивалентны тогда и только тогда, когда $x - y \in I$.
Классы эквивалентности -- это смежные классы вида $x + I$, где $x\in R$.
Обратите внимание, что сейчас мы лишь воспользовались тем, что $\sim$ согласована со сложением.
Давайте покажем, что в этих условиях согласованность с умножением означает, что $I$ -- идеал в кольце $R$.

Пусть $\sim$ согласована с умножением.
Нам надо показать, что для любого $x\in I$ и $a\in R$ выполнено, что $ax\in I$ и $xa\in I$.
По определению $x\in I$ означает, что $x\sim 0$.
Для любого $a\in R$ верно, что $a\sim a$, а следовательно верно, что $x \cdot a \sim 0 \cdot a = 0$ и $a \cdot x = a \cdot 0  = 0$.
А это значит, что $ax$ и $xa$ лежат в $I$, что и требовалось.
Обратно, пусть $I$ -- идеал, надо показать, что $\sim$ согласована с умножением.
Пусть $x\sim y$ и $u\sim v$.
Это значит, что $x = y + a$, где $a\in I$ и $u = v + b$, где $b \in I$.
Тогда
\[
x\cdot u = (y + a) \cdot ( v + b) = y \cdot v + \underbrace{a \cdot v + y \cdot b + a \cdot b}_{\in I}
\]
Значит действительно, элементы $x\cdot u $ и $y\cdot v$ эквивалентны, что и требовалось.

Давайте я резюмирую, что мы тут получили.
Если у нас есть кольцо $R$ и в нем идеал $I$, то мы определяем отношение эквивалентности $x\sim y$ тогда и только тогда, когда $x - y \in I$.
В этом случае кольцо $R$ нарезается на классы эквивалентности вида $x + I$.
Множество классов эквивалентности $R/\sim = R/I$ является кольцом относительно операций: сложение по правилу $x + I + y + I = x + y + I$ и умножение по правилу $(x + I) (y + I) = xy + I$.
При этом нулем является смежный класс $I$, а единицей  -- $1 + I$.

\paragraph{Пример}

\begin{enumerate}
\item Рассмотрим кольцо целых чисел $(\mathbb Z, +, \cdot)$.
И пусть $I = n\mathbb Z$ -- фиксированный идеал.
Покажем, что кольцо $\mathbb Z/ n \mathbb Z$ изоморфно (как кольцо) кольцу остатков $\mathbb Z_n = \{0,1,\ldots,n-1\}$.
Действительно, элементы $\mathbb Z/n\mathbb Z$ -- это смежные классы вида $k + n\mathbb Z$.
Элементы смежного класса $k + n\mathbb Z$ имеют вид $k +nh$, где $h\in \mathbb Z$.
Среди таких элементов есть только один, который попадает в интервал от $0$ до $n-1$ -- это остаток $k$ при делении на $n$.
Таким образом, каждому смежному классу $k + n\mathbb Z$ соответствует один единственный остаток $k \pmod n$.
Значит, мы можем установить биекцию $\mathbb Z/n\mathbb Z \to \mathbb Z_n$ по правилу $k + n\mathbb Z \mapsto k \pmod n$.
Так же легко проверить, что это отображение уважает умножение, сложение и единицу, а значит является изоморфизмом колец.%
\footnote{Ниже я покажу, как это проверить менее кустарными способами.}

\item Давайте вспомним конструкцию кольца полиномиальных остатков.
Если есть поле $F$ и многочлен $0\neq f \in F[x]$, то мы можем построить $F[x] / (f) = \{g\in F[x] \mid \deg g < \deg f\}$, где операция сложения обыкновенная, а умножение задано по модулю $f$.
Давайте рассмотрим идеал $I = (f) = fF[x]$.
Покажем, что кольцо остатков изоморфно фактор кольцу $F[x]/I$.
Действительно элементы фактор кольца -- это смежные классы $g + (f)$.
Элементы одного смежного класса имеют вид $g + fh$, где $h\in F[x]$.
Среди таких элементов есть только один элемент степени меньше $\deg f$ -- это остаток $g \pmod f$.
Таким образом мы нашли в каждом смежном классе единственный остаток.
То есть у нас есть биекция между смежными классами и остатками.
Аналогично, по определению можно проверить, что это дает изоморфизм между фактор кольцом $F[x]/ I$ и кольцом остатков $F[x]/(f)$.%
\footnote{Ниже я покажу, как это сделать более изящно.}
\end{enumerate}

Как и в случае групп, нам понадобятся две теоремы, которые позволяют работать с фактор кольцами эффективно.
Первая теорема описывает сами факторы, а вторая показывает, как описать их гомоморфизмы.

\begin{claim}
[Первая теорема о гомоморфизме для колец]
Пусть $\phi\colon R\to S$ -- гомоморфизм колец.
Тогда $\bar \phi \colon R/\ker \phi \to \Im \phi$ по правилу $x + \ker \phi \mapsto \phi(x)$ является изоморфизмом колец.
\end{claim}
\begin{proof}
Посмотрим на наш гомоморфизм колец $\phi\colon (R, +, \cdot) \to (S, +, \cdot)$.
Забудем про операцию умножения, тогда получим гомоморфизм абелевых групп: $\phi \colon (R, +) \to (S, +)$.
Тогда по теореме о гомоморфизме для групп (утверждение~\ref{claim::HomoThmGroups}) мы знаем, что $\bar \phi \colon R/\ker \phi \to \Im\phi$ будет изоморфизмом абелевых групп.
И действует этот изоморфизм по правилу $x + \ker \phi \mapsto \phi(x)$.
Так же обратим внимание, что образ и ядро $\phi$ как гомоморфизма колец, совпадает с образом и ядром $\phi$ как гомоморфизма абелевых групп.
Потому, чтобы завершить утверждение, нам лишь надо показать, что отображение $\bar\phi$ полученное выше уважает умножение.
То есть осталось показать, что
\[
\bar\phi((x + \ker \phi) (y + \ker \phi)) = \bar \phi(x + \ker \phi) \bar \phi (y + \ker \phi)
\]
Но для этого надо просто посчитать левую и правую часть.
Начнем с левой части:
\[
\bar\phi((x + \ker \phi) (y + \ker \phi)) = \bar\phi(xy + \ker \phi) = \phi(xy)
\]
С другой стороны, правая часть вычисляется в
\[
 \bar \phi(x + \ker \phi) \bar \phi (y + \ker \phi) = \phi(x) \phi(y)
\]
И так как $\phi$ -- гомоморфизм колец, мы получили одно и то же выражение.
\end{proof}

\paragraph{Примеры}

\begin{enumerate}
\item Давайте опишем фактор кольцо $\mathbb Z / n \mathbb Z$ используя предыдущую теорему.
Для этого посмотрим на гомоморфизм колец $\phi \colon \mathbb Z\to \mathbb Z_n$ по правилу $k \mapsto k \pmod n$.
Ясно, что это сюръективный гомоморфизм, а его ядро есть $n \mathbb Z$.
Значит по теореме о гомоморфизме получаем, что $\mathbb Z/ n \mathbb Z = \mathbb Z_n$.

\item Пусть $F[x]$ -- кольцо многочленов над полем и $I = (f)$ для некоторого $0\neq f \in F[x]$.
Рассмотрим гомоморфизм $\phi\colon F[x]\to F[x]/(f)$ в кольцо остатков, по правилу $g \mapsto g \pmod f$.
Так как это сюръективный гомоморфизм с ядром $I = (f)$, то по теореме о гомоморфизме мы получаем, что фактор кольцо $F[x] / I$ изоморфно кольцу остатков $F[x]/(f)$.
\end{enumerate}

\begin{claim}
[Вторая теорема о гомоморфизме для колец]
Пусть $\phi\colon R\to S$ -- гомоморфизм колец, $I\subseteq R$ -- идеал и $\pi \colon R\to R/I$ -- отображение факторизации.
Тогда существует гомоморфизм колец $\gamma \colon R/I \to S$ такой, что $\phi = \gamma \circ \pi$ тогда и только тогда, когда $I 
\subseteq \ker \phi$.
Изобразим все на рисунке ниже
\[
\xymatrix{
	{R}\ar[r]^{\phi}\ar[d]_{\pi}&{S}\\
	{R/I}\ar@{-->}[ru]_{\gamma}&{}
}
\]
При существовании такое $\gamma$ единственно и задается формулой $\gamma(x + I) = \phi(x)$.
\end{claim}
\begin{proof}
Предположим, что $I\not\subseteq \ker \phi$.
Мы можем забыть про операцию умножения и в этом случае мы находимся под действием утверждения~\ref{claim::HomoThmGroups2}, которое гарантирует, что не существует такого $\gamma\colon R/I\to S$, который бы уважал операцию сложения.
А значит и подавно не существует такого $\gamma$, который уважает обе операции -- сложение и умножение.

Обратно.
Пусть $I\subseteq \ker \phi$.
Тогда мы опять забудем про операцию умножения и опять находимся под действием утверждения~\ref{claim::HomoThmGroups2}, которое гарантирует, что существует отображение $\gamma \colon R/I \to S$ такое, что $\phi = \gamma\circ \pi$ и которое сохраняет сложение.
При этом это отображение единственное и задается по правилу $\gamma(x + I) = \phi(x)$.
А это значит, что нам остается показать, что $\gamma$ уважает умножение.
То есть нам надо показать равенство
\[
\gamma((x+I) (y+I)) = \gamma(x + I) \gamma(y + I)
\]
Это делается прямым вычислением.
Посчитаем левую часть
\[
\gamma((x+I) (y+I)) = \gamma(xy + I) = \phi(xy)
\]
А теперь правую часть
\[
\gamma(x + I) \gamma(y + I) = \phi(x) \phi(y)
\]
Так как $\phi$ является гомоморфизмом колец, мы получили равные выражения.
Таким образом мы проверили все, что требовалось в утверждении.%
\footnote{Обратите внимание, как мы лихо все свели к факту про группы.}
\end{proof}

\subsection{Свободные коммутативные кольца}

В случае абелевых групп мы говорили про свободные группы с базисами.
Это было полезно потому что мы могли описать все группы с базисами, а остальные группы представить в виде их фактора.
Нам бы очень хотелось по аналогии придумать понятие <<кольца с базисом>>.
Оказывается, что это можно сделать, но для начала надо переформулировать определение базиса так, чтобы оно работало в любой алгебраической структуре.

Я начну с пересмотра ситуации с группами.
Пусть $G$ -- некоторая абелева группа и $e_1,\ldots, e_n\in G$ -- некоторые элементы этой группы.
И пусть выполнено следующее свойство: для любой абелевой группы $H$ и для любого набора элементов $h_1,\ldots, h_n\in H$ существует единственный гомоморфизм $\phi\colon G\to H$ такой, что $\phi(e_1) = h_1,\ldots,\phi(e_n) = h_n$.
Обратите внимание, что сформулировано свойство элементов $e_1,\ldots,e_n$ в группе $G$.
Можно показать,%
\footnote{Я не буду этого делать, но это не очень сложно.}
что сформулированное свойство в случае абелевых групп равносильно тому, что элементы $e_1,\ldots,e_n$ являются базисом в том смысле, в котором мы определяли ранее.
Но в таком виде это определение можно переформулировать и для колец.
Давайте это сделаем.
Пусть $R$ -- некоторое коммутативное кольцо и $e_1,\ldots,e_n\in R$ -- некоторые его элементы.
И пусть выполнено следующее свойство: для любого коммутативного кольца $S$ и любого набора элементов $s_1,\ldots,s_n\in S$ существует единственный гомоморфизм колец $\phi\colon R\to S$ такой, что $\phi(e_1) = s_1,\ldots,\phi(e_n) = s_n$.
Можно показать,%
\footnote{Я опять же этого делать не буду.}
что выполнение этого свойства возможно только если кольцо $R$ изоморфно кольцу $\mathbb Z[x_1,\ldots,x_n]$ и при этом изоморфизме элементы $e_i$ соответствуют переменным $x_i$.

Обратите внимание, что в кольце многочленов $\mathbb Z[x_1,\ldots,x_n]$ любой элемент получается из переменных $x_1,\ldots, x_n$ с помощью операций $+$, $-$ и $\cdot$.
В этом смысле эти элементы порождают кольцо многочленов.
По аналогии можно рассматривать конечнопорожденные коммутативные кольца.

\begin{definition}
Коммутативное кольцо $S$ является конечно порожденным, если в нем найдется набор элементов $s_1,\ldots, s_n\in S$ такой, что любой другой элемент выражается через $s_1,\ldots,s_n$ с помощью операций $+$, $-$ и $\cdot$.
В этом случае элементы $s_1,\ldots, s_n$ называются порождающими, а кольцо порождено ими.
Обратите внимание, что это равносильно тому, что любой элемент $s$ представляется в виде многочлена от $s_1,\ldots,s_n$ с целыми коэффициентами.
\end{definition}

Если у нас есть $S$, коммутативное кольцо порожденное элементами $s_1,\ldots, s_n$, то мы можем рассмотреть гомоморфизм $\phi\colon \mathbb Z[x_1,\ldots,x_n]\to S$ по правилу $f(x_1,\ldots,x_n) \mapsto f(s_1,\ldots,s_n)$.
Тогда этот гомоморфизм будет сюръективным, а значит по теореме о гомоморфизме кольцо $S$ изоморфно фактору $\mathbb Z[x_1,\ldots, x_n] / I$, где $I = \ker \phi$.
Таким образом изучать конечно порожденные коммутативные кольца -- это все равно, что изучать факторы $\mathbb Z[x_1,\ldots, x_n]$ по все возможным идеалам.
Это объясняет почему нас интересуют идеалы именно в таком кольце.

\subsection{Алгебры над полем}

В предыдущем разделе мы разобрались с аналогом свободных коммутативных колец.
Им оказались кольца вида $\mathbb Z[x_1,\ldots,x_n]$.
Однако, мы с вами изучали кольца немного другого вида, а именно вида $F[x_1,\ldots,x_n]$, где $F$ -- некоторое поле.
На самом деле это тоже свободный объект, только не среди коммутативных колец, а немножечко в другом смысле.
Давайте я постараюсь объяснить в каком.

\begin{definition}
Пусть $F$ -- некоторое поле и $R$ -- кольцо.
Тогда если $F$ лежит в центре $R$, то $R$ называется $F$-алгеброй.
\end{definition}

\begin{definition}
Пусть $F$ -- некоторое поле и $R$ и $S$ -- это $F$-алгебры.
Отображение $\phi\colon R\to S$ называется гомоморфизмом $F$-алгебр если
\begin{enumerate}
\item $\phi$ является гомоморфизмом колец.

\item Для любого $a\in F$ выполнено $\phi(a) = a$.
\end{enumerate}
\end{definition}

Таким образом $F$-алгебры -- это просто кольца, содержащие $F$ в своем центре.
Тонкостью тут является, что мы для них разрешаем не все возможные гомоморфизмы, а только те, что действуют на поле $F$ тождественно.

Можно показать, что для коммутативных $F$-алгебр свободным объектом является кольцо многочленов $F[x_1,\ldots,x_n]$, а переменные $x_1,\ldots,x_n$ будут базисом в этом смысле.
Обратите внимание, что любой элемент кольца $F[x_1,\ldots,x_n]$ выражается через $x_1,\ldots,x_n$, через элементы $F$ с помощью операций $+$, $-$ и $\cdot$.
Это дает нам понятие конечно порожденной алгебры над полем $F$.

\begin{definition}
Пусть $S$ -- коммутативная $F$-алгебра и пусть $s_1,\ldots,s_n$ -- ее элементы.
Тогда мы будем говорить, что $S$ порождена $s_1,\ldots, s_n$ как $F$-алгебра, если любой элемент $S$ выражается через $x_1,\ldots,x_n$ и элементы $F$ с помощью операций $+$, $-$ и $\cdot$.
Или другими словами, что любой элемент $s\in S$ является многочленом от $s_1,\ldots,s_n$ с коэффициентами в поле $F$.
\end{definition}

Если $S$ -- коммутативная $F$-алгебра с образующими $s_1,\ldots,s_n$, то мы можем рассмотреть гомоморфизм $F$-алгебр
\[
\phi\colon F[x_1,\ldots, x_n]\to S\quad\text{ по правилу }f(x_1,\ldots,x_n) \mapsto f(s_1,\ldots,s_n)
\]
Тогда это будет сюръективный гомоморфизм, а значит по теореме о гомоморфизме $S$ изоморфно фактор алгебре $F[x_1,\ldots, x_n]/I$, где $I = \ker \phi$.
Таким образом изучать конечно порожденные коммутативные алгебры -- это все равно, что изучать факторы кольца многочленов с коэффициентами в поле по некоторому идеалу.
А для эффективной работы с таким фактором, нам надо уметь эффективно работать с идеалами в кольце многочленов.
А это мы умеем делать с помощью базисов Грёбнера.

\paragraph{Базисы Грёбнера и конечно порожденные алгебры над полем}

Пусть у нас есть некоторое поле $F$, идеал $I\subseteq F[x_1,\ldots,x_n]$ и мы хотим понять, как устроено фактор кольцо $F[x_1,\ldots,x_n]/I$.
Прежде всего, мы хотим понять, как устроено это фактор кольцо как множество.
Оказывается, что надо сделать следующее.
Давайте фиксируем некоторый лексикографический порядок $\operatorname{Lex}$ на мономах.%
\footnote{На самом деле там есть еще много порядков, которые можно использовать кроме лексикографических.}
После этого посчитаем базис Грёбнера $G$ идеала $I$.
Пусть $R\subseteq F[x_1,\ldots,x_n]$ -- множество остатков относительно $G$.
Тогда оказывается, что $F[x_1,\ldots,x_n]/I$ можно отождествить с $R$.
А именно для каждого остатка $r\in R$ ставится в соответствие смежный класс $r+I$.
Оказывается, что это соответствие будет биективным.
На языке остатков операция сложения остается сложением остатков.
А вот операция умножения выполняется так.
Пусть $f,g\in R$ -- два остатка, перемножим их как многочлены, а потом отредуцируем к остатку $fg \stackrel{G}{\rightsquigarrow}r$.
Тогда $r$ и будет произведением $f$ и $g$.

Обратим внимание, что $f,g\in F[x_1,\ldots, x_n]$ задают два элемента в факторе $F[x_1,\ldots,x_n]/I$.
Эти элементы равны тогда и только тогда, когда $f\sim g$ в смысле отношения эквивалентности заданного идеалом $I$.
Это значит, что $f\sim g$ тогда и только тогда, когда $f-g\in I$.
А так как $G$ -- базис Грёбнера -- это равносильно тому, что $f - g \stackrel{G}{\rightsquigarrow}0$.
И можно показать, что это равносильно тому, что $f$ и $g$ имеют одинаковые остатки относительно $G$.
В частности проверка лежит ли элемент $f$ в идеале $I$ означает, задает ли элемент $f$ в фактор кольце нулевой элемент или нет.