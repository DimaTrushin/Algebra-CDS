\ProvidesFile{lecture08.tex}[Lecture 8]


\section{Коды исправляющие ошибки}

Предположим, что у нас есть конечный алфавит $F$ из $q$ символов и мы представляем информацию в виде последовательностей символов из $F$.
Теперь представим себе, что мы хотим передавать последовательность $a_1,a_2,\ldots,a_n,\ldots\in F$ по каналу связи, но при этом канал связи имеет некоторые помехи.
В результате воздействия этих помех получателю приходит не оригинальная последовательность, а испорченная.
Мы будем предполагать, что канал связи может лишь испортить какие-то отдельные символы нашей последовательности.
В результате возникает задача -- как восстановить исходную последовательность символов, зная испорченную.

Давайте я опишу общую концепцию, в рамках которой решается подобная задача.
Прежде всего, пусть у нас передаваемая информация хранится порциями по $k$ символов.
То есть хранимая информация -- это элементы $F^k$.
Теперь нам нужно придумать некоторое инъективное отображение $\phi\colon F^k \to F^n$, которое и называется кодированием.
Смысл этого отображения в том, что мы как-бы добавляем к исходной последовательности дополнительную контрольную информацию.
После этого полученное слово длины $n$ мы уже будем передавать по сети и именно это более длинное слово может как-то испортиться.
После получения испорченного слова, нам нужна некоторая процедура $\psi\colon F^n \to F^k$, которая называется декодированием и которая должна восстановить исходное сообщение если это возможно.
На картинке процесс можно изобразить так
\[
\xymatrix@R=5pt{
  {\text{Сообщение}}\ar@{|->}[r]&{\text{Закодированное}}\ar@{|->}[r]&{\text{Канал}}\ar@{|->}[r]&{\text{Переданное}}\ar@{|->}[r]&{\text{Раскодированное}}\\
  {a\in F^k}&{\phi(a)\in F^n}&{\text{ШУМ}}\ar@<-1ex>[u]\ar[u]\ar@<1ex>[u]&{\tilde{c}\in F^n}&{\psi(\tilde{c})\in F^k}\\
}
\]
Самая важная задача -- сделать так, чтобы $\psi(\tilde{c})$ совпало с исходным сообщением $a\in F^k$.
Конечно же, если канал может как угодно портить сообщение, то мы ничего не можем гарантировать.
Однако, можно добиться некоторых гарантий, если мы знаем, что шум в канале не абы какой.
Частым предположением является предположение о том, что шум может испортить не более $e$ символов в передаваемой последовательности.
То есть процент ошибок в канале равен $e / n$.

\paragraph{Пример}

Давайте начнем с очень простого, но понятного примера.
Пусть мы хотим передавать последовательности $F^k$ по каналу, который может максимум испортить один элемент.
Тогда давайте рассмотрим следующую процедуру кодирования $\phi\colon F^k \to F^{3k}$ по правилу $w\mapsto w, w, w$, то есть мы просто повторяем слово три раза.
Тогда понятно как восстановить одну ошибку.
Если нам пришла последовательность $w_1, w_2, w_3$, то два слова из трех должны быть одинаковыми, вот это слово мы и берем.
Главным недостатком такого подхода является увеличение количества передаваемой информации.
Во-первых, это не эффективно в плане скорости.
Во-вторых, это не эффективно, так как количество ошибок в канале может расти в зависимости от размера передаваемого сообщения и одна ошибка на слове длины $k$ может легко превратиться в три ошибки на слове длины $3k$, а значит и кодирование может оказаться бесполезным.
Тем не менее, это хороший концепт, показывающий как именно может происходить процедура кодирования и раскодирования.
В дальнейшем мы постараемся построить более экономные процедуры кодирования и декодирования.

\subsection{Общие замечания}

\begin{definition}
Пусть $F$ -- конечный алфавит и $a, b\in F^n$ -- два слова длины $n$ в этом алфавите.
Тогда расстояние Хэмминга между словами $a$ и $b$ это:
\[
\rho(a, b) = |\{i \mid a_i \neq b_i\}|\quad\text{количество отличающихся символов}
\]
\end{definition}

Обратите внимание, если мы передавали по сети строчку $a\in F^n$, а нам пришла на вход строчка $b\in F^n$, то $\rho(a, b)$ -- это в точности количество произошедших ошибок.

\begin{definition}
Пусть $F$ -- конечный алфавит.
Тогда при $k < n$ инъективное отображение $\phi\colon F^k \to F^n$ называется кодированием, а образ $C = \Im \phi$ называется кодом.
\end{definition}

Как мы увидим ниже, многие свойства кодирования зависят именно от свойств кода $C$.
Вопросы нахождения функции кодирования и раскодирования больше сопряжены с эффективностью их вычисления, просто потому что при наличии множества $C$ кодирование может осуществляться любым вложением $\phi \colon F^k \to C$, а декодирование в худшем случае можно делать полным перебором, если оно возможно.

\begin{definition}
Пусть $F$ -- конечный алфавит и $C\subseteq F^n$ -- некоторое подмножество.
Тогда код $C$ исправляет $t$ ошибок, если для любого $x\in F^n$ существует не более одного $c\in C$ такого, что $\rho(x, c) \leqslant t$.
\end{definition}

Давайте обсудим последнее определение.
\begin{itemize}
\item
Оно говорит, что если у вас есть произвольное слово $x\in F^n$, то меняя в нем не более $t$ символов, мы максимум найдем одно слово из $C$.
Потому, если у нас слово $x$ возникло в результате не более $t$ изменений в слове $c\in C$, то слово $c$ будет единственным словом из $C$ таким, что $\rho(x, c) \leqslant t$.
А значит, мы точно можем восстановить слово $c$.

\item Кроме того, может случиться такая ситуация, что $\rho(x, c) \leqslant t$ не выполняется ни для какого слова $c$.
Это означает, что при передаче по каналу произошло больше $t$ ошибок.
Таким образом, по-хорошему процедура декодирования $\psi$ должна не просто выплевывать раскодированную последовательность, а еще иногда сообщать, что мы получили недопустимое слово.

\item Если нам дана процедура кодирования $\phi\colon F^k \to F^n$, то мы можем говорить, что $\phi$ допускает исправление $t$ ошибок, если код $\Im \phi$ допускает исправление $t$ ошибок.
\end{itemize}

Теперь я хочу получить численные характеристики кода, которые бы описывали его возможности по восстановлению ошибок.

\begin{definition}
Пусть $F$ -- конечный алфавит и $C\subseteq F^n$ -- некоторое подмножество.
Тогда минимальное расстояние кода $C$ это
\[
d_C = \min_{c\neq c' \in C} \rho(c, c')
\]
\end{definition}

\begin{definition}
Пусть $F$ -- конечный алфавит.
Тогда шаром радиуса $r\in \mathbb Z$ с центром в $x\in F^n$ называется множество
\[
B_r(x) = \{y\in F^n \mid \rho(x, y) \leqslant r\}
\]
\end{definition}

\begin{claim}
Пусть $F$ -- конечный алфавит и $C\subseteq F^n$ -- некоторое подмножество.
Тогда эквивалентно
\begin{enumerate}
\item Код $C$ исправляет $t$ ошибок.

\item Для любых $x\neq y\in C$ верно $B_t(x)\cap B_t(y) = \varnothing$.

\item $d_C \geqslant 2t + 1$.
\end{enumerate}
\end{claim}
\begin{proof}
(1)$\Rightarrow$(2).
Предположим противное, что существует $z\in B_t(x) \cap B_t(y)$ для некоторых $x,y\in C$.
Но тогда для слова $z$ найдутся два слова $x,y\in C$ на расстоянии не более $t$, что противоречит тому, что код исправляет $t$ ошибок.

(2)$\Rightarrow$(3).
Предположим противное, то есть $d_C \leqslant 2t$.
Тогда найдутся два слова $x,y\in C$ такие, что $\rho(x, y) \leqslant 2t$.
Это значит, что слово $y$ отличается от слова $x$ в $s\leqslant 2t$ позициях.
 Давайте изменим в $y$ $s/2$ позиций на значения из $x$.
Тогда получим слово $z$, которое отличается от $x$ и от $y$ не более чем на $s/2 \leqslant t$ позиций.
 А это значит, что мы нашли слово в пересечении $B_t(x) \cap B_t(y)$, противоречие.

(3)$\Rightarrow$(1).
Предположим противное, тогда для какого-то слова $z$ найдутся два слова $x,y\in C$ такие, что $\rho(x,z) \leqslant t$ и $\rho(y,z) \leqslant t$.
Но тогда по неравенству треугольника для $\rho$ имеем $\rho(x,y) \leqslant \rho(x,z)+ \rho (z,y) \leqslant 2t$, противоречие.
\end{proof}

\paragraph{Пример}

Давайте рассмотрим следующий код $C = \{(a, a, \ldots, a)\mid a\in F\}\subseteq F^n$.
Очевидно, что минимальное расстояние кода $d_C = n$.
А значит, код исправляет $t = [(n-1)/2]$ ошибок.

\subsection{Линейная алгебра}

Хороший вопрос -- как эффективно задавать множество $C\subseteq F^n$, чтобы не работать с ним перебором.
Оказывается один из подходов -- воспользоваться достижениями линейной алгебры над конечными полями.
Такой подход приводит к так называемым линейным кодам.
Давайте поговорим о них.

\begin{definition}
Пусть $F = \mathbb F_q$ -- конечное поле из $q = p^n$ элементов.
Тогда подмножество $C\subseteq \mathbb F_q^n$ называется линейным кодом, если $C$ является подпространством над полем $\mathbb F_q$.
Если при этом размерность подпространства равна $k$, то говорят, что $C$ -- это $(n, k)$-код.
\end{definition}

Если у нас есть линейный код $C\subseteq \mathbb F_q^n$, то легко построить кодирующую функцию.
Пусть размер кода $C$ равен $k$, тогда мы можем найти в $C$ базис $c_1,\ldots,c_k$.
Тогда кодирующее отображение $\phi\colon \mathbb F_q^k \to \mathbb F_q^n$ задано по правилу $x \mapsto Ax$, где у матрицы $A$ по столбцам поставлены векторы $c_1,\ldots,c_k$.

\begin{definition}
Пусть $x\in \mathbb F_q^n$, тогда нормой вектора $x$ будет называться его расстояние до нулевого вектора в норме Хэмминга, то есть
\[
\|x\| = \rho(x, 0) = |\{i \mid x_i \neq 0\}|
\]
\end{definition}

\begin{claim}
Пусть $C\subseteq \mathbb F_q^n$ -- линейный код.
Тогда $d_C = \min_{0\neq c \in C}\|c\|$.
\end{claim}
\begin{proof}
Пусть $x,y\in \mathbb F_q^n$ два произвольных слова, обратим внимание, что $\rho(x,y) = \rho(x - y, 0) = \|x - y\|$, потому что количество разных координат у $x$ и $y$ -- это в точности количество ненулевых координат у разности $x-y$.
А тогда
\[
d_C = \min_{c \neq c'\in C}\rho(c,c') = \min_{c\neq c' \in C}\|c-c'\| = \min_{0\neq c\in C}\|c\|
\]
Последнее равенство выполнено в силу того, что $C$ является подпространством, потому если $c,c'$ пробегают все $C$, то и их разность $c-c'$ пробегают все $C$.
\end{proof}

\begin{definition}
Пусть $C\subseteq F_q^n$ -- некоторый линейный код размерности $k$ и пусть $H\in \MatrixFDim{\mathbb F_q}{n-k}{n}$ такая, что $\rk H = n - k$ и $C = \{y\in \mathbb F_q^n \mid Hy = 0\}$.
Тогда матрица $H$ называется проверочной матрицей для кода $C$.
\end{definition}

\paragraph{Пример}

Если взять код $C = \{(a, \ldots, a) \mid a\in \mathbb F_q\}\subseteq \mathbb F_q^n$, то это $(n, 1)$-код с проверочной матрицей (все незаполненные места -- нулевые)
\[
\begin{pmatrix}
{1}&{}&{}&{-1}\\
{}&{\ddots}&{}&{\vdots}\\
{}&{}&{1}&{-1}\\
\end{pmatrix}
\]

\begin{claim}
Пусть $C = \{y\in \mathbb F_q^n\mid Hy = 0\}$.
Тогда эквивалентно
\begin{enumerate}
\item $d_C \geqslant s + 1$

\item Любые $s$ столбцов матрицы $H$ линейно независимы.
\end{enumerate}
\end{claim}
\begin{proof}
Давайте рассмотрим $y\in \mathbb F_q^n$ у которого не более $s$ ненулевых координат.
Тогда $Hy$ равно линейной комбинации не более $s$ столбцов матрицы $H$.
Таким образом условие, что любые $s$ столбцов $H$ линейно независимы означает, что системе $Hy = 0$ не удовлетворяет ни один вектор у которого не более $s$ ненулевых координат.
Или другими словами, условие (2) эквивалентно тому, что в $C$ лежат векторы у которых хотя бы $s+1$ ненулевая координата.
А это условие в точности равносильно тому, что $d_C \leqslant s+1$.
\end{proof}

\subsection{Коды Хэмминга}

Давайте рассмотрим случай бинарного алфавита $F = \mathbb F_2 = \{0, 1\}$.
Фиксируем число $k$ и составим матрицу $H$ следующим образом.
По столбцам матрицы $H_k$ поставим все ненулевые векторы из $\mathbb F_2^k$.
Тогда получится матрица размера $k$ на $2^k - 1$.
Пусть $n = 2^k - 1$.
Тогда можно определить код с такой проверочной матрицей
\[
C_k = \{y\in \mathbb F_2^n \mid H_ky = 0\}\subseteq \mathbb F_2^n
\]
Полученный код называется бинарным кодом Хэмминга.
Это $(2^k -1, 2^k - k - 1)$-код.
Обратите внимание, что закодированное слово имеет длину $n = 2^k - 1$, при этом количество проверочной информации равно $k \approx \log_2 n$.
Чтобы построить кодирующее отображение $\phi\colon \mathbb F_2^{2^k - k - 1}\to \mathbb F_2^{2^k - 1}$ нам надо найти ФСР для системы $H_k y = 0$ и составить их по столбцам в матрицу $A_k$.
Тогда $\phi(x) = A_kx$.

Если мы хотим раскодировать кодовые слова, то для начала давайте посчитаем минимальное расстояние кода.
Мы видим, что в матрице $H_k$ любые $2$ вектора линейно независимы и существует $3$ линейно зависимых вектора.
Значит $d_{C_k} = 3$.
То есть количество исправляемых ошибок будет $e = [(d_{C_k} - 1)/2] = 1$.
Если ошибка произошла в $i$-ом символе, это значит, что кодовое слово $y$ изменилось на $y + e_i$.
А тогда $H_k(y + e_i) = H_ke_i$ -- $i$-ый столбец матрицы $H_k$.
Так как все столбцы уникальны это дает возможность однозначно узнать номер $i$.
Более того, давайте смотреть на столбцы в $H_k$ как на представление целых чисел в двоичной системе, то есть вектору $(a_1,\ldots,a_k)$ ставим в соответствие число $a_1 + a_2 2 +\ldots + a_k 2^{k-1}$.
Расставим столбцы в порядке увеличения чисел.
Тогда $H_k e_i$ будет двоичным разложением числа $i$.

Двоичные коды Хэмминга интересны тем, что они дают плотную упаковку шаров радиуса $1$ в пространстве $\mathbb F_2^{2^k - 1}$, то есть шары радиуса $1$ с центрами в кодовых словах кода Хэмминга покрывают все пространство без пустот и пересечений.
У этого кода очень простая функция кодирования и декодирования.
Единственным его недостатком является малое число исправляемых ошибок.

\subsection{Коды Рида-Соломона}

Оказывается, что для линейных кодов можно оценить сверху минимальное кодовое расстояние, через размерность пространства.
Это утверждение носит название <<Неравенство Синглтона>>.

\begin{claim}
[Неравенство Синглтона]
Пусть $C\subseteq \mathbb F_q^n$ -- некоторый линейный $(n,k)$-код.
Тогда $d_C \leqslant n - k + 1$.
\end{claim}
\begin{proof}
Пусть $q_1,\ldots,q_{k-1}\in \mathbb F_q$ -- фиксированные числа.
Рассмотрим подмножество
\[
D_{q_1,\ldots, q_{k-1}} = \{y\in \mathbb F_q^n \mid y = (q_1,\ldots, q_{k-1}, *,\ldots, *)\}\subseteq \mathbb F_q^n
\]
То есть мы зафиксировали первые $k-1$ координату в последовательности, а остальные координаты могут быть какие угодно.
Всего таких подмножеств $q^{k-1}$ по количеству последовательностей $q_1,\ldots,q_{k-1}$.
Более того, подмножества $D_{q_1,\ldots, q_{k-1}}$ образуют разбиение $\mathbb F_q^n$ на непересекающиеся множества.
Теперь посмотрим на подпространство $C$ в нем $q^k$ элементов, так как размерность равна $k$.
А значит по принципу Дирихле найдется хотя бы одна <<клетка>> $D_{q_1,\ldots,q_{k-1}}$ содержащая хотя бы два <<кролика>> -- точки из $C$.
То есть хотя бы два элемента из $C$ совпадают в первых $k-1$ позиции.
А это значит, что расстояние между этими точками не больше $n-k+1$.
А значит и минимальное расстояние между точками $C$ не больше $n - k + 1$.
\end{proof}

Следующий пример кодов -- коды Рида-Соломона.
На этих кодах достигается равенство в неравенстве Синглтона.
Пусть у нас $F = \mathbb F_q$ и число $n$ выбрано так, что $n \leqslant q$.
Теперь для любого $k < n$ построим кодирующую функцию $\phi\colon \mathbb F_q^k \to \mathbb F_q^n$.
Обратите внимание, что для построения такого кода нам требуется достаточно большой алфавит (условие $n \leqslant q$).

Мы можем отождествить $\mathbb F_q^k$ с многочленами степени меньше $k$, а именно
\[
a = (a_0,\ldots,a_{k-1})\in \mathbb F_q^k \mapsto f_a = a_0 + a_1 x + \ldots + a_{k-1}x^{k-1}
\]
Выберем $n$ различных точек $x_1,\ldots, x_n$ поля $\mathbb F_q$.%
\footnote{В этом месте мы пользуемся неравенством $n \leqslant q$.}
Теперь построим отображение $\phi \colon \mathbb F_q^k \to \mathbb F_q^n$ по правилу $a \mapsto f_a\mapsto (f_a(x_1),\ldots,f_a(x_n))$.
Давайте оценим минимальное кодовое расстояние.
Если у нас есть два разных многочлена степени меньше $k$, то они могут иметь не более $k-1$ общего значения.
Значит разные последовательности $\phi(a)$ и $\phi(b)$ для разных $a,b\in \mathbb F_q^k$ могут иметь не более $k-1$ одинаковых координат.
То есть расстояние между ними будет не меньше $n - k + 1$.
С другой стороны, по неравенству Синглтона минимальное расстояние кода не больше $n - k + 1$.
Значит $d_C = n - k + 1$.
А значит, количество исправляемых кодом ошибок равно $e = [(n - k)/2]$.

Важно, что коды Рида-Соломона допускают эффективную процедуру декодирования.
Предположим, что нам на вход пришла последовательность $b_1,\ldots, b_n\in \mathbb F_q$ и мы хотим восстановить многочлен $f_a$ степени меньше $k$.
Давайте я в начале опишу процедуру, как восстановить соответствующий многочлен, а уже потом обсудим, почему эта процедура корректная.
Рассмотрим два многочлена $D, Q\in \mathbb F_q[x]$ следующего вида
\[
D = x^e + d_{e-1}x + \ldots + d_0\quad\text{и}\quad
Q = q_{e+k - 1}x^{e+k-1} + \ldots + q_0
\]
то есть многочлен $D$ имеет степень $e$ и старший коэффициент $1$, а многочлен $Q$ имеет степень меньше $e + k$.
Оба многочлена берутся с неопределенными коэффициентами.
Теперь надо составить систему уравнений
\[
Q(x_1) = b_1 D(x_1),\ldots,Q(x_n) = b_n D(x_n)
\]
Это неоднородная система линейных уравнений на коэффициенты $Q$ и $D$.
Если последовательность $b_1,\ldots, b_n$ получена из последовательности кода Рида-Соломона в результате не более $e$ ошибок, то оказывается, что такая система всегда имеет хотя бы одно решение.
Пусть $Q$ и $D$ -- какое-то решение.
Оказывается, что в этом случае $D$ делит $Q$ и при этом $Q/D$ -- это искомый многочлен $f_a$.
Таким образом мы получаем алгоритм декодирования последовательности $b$ в последовательность $a$.
Надо лишь объяснить, почему эта процедура действительно работает и дает, что надо.

Пусть $a\in \mathbb F_q^k$ -- некоторое слово, $c =\phi(a)= (f_a(x_1),\ldots,f_a(x_n))$ -- соответствующее кодовое слово и $b = (b_1,\ldots,b_n)$ -- испорченное кодовое слово, где произошло не более $e$ ошибок.
Пусть ошибки произошли в позициях $i_1,\ldots, i_e$.
Тогда мы можем рассмотреть многочлен $D = (x - x_{i_1})\ldots ( x - x_{i_e})$.
Тогда последовательность $d = (D(x_1),\ldots, D(x_n))$ имеет нули ровно в тех позициях, где произошли ошибки.
Тогда последовательность $b d$ совпадает с последовательностью $c d$, потому что $b$ и $c$ отличаются в позициях $i_1,\ldots, i_e$, где $d$ равна нулю.
Но что означает равенство $bd = cd$.
Давайте его распишем.
\[
f_a(x_1) D(x_1) = b_1 D(x_1),\ldots,f_a(x_1) D(x_n) = b_n D(x_n)
\]
Но тогда обозначим $f_a D$ через $Q$.
Мы получим многочлен степени строго меньше $k + e$.
Последнее означает, что если последовательность $b_1,\ldots, b_n$ получена из кодового слова $c$ не более чем $e$ ошибками, то система на многочлены $Q$ и $D$ имеет решение.

Теперь пусть
\[
Q(x_1) = b_1 D(x_1),\ldots,Q(x_n) = b_n D(x_n)
\]
и $Q$ и $D$ -- произвольное решение.
При этом отметим, что степень $Q$ меньше $k + e$, а степень $D$ равна $e$ и его старший коэффициент -- $1$.
Пусть $a$ -- кодовое слово, из которого получилась последовательность $b$.
Рассмотрим последовательность
\[
f_a(x_1) D(x_1),\ldots, f_a(x_n)D(x_n)
\]
Эта последовательность имеет не менее $n - e \geqslant k + e$ общих элементов с последовательностью 
\[
b_1D(x_1),\ldots,b_nD(x_n)
\]
А значит не менее $k + e$ общих элементов с последовательностью
\[
Q(x_1),\ldots, Q(x_n)
\]
То есть многочлены $f_aD$ и $Q$ имеющие степень меньше $k + e$ имеют хотя бы $k + e$ общих точек.
А это значит, что они обязаны совпадать.
Отсюда $f_a D = Q$ и мы доказали, что требовалось.
