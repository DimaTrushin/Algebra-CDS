\ProvidesFile{lecture07.tex}[Lecture 7]


\begin{proof}
В начале проверим, что отображение является гомоморфизмом колец.
Мы должны показать, что оно сохраняет сложение, умножение и единицу кольца.
Это делается прямым вычислением и я позволю себе пропустить это вычисление.

Теперь покажем, что отображение инъективно.
По утверждению~\ref{claim::RingHomProp} достаточно проверить, что ядро $\Phi$ состоит только из нуля.
Предположим $h\in \ker \Phi$.
Это значит, что $h = 0 \pmod f$ и $h = 0 \pmod g$, то есть $h$ делится на $f$ и $g$.
Так как $f$ и $g$ взаимно просто, последнее означает, что $h$ делится на $fg$.
Но $\deg h < \deg fg$.
Значит такое может быть только если $h = 0$.

Теперь мы хотим показать сюръективность.
Так как $F$ -- поле, мы можем рассматривать $F[x]/(fg)$и $F[x]/(f)\times F[x]/(g)$ как векторные пространства над $F$.
Кроме того, $\Phi$ является линейным отображением потому что по определению $\Phi(\lambda) = \lambda$ и значит $\Phi(\lambda f) = \Phi(\lambda)\Phi(f) = \lambda \Phi(f)$.
Так как $\Phi$ инъективно, то для доказательства сюръективности достаточно показать, что оба пространства имеют одинаковую размерность.
Ясно, что размерность $\dim_F F[x]/(fg) = \deg(fg)$.
С другой стороны, с точки зрения векторных пространств $F[x]/(f)\times F[x]/(g)$ является прямой суммой $F[x]/(f)$ и $F[x]/(g)$.
А значит размерность $F[x]/(f)\times F[x]/(g)$ совпадает с $\deg f + \deg g$.
Теперь результат следует из утверждения~\ref{claim::Degree}.
\end{proof}

\begin{claim}
\label{claim::PolyRemIdeals}
Пусть $F$ -- поле, $f\in F[x]$ -- многочлен и $I\subseteq F[x]/ (f)$ -- некоторый идеал.
Тогда существует многочлен $g\in F[x]$ делящий $f$ такой, что $I = (g) = \{g h\!\!\mod{f}\mid h\in F[x]\}$.
\end{claim}
\begin{proof}
Случай $f = 0$ разобран в утверждении~\ref{claim::PolyIdeals}.
Теперь мы предположим, что $f \neq 0$.
Если $I$ состоит только из нулей, то $I = (f)$ и все доказано.

Пусть $h\in I$ -- ненулевой многочлен минимально возможной степени в $I$.
Тогда для любого $g\in I$, разделим $g$ на $h$ с остатком и получим $g = qh + r$ где $\deg r < \deg h$.
Также $r = g - qh\in I$.
Так как $h$ был ненулевым многочленом минимально возможной степени в $I$, такое может быть только если $r = 0$.
Значит $h$ делит любой $g\in I$.
последнее означает, что $I = (h)$.

Теперь мы должны показать, что $h$ делит $f$.
Давайте разделим $f$ на $h$ с остатком, получим $f = qh + r$, где $\deg r < \deg h$.
Это означает, что $r = -qh$ в кольце $F[x]/(f)$.
В частности $r\in I$ и имеет степень меньше, чем $h$.
Такое возможно только если $r = 0$, что завершает доказательство.
\end{proof}

\section{Поля}

\subsection{Характеристика}

\begin{definition}
Пусть $F$ -- поле.
Характеристика поля $F$ -- это такое минимальное натуральное число $p$, что
\[
\underbrace{1+\ldots + 1}_p = 0
\]
Если такого числа $p$ нет, то говорят, что характеристика равна нулю.
Характеристика поля $F$ обозначается через $\chr F$.
\end{definition}

Давайте я введу удобное обозначение, если мы хотим сложить элемент $x\in F$ $n$ раз сам с собой, где $n\in \mathbb N$, то мы можем обозначить эту сумму следующим образом
\[
n x = \underbrace{x+\ldots + x}_n
\]
То есть мы можем говорить об умножении на натуральное число в поле.
В частности, характеристика поля $F$ -- это такое наименьшее положительное число $p$ , что $p \cdot 1 = 0$ если оно существует, и равна нулю если такого числа нет.

\begin{examples}
\begin{enumerate}
\item Если $F = \mathbb Q$, тогда сумма $1+\ldots + 1$ никогда не равна нулю.
Значит, $\chr \mathbb Q = 0$.

\item Если $F = \mathbb Z_p$, где $p$ -- простое число, то $p$ является наименьшим положительным числом таким, что $1 + \ldots + 1 = p \cdot 1 = 0$ в $\mathbb Z_p$.
Значит, $\chr \mathbb Z_p = p$.
\end{enumerate}
\end{examples}

\begin{claim}
Если $F$ -- поле, то $\chr F$ либо равна нулю, либо простое число.
\end{claim}
\begin{proof}
Предположим характеристика не ноль.
Пусть $\chr F = p = n m$ не простое.
Тогда
\[
0 = \underbrace{1+\ldots + 1}_{nm} = (\underbrace{1+\ldots + 1}_{n})(\underbrace{1+\ldots + 1}_{m}) = (n \cdot 1) (m \cdot 1)
\]
Кроме того, числа $n\cdot 1$ и $m\cdot 1$ не равны нулю, так как $p = nm$ было минимальным зануляющим единицу по определению.
С другой стороны, мы получили, что произведение $n\cdot 1$ и $m \cdot 1$ -- ноль.
Это противоречит отсутствию ненулевых делителей нуля в поле $F$.
\end{proof}

\begin{remark}
Пусть $F$ -- поле.
У нас есть единственный гомоморфизм колец $\phi\colon \mathbb Z\to F$, который действует по правилу $n \mapsto n \cdot 1$.
Ядро этого гомоморфизма является идеалом в $\mathbb Z$.
Каждый идеал в кольце целых чисел имеет вид $(p)$ для некоторого неотрицательного числа $p$ (см.~утверждение~\ref{claim::ZIdeals}).
В этом случае $p$ есть ни что иное как характеристика поля $F$.
Это объясняет связь характеристики с идеалами в кольце целых чисел.
\end{remark}

\begin{claim}
\label{claim::FieldIdeals}
Пусть $R$ -- коммутативное кольцо.
Тогда $R$ является полем тогда и только тогда, когда в нем только два идеала $0$ и $R$.
\end{claim}
\begin{proof}
Предположим, что $R$ является полем и $I\subseteq R$ -- некоторый идеал.
Если $I = 0$, то доказывать нечего.
Мы можем предположить, что $I$ содержит ненулевой элемент и должны доказать, что в этом случае $I = R$.
Пусть $x\in I$ -- ненулевой элемент.
Так как $R$ -- поле, то существует $x^{-1}\in R$.
Так как $I$ идеал, то $1 =x^{-1}x\in I$.
А значит, для любого $y\in R$, элемент $y = y 1 \in I$.

Теперь предположим, что $R$ содержит только два идеала $0$ и $R$.
Пусть $x\in R$ -- ненулевой элемент.
Тогда множество $I = \{rx \mid r\in R\}$ является идеалом в $R$.%
\footnote{Здесь важно, что $R$ коммутативно.}
Так как у нас только два идеала $0$ и $R$, то $I$ должен быть одним из них.
Так как $I$ содержит ненулевой элемент, то он обязан совпасть с $R$.
В частности $1\in I$, то есть $1 = rx$ для некоторого $r\in R$.
А так как $R$ коммутативно, то $x$ обратим и все доказано.
\end{proof}

Предыдущее утверждение показывает, что мы должны изучать поля в терминах элементов, так как идеалы не дают о полях никакой информации.
С другой стороны, идеалы как бы измеряют разницу между произвольным кольцом и полем.
Чем меньше идеалов в кольце, тем ближе оно к полю.

\begin{claim}
\label{claim::ZpField}
Кольцо $\mathbb Z_n$ является полем тогда и только тогда, когда $n$ является простым.
\end{claim}
\begin{proof}
Мы уже обсуждали это утверждение и показывали напрямую, что любой ненулевой элемент такого кольца обратим.
Давайте теперь обсудим, как это можно показать через идеалы.
В силу утверждения~\ref{claim::ZnIdeals} мы знаем, что в $\mathbb Z_n$ только два идеала лишь в случае, когда $n$ простое.
А значит по предыдущему утверждению~\ref{claim::FieldIdeals} это равносильно тому, что $\mathbb Z_n$ -- поле.
\end{proof}

\begin{claim}
\label{claim::PrimeSubfieldCharP}
Предположим $F$ -- поле простой характеристики $p$.
Тогда $F$ содержит $\mathbb Z_p$ как подполе.
\end{claim}
\begin{proof}
Поле $F$ содержитт единицу $1\in F$.
Давайте рассмотрим множество элементов $\{0, 1, 2\cdot 1, \ldots , (p-1)\cdot 1\}\subseteq F$.
Тогда есть очевидная биекция между этим множеством и полем $\mathbb Z_p$.
Мы должны показать, что эта биекция является изоморфизмом.
Для этого нам достаточно показать
\[
n\cdot 1 + m \cdot 1 = (n+m\!\!\mod p)\cdot 1\quad\text{и}\quad
(n\cdot 1)(m\cdot 1) = (mn\!\!\mod p)\cdot 1
\]
Действительно, если $m+n = qp + r$, тогда
\[
n\cdot 1 + m \cdot 1 = (m+n)\cdot 1 = (qp) \cdot 1 + r\cdot 1 = q(p \cdot 1) + r\cdot 1 = r\cdot 1
\]
Второе утверждение показывается аналогично.
\end{proof}

\subsection{Расширения полей}

Предположим $F$ -- некоторое поле и $K$ -- другое поле содержащее $F$ в качестве подполя.
В этом случае мы будем говорить, что $K$ -- расширение поля $F$.
В этом случае $K$ можно рассматривать как векторное пространство над полем $F$.
А значит, можно говорить о размерности $K$ над $F$
Размерность поля $K$ над полем $F$ называется степенью поля $K$ над $F$ и обозначается $[K:F] = \dim_F K$.

Я хочу обсудить методы по построению новых полей.
Следующее утверждение объясняет один из наиболее удобных способов.

\begin{claim}
\label{claim::PolyQuotField}
Пусть $F$ -- поле, $f\in F[x]\setminus F$ -- некоторый многочлен.
Тогда кольцо $F[x]/(f)$ является полем тогда и только тогда, когда $f$ неприводимый многочлен.
\end{claim}
\begin{proof}
Давайте дадим прямой доказательство через элементы.
В начале предположим, что $f$ приводим.
Тогда $f = gh$, где $\deg g < \deg f$ и $\deg h < \deg f$.
Тогда $h\neq 0$ в $F[x]/(f)$ так же как $g\neq 0$ в $F[x]/(f)$.
Но $gh = f = 0$ в $F[x]/(f)$.
Значит, $g$ и $h$ являются ненулевыми делителями нуля.
Так как делители нуля необратимы, то это противоречит определению поля.

Если $f$ неприводим, мы должны показать, что любой ненулевой многочлен $g\in F[x]/(f)$ обратим.
Так как $\deg g < \deg f$ и $g\neq 0$, $f$ и $g$ взаимно просты.
Значит, $1 = (g, f)$.
По утверждению~\ref{claim::PolyGCD} пункт~(1) следует, что $1 = ug + vf$ для некоторых $u,v\in F[x]$.
Но последнее означает, что $1 = ug\pmod f$ а значит, $u = g^{-1}$ в $F[x]/(f)$.%
\footnote{Мы могли бы дать другое доказательство с помощью идеалов.
По утверждению~\ref{claim::PolyRemIdeals} мы знаем, что в кольце остатков только два идеала лишь в случае неприводимого многочлена.
Что по утверждению~\ref{claim::FieldIdeals} равносильно тому, что кольцо остатков является полем.}
\end{proof}

\begin{remarks}
\begin{itemize}
\item Давайте подчеркнем, что элемент $x\in F[x]/(p)$ является корнем многочлена $p$ в поле $F[x]/(p)$.
Действительно, $p(x) = 0$ в $F[x]/(p)$ по определению.

\item Давайте рассмотрим поле вещественных чисел $\mathbb R$.
Тогда многочлен $x^2 + 1\in \mathbb R[x]$ неприводим.
Значит $\mathbb R[x]/(x^2 +1)$ является полем и элемент $x$ становится корнем многочлена $x^2 + 1$.
Явно элементы поля $\mathbb R[x]/(x^2 + 1)$ имеют вид $a + bx$, где $a, b\in \mathbb R$ и мы знаем, что $x^2 = -1$.
Значит это привычная нам модель комплексных чисел, где мы $i$ обозначили за $x$.

\item Если $\deg f = n$, тогда элементы $1, x, \ldots, x^{n-1}$ образуют базис поля $F[x]/(f)$ над полем $F$.
Другими словами, $\dim_F F[x]/(f) = \deg f$.
\end{itemize}
\end{remarks}

\paragraph{Расширение с помощью корня}

Давайте обсудим, что нам дает доказанное утвержение в плане построения полей.
Предположим $F$ -- поле и $f\in F[x]$ -- некоторый многочлен.
Теперь я хочу построить поле $L$ содержащее $F$ и некоторый элемент $\alpha \in L$ такой, что $f(\alpha) = 0$.
То есть по простому, я хочу добавить к полю $F$ корень многочлена $f$.
Если нам повезло и $F$ уже содержит корень, то нам нечего делать.
Теперь обсудим, что делать, если желаемого корня в поле $F$ нет.
Мы знаем (см.~утверждение~\ref{claim::PolyUFD}), что $f$ раскладывается в произведение неприводимых многочленов, то есть $f = p_1\ldots p_n$ ($p_i$ -- необязательно различные неприводимые многочлены).
Нам достаточно научиться присоединять корень хотя бы одного $p_i$ к полю $F$, так как корень $p_i$ автоматически корень $f$.

А теперь мы просто берем поле остатков по модулю неприводимого многочлена $p_i$, то есть $L = F[x]/(p_i)$.
Утверждение~\ref{claim::PolyQuotField}, гарантирует, что $L$ является полем.
Как мы уже отмечали элемент $x\in F[x]/(p_i)$ будет корнем $p_i$, а значит и корнем $f$.
Таким образом мы можем выбрать его за $\alpha$.
Обратим еще внимание на то, что степень $L$ над $F$ совпадает в точности со степенью многочлена $f$.


\subsection{Конечные поля}

Давайте обсудим как устроены поля состоящие только из конечного числа элементов.
Такие поля будут самыми полезными в приложениях.

\begin{claim}
Если $F$ -- конечное поле, то его характеристика не ноль.
В частности, $F$ содержит $\mathbb Z_p$, где $p = \chr F$.
\end{claim}
\begin{proof}
Так как поле $F$ конечно, то в частности это значит, что $(F, +)$ -- конечная абелева группа.
А значит любой элемент в ней имеет конечный порядок.
В частности $1$ имеет конечный порядок по сложению, а это и есть характеристика поля.
Второе утверждение следует из утверждения~\ref{claim::PrimeSubfieldCharP}.
\end{proof}

\begin{claim}
Предположим $F$ -- конечное поле и $\chr F = p$.
Тогда, $|F| = p^n$ для некоторого $n$.
Более того, можно уточнить, что $n = [F:\mathbb Z_p]$.
\end{claim}
\begin{proof}
По утверждению~\ref{claim::PrimeSubfieldCharP}, $F$ содержит $\mathbb Z_p$ как подполе.
Теперь мы рассмотрим $F$ как векторное пространство над $\mathbb Z_p$.
Так как $F$ конечно, то оно имеет и конечную размерность.
Поэтому $F$ изоморфно $\mathbb Z_p^n$ как векторное пространство.
Но, $\mathbb Z_p^n$ имеет в точности $p^n$ элементов.
Более того, мы видим, что $n$ равно размерности $F$ над $\mathbb Z_p$.
\end{proof}

\begin{claim}
Пусть $F$ -- конечное поле.
Тогда группа $F^*$ является циклической порядка $|F| - 1$.
\end{claim}
\begin{proof}
Группа $F^*$ является конечной абелевой группой.
А значит, она изоморфна группе вида $\mathbb Z_{d_1}\times \ldots \times \mathbb Z_{d_k}$, где $d_1|\ldots |d_k$ (см.~утверждение~\ref{claim::FAGClass}).
В частности для любого элемента группы $F^*$ выполнено $x^{d_k} = 1$.
Потому все элементы группы $F^*$ являются корнями многочлена $x^{d_k}-1$.
В силу единственного разложения на множители, каждый многочлен $f$ имеет не более $\deg f$ корней.
Значит, $|F^*|\leqslant d_k$.
С другой стороны $|F^*| = d_1\ldots d_k$.
Единственное, когда такое возможно -- у нас был один циклический множитель $\mathbb Z_{d_k}$.
Но это и значит, что $F^*$ изоморфно $\mathbb Z_{d_k}$ и все доказано.
\end{proof}

А вот очень важный классификационный результат для конечных полей (я не собираюсь его доказывать).

\begin{claim}
Для любого простого числа $p$ и любого натурального $n$ существует единственное с точностью до изоморфизма поле $F$ содержащее $p^n$ элементов.
\end{claim}

Так как поле из $p^n$ элементов единственно, то оно имеет специальное название $\mathbb F_{p^n}$.
Обратите внимание, что $\mathbb F_p = \mathbb Z_p$.
Однако, надо отметить, что для $n > 1$ такая конструкция не работает, то есть $\mathbb F_{p^n}\not\simeq\mathbb Z_{p^n}$, хотя бы потому что кольцо $\mathbb Z_{p^n}$ содержит делители нуля.

Хороший вопрос -- как построить все конечные поля размера $p^n$.
Прежде всего у нас есть хорошее начало $\mathbb F_p = \mathbb Z_p$.
Теперь мы должны найти неприводимый многочлен $f\in \mathbb Z_p[x]$ степени $n$.
И искомое поле строится так $\mathbb F_{p^n} = \mathbb Z_p[x]/(f)$.
Надо отметить, что количество неприводимых многочленов степени $n$ над $\mathbb Z_p$ может быть большим.
Однако, все такие многочлены дают изоморфные поля.
В частности это означает, что вы можете выбрать любой многочлен $f$ или тот, который вам удобен.

\begin{example}
Давайте построим поле $\mathbb F_4$.
Базовое поле будет $\mathbb Z_2$.
Существует только один неприводимый многочлен степени $2$ над полем $\mathbb Z_2$ -- это $x^2 + x + 1 \in \mathbb Z_2[x]$.
Тогда требуемое поле будет
\[
\mathbb F_4 = \mathbb Z_2[x]/(x^2 + x + 1) = \{a + b x\mid a, b\in \mathbb Z_2\}
\]
Сложение задается обычным покомпонентным сложением.
Чтобы посчитать произведение элементов в $\mathbb F_4$ нам достаточно уметь перемножать все степени $x$ между собой.
Произведения $1\cdot 1 = 1$ и $1\cdot x = 1$ считаются легко.
И нам лишь надо научиться считать $x\cdot x$.
По определению $x^2 = 1 + x  \pmod{x^2 + x + 1}$.
Тогда в общем виде произведение выглядит так
\[
(a+bx)(c+d x) = ac + ad x + bc x + bd x^2 = ac + ad x + bc x + bd (1+x) = ac + bd + (ad + bc + bd) x
\]
\end{example}

\begin{remarks}
\begin{itemize}
\item 
Так как поле $\mathbb Z_p$ содержится в $\mathbb F_{p^n}$, то и группа $\mathbb Z_p^*$ содержится в группе $\mathbb F_{p^n}^*$.
Мы уже отмечали в разделе~\ref{section::DiscreteLog}, что проблема дискретного логарифмирования является сложной в группе $\mathbb Z_p^*$.
Это означает, что и проблема дискретного логарифмирования является сложной в $\mathbb F_{p^n}^*$.
Действительно, если бы мы могли быстро решать уравнения $g^x = h$ по $x$ для всех $g,h \in \mathbb F_{p^n}^*$, мы могли бы решать такое и для $g, h\in \mathbb Z_p^* \subseteq \mathbb F_{p^n}^*$.
Это делает привлекательной идею использовать группу $\mathbb F_{p^n}^*$ в криптографии.
Однако, порядок этой группы достаточно далек от простого числа и это делает ее криптографически менее стойкой.
Тем не менее, с помощью конечных полей можно построить другой класс интересных групп, которые будут криптографически более стойкими, чем $\mathbb Z_p^*$.

\item
Для различных вопросов бывает полезно знать образующий группы $\mathbb F_{p^n}^*$.
Если мы строим поле как кольцо остатков вида $\mathbb Z_p[x]/(f)$, где $f\in \mathbb Z_p[x]$ -- неприводимый многочлен степени$n$, то хорошим кандидатом в качестве образующего $\mathbb F_{p^n}^*$ кажется элемент $x$.
Однако, $x$ не обязан быть образующим $\mathbb F_{p^n}^*$.
Является ли он образующим группы $\mathbb F_{p^n}^*$ зависит от выбора неприводимого многочлена $f$.
Для одних многочленов он образующий, для других -- нет.

Давайте рассмотрим следующий пример.
Над полем $\mathbb Z_3$ существует три неприводимых унитальных многочлена степени $2$: $f_1 = x^2+1$, $f_2 = x^2 + x - 1$, $f_3 = x^2 - x - 1$.
Если мы используем первый многочлен, мы получим $\mathbb F_9 = \mathbb Z_3[x]/(x^2 + 1)$.
Группа $\mathbb F_9^* \simeq \mathbb Z_8$, значит образующий должен быть порядка $8$.
Однако, $x^4 = (x^2)^2=(-1)^2 = 1$ в этом случае.
А значит, порядок $x$ равен $4$ и он не образующий.
С другой стороны, если мы возьмем $f_2$, то получим $\mathbb F_9 = \mathbb Z_3[x]/(x^2+x-1)$ или $f_3$, то получим $\mathbb F_9 = \mathbb Z_3[x]/(x^2-x-1)$.
В обоих случаях прямой подсчет показывает, что $x$ оказывается образующим группы $\mathbb F_9^*$.
\end{itemize}
\end{remarks}

\subsection{Случайный генератор Галуа}

Существует понятие регистра сдвига с линейной обратной связью.
Это специальный вид регистра производящий последовательность бит.
На его основе можно строить генераторы псевдослучайных чисел.
Существует две основные конструкции называемые в честь Фибоначчи и Галуа.
Оказывается эти понятия можно обсуждать в терминах теории конечных полей.
Я сделаю это на примере генератора Галуа.

Предположим у нас есть конечный алфавит $\mathbb Z_p = \{0,1,\ldots,p-1\}$ и мы хотим создать случайную последовательность элементов из этого алфавита.
Существует много разных способов сделать это, но вот как выглядит подход на основе конечных полей.
Давайте зафиксируем конечное поле $\mathbb F_{p^n}$.
Тогда нам надо научиться производить случайную последовательность элементов $a_1,a_2,\ldots\in \mathbb F_{p^n}$, а потом по каждому такому элементу надо научиться строить элемент $\phi(a_1),\phi(a_2),\ldots\in \mathbb Z_p$.
Оказывается, что хорошую последовательность из $a_i$ можно построить с помощью возведения в степень в конечном поле.
А чтобы научиться вычислять элементы $\mathbb Z_p$ по последовательности $a_i$, надо вспомнить, что $\mathbb F_{p^n}$ является векторным пространством над $\mathbb Z_p$.
В этом случае в качестве $\phi$ годится любая процедура по вычислению какой-нибудь координаты.
Давайте я теперь расскажу, как устроен весь процесс более детально.

Предположим $p$ -- простое число и мы фиксируем некоторое натуральное $n$.
Возьмем неприводимый многочлен $f\in \mathbb Z_p[x]$ степени $n$ и построим с помощью него поле $\mathbb F_{p^n} = \mathbb Z_p[x]/(f)$.
По определению
\[
\mathbb F_{p^n} = \{a_0 + a_1 x + \ldots + a_{n-1}x^{n-1}\mid a_i\in \mathbb Z_p\}
\]
То есть каждый элемент поля определяется последовательностью $(a_0, a_1,\ldots,a_{n-1})$.
Теперь мы выберем образующий $g\in \mathbb F_{p^n}^*$.
Это неприятная процедура, но это надо сделать единожды.
Обычно, мы выбираем $f$ так, чтобы $x$ оказался образующим.
Теперь мы производим последовательность $g, g^2, g^3, g^4,\ldots$.
Каждая степень элемента $g$ соответствует последовательности коэффициентов $a_i$ как выше.
Тогда в качестве случайного элемента из $\mathbb Z_p$ можно выбрать $a_0$.
Кроме того, мы можем начать нашу последовательность не обязательно с нулевой степени.
Мы можем начать с произвольной степени $k$: $g^k, g^{k+1}, g^{k+2}, \ldots$.
С практической точки зрения, это значит, что мы выбрали элемент $h\in \mathbb F_{p^n}^*$ и строим последовательность $hg, hg^2, hg^3, hg^4,\ldots$.
Так как $h = g^k$ для некоторого $k$, это эквивалентный подход.

\paragraph{Матричная форма}

Теперь давайте перепишем конструкцию выше в координатах.
Как и раньше, мы предполагаем, что
\[
\mathbb F_{p^n} = \mathbb Z_p[x]/(f)= \{a_0 + a_1 x + \ldots + a_{n-1}x^{n-1}\mid a_i\in \mathbb Z_p\}
\]
В этом случае $1, x, x^2,\ldots,x^{n-1}$ является базисом $\mathbb F_{p^n}$ над $\mathbb Z_p$.
Предположим, что многочлен $f = x^n + c_{n-1}x^{n-1} +\ldots + c_1 x + c_0$ выбран так, что $x$ является образующим $\mathbb F_{p^n}^*$.
Отображение $\phi \colon \mathbb F_{p^n}\to \mathbb F_{p^n}$ по правилу $h \mapsto xh$ является линейным.
Давайте напишем матрицу этого отображения в базисе из степеней $x$:
\[
\phi(1, x,\ldots, x^{n-2}, x^{n-1}) = (1, x,\ldots, x^{n-2}, x^{n-1})
\begin{pmatrix}
{0}&{0}&{\ldots}&{0}&{-c_0}\\
{1}&{0}&{\ldots}&{0}&{-c_1}\\
{0}&{1}&{\ddots}&{0}&{-c_2}\\
{\vdots}&{\vdots}&{\ddots}&{\vdots}&{\vdots}\\
{0}&{0}&{\ldots}&{1}&{-c_{n-1}}\\
\end{pmatrix}
\]
Обозначим матрицу отображения $\phi$ за $A$.
Элемент $h= a_0 + a_1 x + \ldots + a_{n-1}x^{n-1}$ в координатах описывается столбцом $v = (a_0,a_1,\ldots,a_{n-1})^t$.
Тогда элемент $xh$ описывается произведением$Av$.
Потому случайный генератор работает следующим образом.
Мы фиксируем некоторый начальный вектор $v = (a_0,a_1,\ldots,a_{n-1})^t$ и строим последовательность векторов $v, Av, A^2v, A^3v, \ldots$.
Далее у каждого вектора мы вычисляем первую координату и получаем требуемую случайную последовательность.
Существует аналогичный подход под называнием Фибоначчи.
В этом случае используется матрица $A^t$ вместо матрицы $A$.
Существует так же концептуальный подход к генератору Фибоначчи, описывающий его в терминах поля $\mathbb F_{p^n}$, но я не буду на этом заострять внимание.

\subsection{Потоковое шифрование}

Существует применение случайных генераторов в шифровании.
Давайте обсудим потоковое шифрование.
Обратите внимание, что когда вы передаете данные, то шифрование накладывает дополнительные расходы.
Потому передавать данные по шифрованному каналу может оказаться дорого, в том смысле, что мы не можем себе этого позволить делать часто.
Тем не менее, мы все же не хотим передавать по незащищенному каналу открытые данные.
Вот что можно сделать в этом случае.

Предположим, что у нас алфавит состоит из двух символов $\mathbb Z_2=\{0,1\}$.
Наше сообщение -- это последовательность бит $a_0, a_1, a_2,a_3,\ldots$.
Предположим, что у нас есть некоторый случайный генератор $S\colon \mathbb N \to \mathbb Z_2$, Тогда мы можем построить последовательность случайных бит с помощью него $s_0 = S(0), s_1 = S(1), s_2 = S(2),\ldots$.
Теперь каждый бит исходного сообщения можно зашифровать с помощью случайного бита следующим образом $d_k = a_k + s_k\pmod 2$ и передвать $d_k$ вместо $a_k$.
Чтобы восстановить сообщение, нам надо посчитать $a_k = d_k + s_k \pmod 2$.
А для этого нам надо знать в точности последовательность случайных бит $s_k$.
Вместо того, чтобы передавать последовательность $s_k$ по зашифрованному каналу, мы передадим по нему <<копию>> нашего случайного генератора, а точнее мы передадим информацию для инициализации точно такого же случайного генератора.
Тогда при наличии двух синхронизированных случайных генераторов, мы можем свободно шифровать и расшифровывать сообщения по озвученной схеме.

Давайте продемонстрирую описанную схему на диаграмме
\[
\xymatrix{
	{}&{\boxed{\text{Случайный генератор}}}\ar[d]^{s_k}\ar@{<->}[rr]^{\text{синхронизация}}&{}&{\boxed{\text{Случайный генератор}}}\ar[d]^{s_k}&{}\\
	{\boxed{\text{Отправитель}}}\ar[r]^{a_k}&{\boxed{+}}\ar[rr]^{d_k = a_k + s_k}&{}&{\boxed{+}}\ar[r]^{a_k = d_k + s_k}&{\boxed{\text{Получатель}}}\\
}
\]

Примером использования потокового шифрования является стандарт GSM.
Современный стандарт GSM использует более сложную схему шифрования, однако в основе ее лежит идея описанная выше.
